\item Demuestre que la intersección de subespacios es un subespacio, pero la unión de subespacios no.
    \ResetCases{}
    \begin{mathcase}{\(W \cap U\)}
        \begin{proof}
            Inicialmente, tomemos \(V\) es un espacio vectorial, y \(W, U\) son subespacios vectoriales de \(V\), 
            ya que un espacio vectorial necesariamente no debe ser vacío, vamos a restringir \(W, U\) no son disyuntos y \(W \neq U\). 
            Ahora, sea \(A = W \cap U\).
            Partiendo de que, por la restricción inicial, \(A\) es un conjunto no vació y es un subconjunto de \(U\) y \(W\), 
            podemos ver si es un subespacio vectorial de \(U\) o \(W\), heredando la suma entre sus vectores y producto por escalar.
            Tomando a los vectores \(u, v \in A\), vamos a ver que su suma y producto por escalar estan bien definidos en \(A\). 
            Primero, vamos a tener que \(u, v \in W\) y \(u, v \in U\), y al ser ambos conjuntos subespacios vectoriales se 
            tiene que \(u + v \in W\) y \(u + v \in U\), por lo que por definición de intersección entre conjuntos vamos a tener que \(u + v \in W \cap U = A\).
            Resultando que la suma en \(A\) esta bien definida.
            Segundo, sea \(\lambda\) un escalar real cualquiera, tenemos que \(\lambda u \in W\) y \(\lambda u \in U\) para cualquier \(\lambda\) en ambos conjuntos 
            al ser espacios vectoriales,
            y de manera similar, por la definición de intersección de conjuntos \(\lambda u \in W \cap U = A\).
            Al cumplir las propiedades clausurativas de los espacios vectoriales, el conjunto \(A\) va a ser un subespacio vectorial de \(W\) y \(U\).
        \end{proof}
    \end{mathcase}
    \begin{mathcase}{\(F \cup G\)}
        \begin{proof}
            Partiendo desde el conjunto vectorial \(\realR^3\), y 
            \[
                F = 
                \left\{
                    \begin{pmatrix}
                        x \\ y \\ z
                    \end{pmatrix}
                    :
                    x = 0
                \right\}
                \hspace{1cm}
                G = 
                \left\{
                    \begin{pmatrix}
                        x \\ y \\ z
                    \end{pmatrix}
                    :
                    z = 0
                \right\}
            \]
            subconjuntos vectoriales de \(\realR^3\), entonces podemos ver que para los vectores \(v \in F\) y \(u \in G\),
            no se cumple con la propiedad clausurativa para la suma.
            \[
                v + u =
                \begin{pmatrix}
                    0 \\ y_1 \\ z_1
                \end{pmatrix}
                +
                \begin{pmatrix}
                    x_2 \\ y_2 \\ 0
                \end{pmatrix}
                =
                \begin{pmatrix}
                    x_2 \\ y_1 + y_2 \\ z_1
                \end{pmatrix}
                \not\in F \cup G
            \]
            Como \(v + u\) no esta contenido en \(F \cup G\), vamos a tener que este no es un subespacio vectorial.
        \end{proof}
    \end{mathcase}
