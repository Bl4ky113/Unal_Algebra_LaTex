\item Encuentre un conjunto generador, un conjunto \(l.d\) y un conjunto \(l.i\) de cada uno de los siguientes espacios vectoriales.
    \begin{enumerate}[label=\listAlph]
        \item \(C = \left\{a + bi: a,b \in \realR, i = \sqrt(-1)\right\}\).
        \item El hiperplano \(H: 3x - 2y + w = 0\) de \(R^4\).
        \item \(N_A = \left\{x \in \realR^3: Ax = 0\right\}\), 
            \(
                A = \left(\begin{smallmatrix}
                    1 & -2 & 7 \\
                    0 & 1 & -2 \\
                    0 & -3 & 9
                \end{smallmatrix}\right)
            \)
        \item \(S = \bgen{3x - 2x^2, 2 + x, -4 + x - 2x^2}\)
        \item \(S = \left\{A \in \fm_{3 \times 3}: A = A^T\right\}\)
        \item \(S = \left\{A = \left(a_{ij}\right){}_{3 \times 3}: a_{ij} = 0, i \neq j\right\}\) \\
            Un conjunto generador para \(S\), podriamos tomar un subconjunto de las matrices de la base canónica de \(\fm_{3 \times 3}\)
            tal que genere a \(S\). Entonces sea \(S_g\) un conjunto generador de \(S\), definido como
            \[
                S_g =
                \left\{
                    \begin{pmatrix}
                        1 & 0 & 0 \\
                        0 & 0 & 0 \\
                        0 & 0 & 0 \\
                    \end{pmatrix},
                    \begin{pmatrix}
                        0 & 0 & 0 \\
                        0 & 1 & 0 \\
                        0 & 0 & 0 \\
                    \end{pmatrix},
                    \begin{pmatrix}
                        0 & 0 & 0 \\
                        0 & 0 & 0 \\
                        0 & 0 & 1 \\
                    \end{pmatrix},
                \right\}
            \]
            De manera sencilla podemos notar de manera sencilla que \(S_g\) es \(l.i\) y \(\pgen{S_g} \subseteq S\), ahora para cualesquiera \(\lambda_1, \lambda_2, \lambda_3\)
            escalares reales vamos a tener que para un vector \(u\) en \(S\)
            \[
                u = 
                \begin{pmatrix}
                    \lambda_1 & 0 & 0 \\
                    0 & \lambda_2 & 0 \\
                    0 & 0 & \lambda_3 \\
                \end{pmatrix}
                =
                \lambda_1
                \begin{pmatrix}
                    1 & 0 & 0 \\
                    0 & 0 & 0 \\
                    0 & 0 & 0 \\
                \end{pmatrix}
                +
                \lambda_1
                \begin{pmatrix}
                    0 & 0 & 0 \\
                    0 & 1 & 0 \\
                    0 & 0 & 0 \\
                \end{pmatrix}
                +
                \lambda_3
                \begin{pmatrix}
                    0 & 0 & 0 \\
                    0 & 0 & 0 \\
                    0 & 0 & 1 \\
                \end{pmatrix}
            \]
            el vector \(u\) es una combinación lineal de \(S_g\), es decir, esta contenido en \(\pgen{S_g}\). Así, afirmando que \(\pgen{S_g} = S\).
            \\
            Para un conjunto \(l.d\) de \(S\), definido como \(S_d\), simplemente vamos a tomar
            \[
                S_d 
                =
                \left\{
                \begin{pmatrix}
                    1 & 0 & 0 \\
                    0 & 1 & 0 \\
                    0 & 0 & 0 \\
                \end{pmatrix},
                \begin{pmatrix}
                    0 & 0 & 0 \\
                    0 & 1 & 0 \\
                    0 & 0 & 1 \\
                \end{pmatrix},
                \begin{pmatrix}
                    1 & 0 & 0 \\
                    0 & 1 & 0 \\
                    0 & 0 & 1 \\
                \end{pmatrix}
                \right\}
            \]
            Donde el tercer vector del conjunto es una combinación lineal del primero y segundo vector, entonces por definición 
            sabemos que \(S_d\) es un conjunto \(l.d\).
            \\
            Para un conjunto \(l.i\) de \(S\), definido como \(S_i\), vamos a tomar a el conjunto
            \[
                S_i =
                \left\{
                \begin{pmatrix}
                    1 & 0 & 0 \\
                    0 & 1 & 0 \\
                    0 & 0 & 0 \\
                \end{pmatrix},
                \begin{pmatrix}
                    0 & 0 & 0 \\
                    0 & 1 & 0 \\
                    0 & 0 & 1 \\
                \end{pmatrix}
                \right\}
            \]
            Vamos a verificar si es \(l.i\) viendo si los únicos escalares \(\alpha\) y \(\beta\) tales que 
            \[
                \alpha
                \begin{pmatrix}
                    1 & 0 & 0 \\
                    0 & 1 & 0 \\
                    0 & 0 & 0 \\
                \end{pmatrix}
                +
                \beta
                \begin{pmatrix}
                    0 & 0 & 0 \\
                    0 & 1 & 0 \\
                    0 & 0 & 1 \\
                \end{pmatrix}
                =
                0_3
            \]
            Son \(\alpha = \beta = 0\); para esto simplemente multiplicamos los escalares y sumamos ambas matrices, 
            entonces podemos ver que la matriz cuenta con 3 pivotes, por lo que la única solución de un sistema de ecuaciones 
            homogeneo equivalente sea la trivial.
            \[
                \alpha
                \begin{pmatrix}
                    1 & 0 & 0 \\
                    0 & 1 & 0 \\
                    0 & 0 & 0 \\
                \end{pmatrix}
                +
                \beta
                \begin{pmatrix}
                    0 & 0 & 0 \\
                    0 & 1 & 0 \\
                    0 & 0 & 1 \\
                \end{pmatrix}
                =
                \begin{pmatrix}
                    \alpha & 0 & 0 \\
                    0 & \alpha + \beta & 0 \\
                    0 & 0 & \beta \\
                \end{pmatrix}
            \]
            Cumpliendo que el conjunto de \(S_i\) es \(l.i\) por el teorema de equivalencia de conceptos.
    \end{enumerate}
