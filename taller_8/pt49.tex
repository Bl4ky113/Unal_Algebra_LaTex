\item Encuentre un conjunto generador, un conjunto \(l.d\) y un conjunto \(l.i\) de cada uno de los siguientes espacios vectoriales.
    \begin{enumerate}[label=\listAlph]
        \item \(C = \left\{a + bi: a,b \in \realR, i = \sqrt(-1)\right\}\). \\
            Podemos ver que un conjunto generador para \(C\) va a ser \(C_g = \left\{1 + 0i, 0 + i\right\}\), donde 
            \(1 + 0i\) va a generar todo número real y \(0 + i\) toda parte imaginaria. Además, de manera sencilla 
            se puede ver que el conjunto generador es \(l.i\) al tomar la matriz de coeficientes de la parte real y 
            la parte imaginaria y notar que cuenta con 2 pivotes.
            \\
            De manera similar, vamos a tener que el conjunto \(C_i = \left\{1 + 1i, i\right\}\) también es \(l.i\),
            ya que la matriz de coeficientes del conjunto va a tener 2 pivotes.
            \[
                \begin{pmatrix}
                    1 & 0 \\ 
                    0 & 1 
                \end{pmatrix}
                \hspace{1cm}
                \begin{pmatrix}
                    1 & 1 \\ 
                    0 & 1 
                \end{pmatrix}
            \]
            \\
            De forma equivalente, también vamos a tener que \(C_d = \left\{1 + i, 2 + 2i\right\}\) es un conjunto \(l.d\) 
            ya que \(2 + 2i\) es una combinación lineal, si no es que un simple producto por escalar, de \(1 + i\), con \(\lambda = 2\). 
            Sin embargo, para finalizar verifiquemos formalmente que \(\pgen{C_g} = C\). Primero, si se toma un \(v\) de \(\pgen{C_g}\),
            vamos a tener que va a ser una combinación lineal de \(\lambda_1 \cdot 1\) y \(\lambda_2 \cdot i\), lo cual por definición,
            \(v\) pertenece a \(C\), es decir \(\pgen{C_g} \subseteq C\). Segundo, de manera parecida se obtiene que \(\pgen{C_g} \supseteq C\).
        %\item El hiperplano \(H: 3x - 2y + w = 0\) de \(R^4\).
        \setcounter{enumii}{2}
        \item \(N_A = \left\{x \in \realR^3: Ax = 0\right\}\), 
            \(
                A = \left(\begin{smallmatrix}
                    1 & -2 & 7 \\
                    0 & 1 & -2 \\
                    0 & -3 & 9
                \end{smallmatrix}\right)
            \) \\
            Al ser \(N_A\) el conjunto de los vectores \(x\) tal que \(Ax = 0\), si la matriz escalonada equivalente a \(A\) cuenta con \(3\) pivotes 
            vamos a tener que el  único elemento de \(N_A\) va a ser el vector nulo, o \(\vec{0}\). Entonces, escalonemos la matriz \(A\).
            \[
                A = 
                \begin{pmatrix}
                    1 & -2 & 7 \\
                    0 & 1 & -2 \\
                    0 & -3 & 9
                \end{pmatrix}
                \sim
                \begin{aligned}
                    F_3 +3F_2 &\mapsto F_3 \\
                \end{aligned}
                \begin{pmatrix}
                    1 & -2 & 7 \\
                    0 & 1 & -2 \\
                    0 & 0 & 3
                \end{pmatrix}
            \]
            Como, mencionado anteriormente, por el \emph{Teorema de equivalencias} vamos a tener que el sistema de ecuaciones homogeneo asociado solo 
            tiene como única solución al vector nulo. Entonces \(N_A = \left\{\vec{0}\right\}\).
            Ahora, por definición sabemos el conjunto \(\left\{\vec{0}\right\}\) es \(l.d\) ya que existe un escalar tal que 
            \[
                \lambda_1 \cdot \vec{0} = \vec{0}
            \]
            Sin embargo, por este mismo argumento, al ser \(\vec{0}\) el único elemento de \(N_A\) no vamos a poder tener un conjunto \(l.i\). 
            Cabe añadir que dado a que no se tiene ningún conjunto \(l.i\) en \(N_A\), este espacio vectorial y únicamente este espacio vectorial 
            no cuenta con una base, como lo afirman en algunos textos~\cite{martinez_algebra_2024}~\cite{lipschutz_algebra_1991}. Donde además, 
            se toma como dimensión del espacio vectorial como 0. Agregando que existen textos, y definiciones que hacen que el generador de \(N_A\) 
            pueda existir, sin embargo, estos estan por fuera del programa del curso.
        %\item \(S = \bgen{3x - 2x^2, 2 + x, -4 + x - 2x^2}\)
        %\item \(S = \left\{A \in \fm_{3 \times 3}: A = A^T\right\}\)
        \setcounter{enumii}{4}
        \item \(S = \left\{A = \left(a_{ij}\right){}_{3 \times 3}: a_{ij} = 0, i \neq j\right\}\) \\
            Un conjunto generador para \(S\), podriamos tomar un subconjunto de las matrices de la base canónica de \(\fm_{3 \times 3}\)
            tal que genere a \(S\). Entonces sea \(S_g\) un conjunto generador de \(S\), definido como
            \[
                S_g =
                \left\{
                    \begin{pmatrix}
                        1 & 0 & 0 \\
                        0 & 0 & 0 \\
                        0 & 0 & 0 \\
                    \end{pmatrix},
                    \begin{pmatrix}
                        0 & 0 & 0 \\
                        0 & 1 & 0 \\
                        0 & 0 & 0 \\
                    \end{pmatrix},
                    \begin{pmatrix}
                        0 & 0 & 0 \\
                        0 & 0 & 0 \\
                        0 & 0 & 1 \\
                    \end{pmatrix},
                \right\}
            \]
            De manera sencilla podemos notar de manera sencilla que \(S_g\) es \(l.i\) y \(\pgen{S_g} \subseteq S\), ahora para cualesquiera \(\lambda_1, \lambda_2, \lambda_3\)
            escalares reales vamos a tener que para un vector \(u\) en \(S\)
            \[
                u = 
                \begin{pmatrix}
                    \lambda_1 & 0 & 0 \\
                    0 & \lambda_2 & 0 \\
                    0 & 0 & \lambda_3 \\
                \end{pmatrix}
                =
                \lambda_1
                \begin{pmatrix}
                    1 & 0 & 0 \\
                    0 & 0 & 0 \\
                    0 & 0 & 0 \\
                \end{pmatrix}
                +
                \lambda_1
                \begin{pmatrix}
                    0 & 0 & 0 \\
                    0 & 1 & 0 \\
                    0 & 0 & 0 \\
                \end{pmatrix}
                +
                \lambda_3
                \begin{pmatrix}
                    0 & 0 & 0 \\
                    0 & 0 & 0 \\
                    0 & 0 & 1 \\
                \end{pmatrix}
            \]
            el vector \(u\) es una combinación lineal de \(S_g\), es decir, esta contenido en \(\pgen{S_g}\). Así, afirmando que \(\pgen{S_g} = S\).
            \\
            Para un conjunto \(l.d\) de \(S\), definido como \(S_d\), simplemente vamos a tomar
            \[
                S_d 
                =
                \left\{
                \begin{pmatrix}
                    1 & 0 & 0 \\
                    0 & 1 & 0 \\
                    0 & 0 & 0 \\
                \end{pmatrix},
                \begin{pmatrix}
                    0 & 0 & 0 \\
                    0 & 1 & 0 \\
                    0 & 0 & 1 \\
                \end{pmatrix},
                \begin{pmatrix}
                    1 & 0 & 0 \\
                    0 & 1 & 0 \\
                    0 & 0 & 1 \\
                \end{pmatrix}
                \right\}
            \]
            Donde el tercer vector del conjunto es una combinación lineal del primero y segundo vector, entonces por definición 
            sabemos que \(S_d\) es un conjunto \(l.d\).
            \\
            Para un conjunto \(l.i\) de \(S\), definido como \(S_i\), vamos a tomar a el conjunto
            \[
                S_i =
                \left\{
                \begin{pmatrix}
                    1 & 0 & 0 \\
                    0 & 1 & 0 \\
                    0 & 0 & 0 \\
                \end{pmatrix},
                \begin{pmatrix}
                    0 & 0 & 0 \\
                    0 & 1 & 0 \\
                    0 & 0 & 1 \\
                \end{pmatrix}
                \right\}
            \]
            Vamos a verificar si es \(l.i\) viendo si los únicos escalares \(\alpha\) y \(\beta\) tales que 
            \[
                \alpha
                \begin{pmatrix}
                    1 & 0 & 0 \\
                    0 & 1 & 0 \\
                    0 & 0 & 0 \\
                \end{pmatrix}
                +
                \beta
                \begin{pmatrix}
                    0 & 0 & 0 \\
                    0 & 1 & 0 \\
                    0 & 0 & 1 \\
                \end{pmatrix}
                =
                0_3
            \]
            Son \(\alpha = \beta = 0\); para esto simplemente multiplicamos los escalares y sumamos ambas matrices, 
            entonces podemos ver que la matriz cuenta con 3 pivotes, por lo que la única solución de un sistema de ecuaciones 
            homogeneo equivalente sea la trivial.
            \[
                \alpha
                \begin{pmatrix}
                    1 & 0 & 0 \\
                    0 & 1 & 0 \\
                    0 & 0 & 0 \\
                \end{pmatrix}
                +
                \beta
                \begin{pmatrix}
                    0 & 0 & 0 \\
                    0 & 1 & 0 \\
                    0 & 0 & 1 \\
                \end{pmatrix}
                =
                \begin{pmatrix}
                    \alpha & 0 & 0 \\
                    0 & \alpha + \beta & 0 \\
                    0 & 0 & \beta \\
                \end{pmatrix}
            \]
            Cumpliendo que el conjunto de \(S_i\) es \(l.i\) por el teorema de equivalencia de conceptos.
    \end{enumerate}
