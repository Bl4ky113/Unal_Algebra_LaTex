\item Sabiendo que \(A_T= \fsmmatrix{1,0,-1;0,2,1}\) es la matriz asociada a la transformación lineal \(T: V \rightarrow W\), 
    en las bases \(\fb\) y \(\fb'\) de \(V\) y \(W\) respectivamente, 
    encuentre el núcleo y la imagen de cada una de las siguientes situaciones y diga cuáles son sus dimensiones.
    \begin{enumerate}[label=\listAlph]
        \item \(V = \realR^3, W = \realR^2\), \(\fb\) y \(\fb'\) son sus bases usuales. \\
            Inicialmente, podemos ver que la nulidad y rango de \(A_T\) es 1 y 2 respectivamente, 
            ya que al ser una matriz escalonada y contar con dos columnas con pivotes y una variable libre, 
            por el teorema de equivalencias vamos a tener que \(\nu(A_T) = 1\) y \(\rho(A_T) = 2\).
            A continuación, para encontrar la imagen y kernel de \(T\) vamos a usar el teorema de la caracterización de la matriz asociada a una transformación lineal, 
            \(\veccoord{T(v)}{\fb'} = A_T\veccoord{v}{\fb}\) para cualquier vector \(v\) en \(V\). Partiendo desde un vector cualquiera, \(v\), de \(V\), 
            se sabe que es una combinación lineal de los vectores de \(\fb\), el cual es la base canónica en \(\realR^3\), entonces
            \[
                v = \lambda_1 \fvec{1,0,0} + \lambda_2 \fvec{0,1,0} + \lambda_3 \fvec{0,0,1} = \fvec{\lambda_1, \lambda_2, \lambda_3} 
                \hspace{0.5cm}
                \veccoord{v}{\fb} = \fvec{\lambda_1,\lambda_2,\lambda_3}
            \]
            Ahora, por el teorema mencionado, vamos a tener que \(\veccoord{T(v)}{\fb'} = A_T\veccoord{v}{\fb}\), es decir
            \[
                \veccoord{T(v)}{\fb'}
                = \fmatrix{1,0,-1;0,2,1} \fvec{\lambda_1, \lambda_2, \lambda_3}
                = \fvec{\lambda_1 - \lambda_3, 2\lambda_2 + \lambda_3}
            \]
            Apartir de esto vamos a poder encontrar el kernel de \(T\) con el siguiente sistema de ecuaciones, apartir de cualquier vector \(a\) en \(v\)
            se va a tener que pertence al kernel de \(T\) si
            \[
                A_T \veccoord{a}{\fb} = \fmatrix{1,0,-1;0,2,1} \fvec{\alpha_1, \alpha_2, \alpha_3} =  \vec{0} \in \realR^2
            \]
            Como \(A_T\) ya esta escalonada, mediante \emph{Gauss}, vamos a tener que \(\alpha_1 = \alpha_3\) y además \(\alpha_3 = -2\alpha_2\), 
            es decir obtenemos el vector \(\fsmvec{-2\alpha_2, \alpha_2, -2\alpha_2}\) como solución al sistema; entonces vamos a tener que el kernel de \(T\) va a ser 
            \[
                \fker{T} = \bgen{\fvec{-2,1,-2}}
            \]
            Ahora, veamos la imagen de \(T\) apartir de \(\veccoord{T(v)}{\fb'}\), el cual va a ser igual a
            \[
                \veccoord{T(v)}{\fb'} 
                = \fvec{\lambda_1 - \lambda_3, 2\lambda_2 + \lambda_3}
                = \fvec{\lambda_1,0} + \fvec{0,2\lambda_2}, + \fvec{-\lambda_3, \lambda_3}
                = \lambda_1 \fvec{1,0} + \lambda_2\fvec{0,2} + \lambda_3 \fvec{-1,1}
            \]
            Como \(\fsmvec{-1,1}\) es una combinación lineal de \(\fsmvec{1,0}\) y \(\fsmvec{0,2}\) con los escalares \(-1\) y \(\frac{1}{2}\) respectivamente,
            vamos a poder eliminarlo para obtener un conjunto de vectores \(l.i\). Además el generador de estos vectores, va a generar todo vector \(T(v)\) y 
            al tener una dimensión de 2 y ser elementos de \(\realR^2\), el cual su dimensión es de 2, vamos a tener que 
            \[
                \fimg{T} = \bgen{\fvec{1,0}, \fvec{0,2}} = \realR^2;
                \hspace{0.25cm}
                \text{para cualquier } v \in V
            \]
            Concluyendo que la dimensión del kernel y la dimensión de la imagen de \(T\) es 1 y 2 respectivamente.
        \item \(V = \realR^3, W = \realR^2, \fb = \fset{\fsmvec{1,0,0}, \fsmvec{1,1,0}, \fsmvec{1,1,1}} \) y \(\fb'\) es la base usual de \(\realR^2\). \\
            De manera similar al punto anterior, tomemos un vector \(v\) cualquiera de \(V\) y encontremos a \(\veccoord{v}{\fb}\)
            \[
                v = \lambda_1 \fvec{1,0,0} + \lambda_2 \fvec{1,1,0} + \lambda_3 \fvec{1,1,1} = \fvec{\lambda_1 + \lambda_2 + \lambda_3, \lambda_2 + \lambda_3, \lambda_3} = \veccoord{v}{\fb}
            \]
            ahora, vamos encontrar el kernel apartir del sistema de ecuaciones \(A_T \veccoord{v}{\fb} = \vec{0}\)
            \[
                A_T \veccoord{v}{\fb} = 
                \fmatrix{1,0,-1;0,2,1} 
                \fvec{\lambda_1 + \lambda_2 + \lambda_3, \lambda_2 + \lambda_3, \lambda_3} =
                \fvec{\lambda_1 + \lambda_2, 2\lambda_2 + 3\lambda_3} =
                \fvec{0,0}
                \hspace{0.25cm}
                \begin{cases}
                    \lambda_1 + \lambda_2 = 0 \\
                    2\lambda_2 + 3\lambda_3 = 0
                \end{cases}
            \]
            Entonces, mediante sustitución hacia atrás se obtiene que \(\lambda_1 = -\lambda_2\) y \(\lambda_2 = -\frac{3}{2}\lambda_3\),
            entonces se tiene que el vector \(T(v)\) es 
            \[
                \fvec{\lambda_1 + \lambda_2 + \lambda_3, \lambda_2 + \lambda_3, \lambda_3} =
                \fvec{\lambda_3, -\frac{1}{2}\lambda_3, \lambda_3} =
                \lambda_3 \fvec{1, -\frac{1}{2}, 1}
            \]
            Teniendo así al kernel de \(T\).
            \[
                \fker{T} = \bgen{\fvec{1,-\frac{1}{2},1}}
            \]
            Ahora, con \(\veccoord{v}{\fb}\) vamos a obtener a la imagen de \(v\) en \(T\) usando el teorema de caracterización de la matriz asociada a una transformación lineal
            \[
                \veccoord{T(v)}{\fb'} = 
                A_T \veccoord{v}{\fb} = 
                \fmatrix{1,0,-1;0,2,1} 
                \fvec{\lambda_1 + \lambda_2 + \lambda_3, \lambda_2 + \lambda_3, \lambda_3} =
                \fvec{\lambda_1 + \lambda_2, 2\lambda_2 + 3\lambda_3} =
                \lambda_1\fvec{1, 0} + \lambda_2\fvec{1,2} + 3\lambda_3\fvec{0,1}
            \]
            Como \(\fsmvec{1,2}\) es una combinación lineal de los otros vectores, con \(\lambda_1 = 1\) y \(lambda_3 = \frac{2}{3}\), 
            lo vamos a poder remover para poder tener un conjunto \(l.i\) de dimensión 2, es decir, esté va a generar a \(\realR^2\) y a todo 
            vector \(T(v)\) donde \(v\) es cualquier vector de \(V\) como inicalmente se había planteado. Es decir
            \[
                \fimg{T} = 
                \bgen{\fvec{1,0}, \fvec{0,1}} =
                \realR^2
            \]
            Concluyendo que la dimensión del kernel y imagen de \(T\) es de 1 y 2 respectivamente.
            %\item \(V = \realR^3, W = \realR^2\), \(\fb\) es la base usual y \(\fb' = \fset{\fsmvec{1,-1}, \fsmvec{1,2}}\).
        \setcounter{enumii}{3}
        \item \(V = \fp_2, W = \fp_1\), \(\fb\) y \(\fb'\) son sus bases usuales. \\
            Como en los puntos anteriores,  tomando un polinomio \(v = \lambda_1 + \lambda_2x + \lambda_3x^2\) de \(\fp_2\) cualquiera 
            y como \(\fb\) es la base canónica de \(\fp_2\), vamos a tener que \(\veccoord{v}{\fb} = \fsmvec{\lambda_1,\lambda_2,\lambda_3}\).
            Con este vector coordenada vamos a obtener al kernel de \(T\) con el sistema de ecuaciones \(A_T \veccoord{v}{\fb} = \veccoord{\vec{0}}{\fb'} \in \fp_1\)
            \[
                \fmatrix{1,0,-1;0,2,1} \fvec{\lambda_1,\lambda_2,\lambda_3} 
                = \fvec{\lambda_1 - \lambda_3, 2\lambda_2 + \lambda_3}
                = \vec{0} \in \fp_2
                \hspace{0.5cm}
                \begin{cases}
                    \lambda_1 - \lambda_3 = 0 \\
                    2\lambda_2 + \lambda_3 = 0
                \end{cases}
            \]
            Reemplazando y sustituyendo hacia atras se puede ver que \(\lambda_1 = \lambda_3\) y \(\lambda_3 = -2\lambda_2\), es decir el vector \(\veccoord{v}{\fb}\)
            se puede reescribir como, también como este vector es la solución del sistema \(A_T \veccoord{v}{\fb} = \veccoord{\vec{0}}{\fb'}\), entonces va a ser el conjunto que 
            genera el kernel de \(T\).
            \[
                \veccoord{v}{\fb} = \fvec{-2\lambda_2,\lambda_2,-2\lambda_2} = \lambda_2 \fvec{-2,1,-2}
                \hspace{1cm}
                \fker{T} = \bgen{-2+x-2x^2}
            \]
            Además notando que la dimensión del kernel es de 1. \\
            Ahora, veamos la imagen de \(T\) apartir del vector \(\veccoord{T(v)}{\fb'}\) teniendo en cuenta que \(v\) es un vector cualquiera de \(V\).
            \[
                \veccoord{T(v)}{\fb'}
                = \fmatrix{1,0,-1;0,2,1} \fvec{\lambda_1,\lambda_2,\lambda_3} 
                = \fvec{\lambda_1 - \lambda_3, 2\lambda_2 + \lambda_3}
                = \lambda_1 \fvec{1,0} + \lambda_2 \fvec{0,2} + \lambda_3 \fvec{-1,1}
            \]
            Obteniendo así un conjunto que genere a todos los vectores de coordenadas respecto a \(\fb'\) de la imagen de \(T\), o \(\veccoord{T(v)}{\fb'}\), 
            sin embargo, este no es base ya que \(\fsmvec{-1,1}\) es una combinación lineal de los 
            otros vectores tomando los escalares \(\lambda_1 = -1\) y \(\lambda_2 = \frac{1}{2}\).
            Entonces al considerar el conjunto de los vectores \(\fsmvec{1,0}\) y \(\fsmvec{0,2}\) vamos a 
            tener que es \(l.i\) y genera a \(T(v)\), es decir es una base de la imagen de \(T\). 
            Teniendo que la dimensión de \(\fimg{T}\) es 2, 
            \[
                \fimg{T} = \bgen{1, 2x} = \fp_1
            \]
        %\item \(V = \fp_2, W = \fp_1\), \(\fb = \fset{1-x,x-x^2,x^2+1}\) y \(\fb' = \fset{1-x,x}\).
    \end{enumerate}
