\item Si \(V = \fp{}_2\) es el espacio de todos los polinomios de grado menor o igual a 2,
    \begin{enumerate}[label=\listAlph]
        \item ¿Es \(H = \{p\left(x\right) = a + bx + cx^2 \in \fp{}_2 : a = −2\}\) un subespacio vectorial de \(\fp{}_2\)? \\
            No, ya que para verificar si es un subespacio de \(\fp_2\) primero tendriamos que mirar si la suma entre polinomios es 
            clausurativa, y podemos ver que con \(p(x) = -2 + bx + cx^2\) y \(q(x) = -2 + dx + ex^2\) se tiene que 
            \[
                p(x) + q(x)
                =
                (-2 -2) + (b + d)x + (c + e)x^2
                =
                -4 + (b + d)x + (c + e)x^2
                =
                o(x)
            \]
            Y como \(o(x) \not\in H\) ya que \(a \neq -2\), entonces \(H \not\lesssim \fp_2\).\footnote{{La notación \(A \lesssim B\) se usa para determinar que un subconjunto \(A\) es un subespacio vectorial de un conjunto \(B\).}}
        \item ¿Es verdad que si \(B\) es un conjunto de 3 polinomios, entonces \(B\) genera a \(\fp_2\)? \\
            No necesariamente, para que \(B\) genere a \(\fp_2\) es necesario que \(B\) sea \(l.i\) y, aunque redundante, 
            que \(B\) genere a \(\fp_2\) lo cual es tener una doble contenencia. 
            Además se tiene que añádir que el generador de \(B\) tiene la misma dimensión que cualquier generador de \(\fp_2\),
            se debe añadir que un conjunto \(B\) no \(l.i\) no va a ser de la misma dimensión de cual\(\fp_2\), ya que 
            el generador sería equivalente a un generador con un conjunto generador de 2 o solo 1 polinomio. 
            Sin embargo, en general y como mencionado anteriormente, que \(B\) tenga 3 polinomios no implica que \(\pgen{B} = \fp_2\).
        \item ¿Es verdad que cualquier subconjunto de \(V\) con 2 o menos polinomios es linealmente independiente? \\
            Falso, ya que puede exitir del subconjunto de \(V\), \(v\), que sus elementos sean un polinomio cualquiera
            y un multiplo escalar de ese polinomio, es decir
            \[
                v = \left\{p(x), \lambda p(x)\right\}; \hspace{1cm} \text{para algún escalar real } \lambda 
            \]
        \item ¿Es verdad que si \(B\) genera a \(\fp_2\), entonces el numero mínimo de elemento de \(B\) es 3? \\
            Verdadero, por definición de dimensión, el número de elementos de un conjunto generador de un espacio vectorial debe 
            ser igual para todos los conjuntos generadores del espacio vectorial. 
            Y podemos ver que son necesarios 3 elementos para generar a \(\fp_2\), por ejemplo \(\left\{a_0, a_1x, a_2x^2\right\}\), 
            los polinomios desde el grado 0 hasta 2 es un conjunto generador de \(\fp_2\). 
            Entonces, es totalmente verdadero que el mínimo número, aunque verdaderamente único, número de elementos que \(B\)
            puede tener para ser un conjunto generador de \(\fp_2\) es 3.
    \end{enumerate}
