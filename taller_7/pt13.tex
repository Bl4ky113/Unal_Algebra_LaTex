\item Determine si los siguientes conjuntos son subespacios vectoriales (reales). Identifique el espacio vectorial al que pertenecen.
    \begin{enumerate}[label=\listAlph]
        \setcounter{enumii}{1}
        \item El conjunto de matrices simétricas \(8 \times 8\) \\
            Primero, debemos saber que el conjunto de matrices de orden 8 o \(\fm_8\)\footnote{{La notación \(\fm_n, \fs_n, \ft_n\) se usa para determinar matrices de orden \(n\)}}, es un espacio vectorial. 
            Y como las matrices simétricas de orden 8, o \(\fs_8\), son un subconjunto de \(\fm_8\), podriamos confirmar que son un subespacio vectorial 
            verificando la propiedad clausurativa para la suma y producto escalar.\\
            Partiendo de las matrices simétricas \(A\) y \(B\), ambas de orden 8,
            se tiene que \(A = A^T\) y \(B = B^T\), y además sea \(\lambda\) un escalar real cualescualquiera.
            Entonces, tenemos que \(A + B = A^T + B^T = \left(A + B\right){}^T\), por propiedades de la transpuesta de una matriz.
            Ahora, veamos que \(\lambda A = \lambda A^T\). Concluyendo que \(\left(A + B\right){}^T, \lambda A^T \in \fs_8\), 
            y entonces tenemos que \(\fs_8 \lesssim \fm_8\).
        \setcounter{enumii}{10}
        \item El conjunto de números racionales. \\
            Primero, podriamos definir este conjunto de la siguiente manera
            \[
                \realQ = \left\{ \frac{p}{q} \mid p \in \realZ, q \in \realZ \setminus {0} \right\}
            \]
            y directamente podemos ver que falla la propiedad clausurativa de la multiplicación de los escalares \textbf{Reales}.
            Ya que, tomando cualquier \(m \in \realQ\) tal que \(m = \frac{p}{q}\) con \(p \in \realZ\), 
            \(q \in \realZ \setminus {0}\) fijos pero arbitrarios y \(\lambda\) un escalar real cualesquiera. 
            Tenemos que
            \[
                \lambda \cdot m = \frac{\lambda}{1} \cdot \frac{p}{q} = \frac{\lambda p}{q}
            \]
            Y podemos ver que, como \(\lambda p \not \in \realZ\) entonces \(\lambda m \not\in \realQ\).
            Es decir, \(\realQ\) no es un espacio vectorial \textbf{Real}.
        \setcounter{enumi}{13}
        \item \(W = \left\{ A \in \fm_n: A^T = A \right\}\) \\
            Inicialmente, podemos notar que \(\fm_n\) es un espacio vectorial.
            Y como toda matriz símetrica de orden \(n\) esta en \(\fm_n\) 
            podriamos mirar si \(W\) es un subespacio vectorial de \(\fm_n\), 
            por lo que solo debemos verificar las propiedades clausurativa de la suma y multiplicación escalar en \(W\). \\
            Tomando a \(A, B \in W\) cualesquiera y \(\lambda\) un escalar real fijo pero arbitrario, podemos ver que 
            \[
                A + B = A^T + B^T = \left(A + B\right){}^T \in W
            \]
            \[
                \lambda B = \lambda B^T = \left(\lambda B\right){}^T \in W
            \]
            Concluyendo que \(W \lesssim \fm_n\).
        \item El conjunto de matrices triangulares superiores \(3 \times 3\) \\
            Antes de iniciar, podemos definir el conjunto dado como
            \[
                \ft_3 = \left\{A \in \fm_3: a_{ij} = 0, i > j\right\}
            \]
            Como toda matriz triángular superior de orden 3 estan son elementos de el conjunto de matrices de orden 3, 
            tenemos que \(\ft_3 \subset \fm_3\), entonces para ver que \(\ft_3\) es un subespacio vectorial de \(\fm_3\)
            simplemente debemos verificar que se cumplan las propiedades clausurativas.
            \\
            Ahora, sean culquier \(A, B \in \ft_3\) y \(\lambda\) un escalar real cualesquiera. 
            Se puede realizar la suma y producto escalar entre cada componente de \(A\), \(B\) y \(\lambda\) para verificar que se cumpla la propiedad.
            \[
                A + B = 
                \begin{pmatrix}
                    a_{11} & a_{12} & a_{13} \\
                    0 & a_{22} & a_{23} \\
                    0 & 0 & a_{33}
                \end{pmatrix}
                +
                \begin{pmatrix}
                    b_{11} & b_{12} & b_{13} \\
                    0 & b_{22} & b_{23} \\
                    0 & 0 & b_{33}
                \end{pmatrix}
                =
                \begin{pmatrix}
                    a_{11} + b_{11} & a_{12} + b_{12} & a_{13} + b_{13} \\
                    0 & a_{22} + b_{22} & a_{23} + b_{23} \\
                    0 & 0 & a_{33} + b_{33}
                \end{pmatrix}
            \]
            \[
                \lambda \cdot A =
                \lambda \cdot 
                \begin{pmatrix}
                    a_{11} & a_{12} & a_{13} \\
                    0 & a_{22} & a_{23} \\
                    0 & 0 & a_{33}
                \end{pmatrix}
                =
                \begin{pmatrix}
                    \lambda a_{11} & \lambda a_{12} & \lambda a_{13} \\
                    \lambda 0 & \lambda a_{22} & \lambda a_{23} \\
                    \lambda 0 & \lambda 0 & \lambda a_{33}
                \end{pmatrix}
                \begin{pmatrix}
                    \lambda a_{11} & \lambda a_{12} & \lambda a_{13} \\
                    0 & \lambda a_{22} & \lambda a_{23} \\
                    0 & 0 & \lambda a_{33}
                \end{pmatrix}
            \]
            Podemos ver que por la definición de una matriz triángular superior tenemos que 
            \(A + B, \lambda A \in \ft_3\), entonces tenemos que \(\ft_3 \lesssim \fm_3\).
        \item El conjunto de funciones de \(\realR \mapsto \realR\), derivables.
            Inicialmente, definimos el conjunto dado como
            \[
                \ff = \left\{ f: \realR \mapsto \realR \mid \exists f'\right\}
            \]
            Como cualquier función definida en \(\realR\) a \(\realR\) derivable esta contenida en el 
            conjunto de las funciones \(\realR \mapsto \realR\), el cual es un espacio vectorial. 
            Por lo que solo es necesario tener que verificar que \(\ff\) cumple con las propiedades clausurativas 
            de la suma y producto escalar para que esté sea un subespacio vectorial del conjunto de las funciones \(\realR \mapsto \realR\).
            \\
            Ahora, sean funciones \(f, g \in \ff\) cualesquiera y \(\gamma\) un escalar real cualquiera,
            se puede ver que 
            \[
                f' + g' = \left(f + g\right)'
            \]
            \[
                \gamma f' = \left(\gamma f\right)'
            \]
            Como \(f + g, \gamma f \in \ff\) vamos a tener que \(\ff\) es un subespacio vectorial del conjunto de las funciones \(\realR \mapsto \realR\).
    \end{enumerate}
