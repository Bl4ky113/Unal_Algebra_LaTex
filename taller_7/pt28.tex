\item Determinar si los siguientes conjuntos son subespacios del espacio vectorial \(\fm_2\)
    \begin{enumerate}[label=\listAlph]
        \item \(W = \left\{ A \in \fm_2 : A = A^T\right\}\) \\
            Inicialmente, podemos notar que \(\fm_2\) es un espacio vectorial.
            Y como toda matriz símetrica de orden \(2\) esta en \(\fm_2\) 
            podriamos mirar si \(W\) es un subespacio vectorial de \(\fm_2\), 
            por lo que solo debemos verificar las propiedades clausurativa de la suma y multiplicación escalar en \(W\). \\
            Tomando a \(A, B \in W\) cualesquiera y \(\lambda\) un escalar real fijo pero arbitrario, podemos ver que 
            \[
                A + B = A^T + B^T = \left(A + B\right){}^T \in W
            \]
            \[
                \lambda B = \lambda B^T = \left(\lambda B\right){}^T \in W
            \]
            Concluyendo que \(W \lesssim \fm_2\).
        \item \(W = \left\{ \begin{pmatrix}a & b \\ c & d\end{pmatrix}: a = d \right\}\) \\
            Inicialmente, podemos notar que \(\fm_2\) es un espacio vectorial.
            Y como toda matriz perteneciente a \(W\) esta en \(\fm_2\), 
            con esta información, para verificar que \(W\) es un subespacio vectorial de \(\fm_2\)
            se debe verificar las propiedades clausurativas de la suma y producto escalar. \\
            Tomando a \(A, B \in W\) cualesquiera y \(\lambda\) un escalar real fijo pero cualquiera, 
            podemos operar componente a componente, simplificamos \(m_{11} = m_{22}\) por la definición de las matrices en \(W\)
            \[
                A + B =
                \begin{pmatrix}
                    a_{11} & a_{12} \\
                    a_{21} & a_{11}
                \end{pmatrix}
                +
                \begin{pmatrix}
                    b_{11} & b_{12} \\
                    b_{21} & b_{11}
                \end{pmatrix}
                =
                \begin{pmatrix}
                    a_{11} + b_{11} & a_{12} + b_{12} \\
                    a_{21} + b_{21} & a_{11} + b_{11}
                \end{pmatrix}
            \]
            \[
                \lambda A =
                \lambda
                \begin{pmatrix}
                    a_{11} & a_{12} \\
                    a_{21} & a_{11}
                \end{pmatrix}
                =
                \begin{pmatrix}
                    \lambda a_{11} & \lambda a_{12} \\
                    \lambda a_{21} & \lambda a_{11}
                \end{pmatrix}
            \]
            Como \(A + B, \lambda A \in W\), entonces vamos a tener que \(W \lesssim \fm_2\)
        \item \(W = \left\{ A \in \fm_2 : \pdet{A} = 0 \right\}\) \\
            Se puede ver que \(W \subset \fm_2\), sin embargo, con la misma rápidez también se puede notar que \(W \not\lesssim \fm_2\).
            Ya que, no se cuenta con la propiedad clausurativa, necesaria para que \(W\) sea un subespacio vectorial de \(\fm_2\), vease que
            con las matrices \(A, B \in W\)
            \[
                \begin{aligned}
                    A &= 
                    \begin{pmatrix}
                        1 & 0 \\
                        0 & 0
                    \end{pmatrix}
                    \\
                    \pdet{A} &= 0
                \end{aligned}
                \hspace{1cm}
                \begin{aligned}
                    B &= 
                    \begin{pmatrix}
                        0 & 0 \\
                        0 & 1
                    \end{pmatrix}
                    \\
                    \pdet{B} &= 0
                \end{aligned}
            \]
            se tiene que
            \[
                A + B 
                =
                \begin{pmatrix}
                    1 & 0 \\
                    0 & 0
                \end{pmatrix}
                +
                \begin{pmatrix}
                    0 & 0 \\
                    0 & 1
                \end{pmatrix}
                =
                \begin{pmatrix}
                    1 & 0 \\
                    0 & 1
                \end{pmatrix}
                \hspace{1cm}
                \pdet{A + B} = 1
            \]
            Como \(\pdet{A + B} \neq 0\) entonces \(A + B \not \in W\) y se concluye que \(W \not\lesssim \fm_2\).
    \end{enumerate}
