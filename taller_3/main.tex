\documentclass{article}

\def\MyClass{Álgebra Lineal}
\def\MyTitle{Taller 3}
\def\MyAuthor{Martín Steven Hernández Ortiz}
\def\MyEmail{mahernandezor@unal.edu.co}
\def\MyDate{\today}

\def\TemplatePath{../template/}

%%% Template Packages %%%

\usepackage{graphicx} % Images
\usepackage{tcolorbox} % Color Box
\usepackage[%
vmargin=2.25cm,%
hmargin=2.25cm%
]{geometry} % Page Geometry
\usepackage{fancyhdr} % Header / Footer Styles

\usepackage{ragged2e} % Text Align
\usepackage{amsmath} % Math Align

\usepackage{booktabs} % Tables Package

\usepackage{mathtools} % Math General
\usepackage{amsthm} % Math Envs
\usepackage{unicode-math} % Math Symbols

\usepackage{polyglossia} % Language
    \setdefaultlanguage{spanish}

\renewcommand{\thefootnote}{\Roman{footnote}} % Changing footnotes arabic to roman numbers

%%% Template Styles %%%

% Header / Footer Styles
\pagestyle{fancy}
\RenewDocumentCommand{\headrule}{}{%
    \rule[0.1cm]{\textwidth}{0.1mm}%
}
\RenewDocumentCommand{\footrule}{}{%
    \rule[0.1cm]{\textwidth}{0.1mm}%
}

\fancyhf[HC]{{\slshape \MyTitle{}}}

% Redefine \maketitle
\RenewDocumentCommand{\maketitle}{s}{%
    \begin{@twocolumntrue}%
        \begin{minipage}{0.3\textwidth}%
            \includegraphics[width=0.85\textwidth]{../template/src/unal_logo.pdf}%
        \end{minipage}%
        \begin{minipage}{0.7\textwidth}{{%
        \begin{Center}%
            {\large \itshape \MyClass{}} \\[1ex]
            {\huge  \slshape \MyTitle{}} \\[4ex]
            {\Large  \MyAuthor{}} \\[0ex]
            {\small  \MyEmail{}} \\[4ex] 
            \MyDate{}
        \end{Center}%
        }}%
        \end{minipage}%
    \end{@twocolumntrue}%
    \vspace{0.5cm}%
    \begin{Center}%
        \rule[0cm]{\textwidth}{0.1mm}%
    \end{Center}%
    \vspace{0.5cm}%
}

%%% Template Math stuff %%%

% Theorems
\newtheorem{TMPMathTheorem}{Teorema}
\NewDocumentEnvironment{theorem}{+b} {%
    \begin{TMPMathTheorem}%
        #1 %
    \end{TMPMathTheorem}%
} {}

% Collorary
\newtheorem{TMPMathCorollary}[TMPMathTheorem]{Corolario}
\NewDocumentEnvironment{corollary}{+b} {%
    \begin{TMPMathCorollary}%
        #1 %
    \end{TMPMathCorollary}%
} {}

% Definitions
\newtheorem{TMPMathDefinition}[TMPMathTheorem]{Definición}
\NewDocumentEnvironment{definition}{+b} {%
    \begin{tcolorbox}[left=0mm,right=0mm]%
        \begin{TMPMathDefinition}%
            #1 %
            \begin{FlushRight}%
                \(\bigtriangleup{}\)%
            \end{FlushRight}%
        \end{TMPMathDefinition}%
    \end{tcolorbox}%
} {}



\def\fancyL{\symscr{L}}
\def\fancyP{\symscr{P}}
\def\realR{\symbb{R}}
\def\realN{\symbb{N}}

\begin{document}
\maketitle
\begin{enumerate}
    \item Un punto \(P\) pertenece a la recta que pasa por los puntos \(\left(\begin{smallmatrix}4 \\ 2 \\ 2\end{smallmatrix}\right)\) y \(\left(\begin{smallmatrix}-2 \\ 0 \\ 6\end{smallmatrix}\right)\), si la coordenada \(y\) de \(P\) es 1, hallar sus otras coordenadas. \\
        Primero definamos una ecuación vectorial de la recta que pasa por \(\left(\begin{smallmatrix}4 \\ 2 \\ 2\end{smallmatrix}\right)\) y \(\left(\begin{smallmatrix}-2 \\ 0 \\ 6\end{smallmatrix}\right)\)        
        \[
            \begin{aligned}
                A &= \begin{pmatrix}4 \\ 2 \\ 2\end{pmatrix} \\
                B &= \begin{pmatrix}-2 \\ 0 \\ 6\end{pmatrix} 
            \end{aligned}
            \hspace{0.5cm}
            \text{;}
            \hspace{0.5cm}
            \overline{AB} = \begin{pmatrix}-6 \\ -2 \\ 4\end{pmatrix}
            \hspace{0.5cm}
            \text{;}
            \hspace{0.5cm}
            \fancyL :
            \begin{pmatrix}x \\ y \\ z\end{pmatrix}
                =
            \begin{pmatrix}4 \\ 2 \\ 2\end{pmatrix}
                +
            t \cdot \begin{pmatrix}-6 \\ -2 \\ 4\end{pmatrix}
        \]
        Ahora, si \(P\) es un punto que pasa por \(\fancyL\) entonces va a existir un \(t\) tal que para las ecuaciones parámetricas de \(\fancyL\)
        se pueda reemplazar las coordenadas de \(P\) en las variables de cada ecuación. Sin embargo, por ahora, solo nos vamos a concentrar en la 2\tsup{da} ecuación parámetrica.
        \[
            P = \begin{pmatrix}a \\ 1 \\ c\end{pmatrix}
            \hspace{1cm}
            \left\{
            \begin{aligned}
                a &= 4 + -6t \\
                1 &= 2 + -2t \\
                c &= 2 + 4t
            \end{aligned}
            \right.
            \hspace{0.5cm}
            \text{;}
            \hspace{0.5cm}
            -1 = -2t
            \hspace{0.5cm}
            \text{;}
            \hspace{0.5cm}
            \frac{1}{2} = t
            \hspace{0.5cm}
        \]
        Ahora, veamos que valores de \(a\) y \(c\) vamos a tener con este valor de \(t\)
        \[
            \left\{
            \begin{aligned}
                a &= 4 + -6\left(\frac{1}{2}\right) \\
                c &= 2 + 4\left(\frac{1}{2}\right)
            \end{aligned}
            \right.
            \hspace{0.5cm}
            \text{;}
            \hspace{0.5cm}
            \left\{
                \begin{aligned}
                    a &= \frac{8}{2} + -\frac{6}{2} \\
                    c &= \frac{4}{2} + \frac{4}{2}
                \end{aligned}
            \right.
            \hspace{0.5cm}
            \text{;}
            \hspace{0.5cm}
            \left\{
                \begin{aligned}
                    a &= \frac{2}{2} = 1 \\
                    c &= \frac{8}{2} = 4
                \end{aligned}
            \right.
        \]
        Entonces, vamos a tener que \(P\) es el punto 
        \(
            \begin{pmatrix}
                1 \\ 1 \\ 4
            \end{pmatrix}
        \)

\item Hallar la ecuación de la recta que pasa por \(\left(\begin{smallmatrix}2 \\ 1 \\ 5\end{smallmatrix}\right)\) y corta en forma perpendicular a la recta dada por las ecuaciones simétricas:
    \[
        \frac{x - 1}{3} = \frac{y + 2}{4} = \frac{z - 3}{2}
    \]
    Primero, encontremos el vector director de la recta dada por ecuaciones simétricas
    \[
        \fancyL_1 : 
        \begin{pmatrix}
            x \\ y \\ z
        \end{pmatrix} =
        \begin{pmatrix}
            1 \\ -2 \\ 3
        \end{pmatrix}
        + 
        t \cdot
        \begin{pmatrix}
            3 \\ 4 \\ 2
        \end{pmatrix}
        \hspace{1cm}
        \text{;}
        \hspace{1cm}
        d_1 = 
        \begin{pmatrix}
            3 \\ 4 \\ 2
        \end{pmatrix}
    \]
    Ahora, necesitamos un vector \(d_2\) tal que \(d_1 \cdot d_2 = 0\).
    \[
        \begin{aligned}
            3 \cdot v_1 + 4 \cdot v_2 + 2 \cdot v_3 &= 0 \\
            3 \cdot 2 + 4 \cdot -3 + 2 \cdot 3 &= 0 \\
            6 + -12 + 6 &= 0 \\ 
            0 &= 0
        \end{aligned}
        \hspace{1cm}
        d_2 =
        \begin{pmatrix}
            2 \\ -3 \\ 3
        \end{pmatrix}
    \]
    Consecuentemente, con \(d_2\) vamos a encontrar un recta que pasa por \(\begin{pmatrix}2 \\ 1 \\ 5\end{pmatrix}\) y expresarla en su ecuación vectorial
    \[
        \begin{pmatrix}
            x \\ y \\ z
        \end{pmatrix}
        = 
        \begin{pmatrix}
            2 \\ 1 \\ 5
        \end{pmatrix}
        +
        t \cdot 
        \begin{pmatrix}
            2 \\ -3 \\ 3
        \end{pmatrix}
    \]
\setcounter{enumi}{3}
\item La intersección entre los planos \(\pi_1\): \(x − y + z = 2\) y \(\pi_2\): \(2x − 3y + 4z = 7\) es:
    \begin{enumerate}[label=\listAlph]
        \item Un punto
        \item Una recta
        \item Un plano
        \item Vacı́a
        \item N.A
    \end{enumerate}
\setcounter{enumi}{5}
\item Muestre que las rectas 
    \[
        \fancyL{}_1:
        \begin{aligned}
            x &= 2 − t,\\
            y &= 1 + t,\\
            z &= −2t
        \end{aligned}
        \hspace{1cm}
        \text{y}
        \hspace{1cm}
        \fancyL{}_2:
        \begin{aligned}
            x &= 1 + s,\\
            y &= −2s,\\
            z &= 3 + 2s
        \end{aligned}
    \]
    no tienen puntos en común.
\item Acerca de los planos \(\fancyP{}_1: 2x − y + z = 3\) y \(\fancyP{}_2: x + y − z = 7\), es correcto afirmar que estos son:
    \begin{enumerate}[label=\listAlph]
        \item Paralelos
        \item Ortogonales
        \item El mismo
        \item N.A
    \end{enumerate}

\setcounter{enumi}{9}
\item Encuentre dos puntos y dos vectores directores de la recta de \(\realR^4\), \(\fancyL = \frac{x + 5}{-3} = 1 + z = \frac{w}{3}, y = 0\).
\setcounter{enumi}{13}
\item Determine si la recta \(\fancyL = \frac{x + 5}{-3} = 1 + z = \frac{w}{2}, y = 0\) intercepta a cada uno de los siguientes planos
    \[
        \fancyP_1 = 
        \begin{cases}
            x = 1 - 2t \\
            y = 2t - 4s \\
            z = -3t + 7s \\
            w = 4t - 6s
        \end{cases}
        \hspace{1cm}
        \text{y}
        \hspace{1cm}
        \fancyP_2 = 
        \begin{cases}
            x = -t -s \\
            y = 5t - 6s \\
            z = -4t + 4s \\
            w = -4
        \end{cases}
    \]
    En caso afirmativo, encuentre la intersección.
\item Dadas las ecuaciones de las rectas \(\fancyL_1, \fancyL_2\) y de los planos \(\fancyP_1, \fancyP_2\)
    \[
        \begin{aligned}
            \fancyL_1 &= 
            \begin{cases}
                x = 3 + t \\
                y = 2 - 2t \\
                z = 2t \\
                w = 0
            \end{cases}
            \hspace{1cm}
            &\text{y}
            \hspace{1cm}
            \fancyL_2 &:
            \frac{x - 2}{-1} =
            \frac{y - 3}{2} =
            \frac{z + 2}{-2}, w = 0 \\
            \fancyP_1 &=
            \begin{cases}
                x = 2 + 4t \\
                y = -2 + t + 2s \\
                z = 1 -t + 2s \\
                w = 3 + 3t
            \end{cases}
            \hspace{1cm}
            &\text{y}
            \hspace{1cm}
            \fancyP_2 &=
            \begin{cases}
                w = 6 - t - 3s \\
                y = 2 + s \\
                z = -6 + 2t + 2s \\
                w = -3 + t + 2s
            \end{cases}
        \end{aligned}
    \]
\item Sean \(\fancyL_1, \fancyL_2, \fancyL_3\) y \(\fancyP_1, \fancyP_2, \fancyP_3\) las siguientes rectas y planos de \(\realR^3\) y \(P\) un punto en \(\realR^3\)
    \[
        \begin{aligned}
            \fancyL_1 &=
            \begin{cases}
                x = -t \\
                y = 1 + 2t \\
                z = -2 + 3t
            \end{cases} \\
            \fancyL_2 &=
            \frac{x + 5}{6} =
            \frac{z - 1}{2},
            y = -1 \\
            \fancyL_3 &: 
            \begin{pmatrix}x \\ y \\ z\end{pmatrix}
            \begin{pmatrix}0 \\ -1 \\ 2\end{pmatrix}
            +
            \begin{pmatrix}3 \\ 1 \\ -1\end{pmatrix}
        \end{aligned}
        \hspace{1cm}
        \text{y}
        \hspace{1cm}
        \begin{aligned}
            \fancyP_1 &=
            \begin{cases}
                x = -3 -2t \\
                y = 5t - 6s \\ 
                z = 1 - 4t + 4s
            \end{cases} \\
            \fancyP_2 &:
            x - 4y + 3z - 7 = 0 \\ 
            \fancyP_3 &:
            \begin{pmatrix}x_1 \\ x_2 \\ x_3\end{pmatrix}
            =
            \begin{pmatrix}-1 \\ 0 \\ 2\end{pmatrix}
            + t\begin{pmatrix}3 \\ -3 \\ -5\end{pmatrix}
            + s\begin{pmatrix}6 \\ 0 \\ -2\end{pmatrix}
        \end{aligned}
    \]
    \[
        P = \begin{pmatrix}
            1 \\ -2 \\ 3
        \end{pmatrix}
    \]
    \begin{enumerate}[label=\listAlph]
		\item Encuentre un punto de cada recta y cada plano.
		\item Encuentre un vector director de cada recta.
		\item Encuentre los otros dos tipos de ecuaciones de las rectas.
		\item Encuentre dos vectores directores de cada plano.
		\item Encuentre un vector normal de cada plano.
		\item Encuentre los otros dos tipos de ecuaciones de los planos.
        \item Encuentre una recta paralela a la recta \(\fancyL_1\) que pase por el origen. Existe otra?
        \item Encuentre una recta ortogonal a la recta \(\fancyL_2\) que corte a la recta \(\fancyL_3\). Existe otra?
        \item Encuentre un plano que contenga la recta \(\fancyL_1\). Existe otro?
        \item Encuentre un plano paralelo a la recta \(\fancyL_1\) que pase por el origen. Existe otro?
        \item Encuentre un plano ortogonal a la recta \(\fancyL_3\) que contenga a una de las otras dos rectas. Existe otro?
		\item Encuentre un plano paralelo al plano \(\fancyP_1\) que pase por \(P\). Existe otro?
		\item Encuentre un plano ortogonal al plano \(\fancyP_2\) que contenga a la recta \(\fancyL_1\). Existe otro?
		\item Determine cuales rectas son ortogonales y cuales son paralelas.
		\item Determine cuales planos son ortogonales y cuales son paralelos.
		\item Determine cual de las rectas es ortogonal al plano \(\fancyP_1\).
		\item Determine cual de las rectas es paralela al plano \(\fancyP_2\).
		\item Determine cual de las rectas corta al plano \(\fancyP_3\).
		\item Determine cual de las rectas está contenida en el plano \(\fancyP_1\).
    \end{enumerate}
\setcounter{enumi}{17}
\item Dada la recta \(\fancyL_1: \frac{x + 2}{-2} = \frac{y - 5}{7} = \frac{z}{-5}\)
    \begin{enumerate}[label=\listAlph]
        \item Determine si \(\fancyL_1\) es paralela al plano \(\fancyP: 2x - 7y - 5 = 0\).
        \item Determine si \(\fancyL_1\) es perpendicular a la recta \(\fancyL_2: x = -1 -7t, y = -2t, z = 5\).
    \end{enumerate}
\item Sean \(P = \left(\begin{smallmatrix}2 \\ 0 \\ 1\end{smallmatrix}\right)^T\), la recta \(\fancyL: \begin{cases}x = 1 -t \\ y = 2 + 2t \\ z = 1 - 3t\end{cases}\) y el plano \(\fancyP: 3x -6y + 9z = 2\)
        \begin{itemize}
            \item Un vector director de la recta \(\fancyL\) es:
                \begin{enumerate}[label=\listAlph]
                    \item \(\begin{pmatrix}1 \\ 2 \\ 1\end{pmatrix}^T\)
					\item \(\begin{pmatrix}1 \\ -2 \\ -3\end{pmatrix}^T\)
					\item \(\begin{pmatrix}2 \\ 4 \\ 6\end{pmatrix}^T\)
					\item \(\begin{pmatrix}1 \\ -2 \\ 3\end{pmatrix}^T\)
					\item N.A
                \end{enumerate}
            \item De las afirmaciones siguientes, señale una \textbf{VERDADERA}.
                \begin{enumerate}
                    \item \(P \in \fancyL\)
                    \item \(\fancyL \bot \fancyP\)
                    \item \(P \in \fancyP\)
                    \item \(\fancyL \subset \fancyP\)
                    \item N.A
                \end{enumerate}
            \item Una ecuación vectorial de la recta \(\fancyL\) que pasa por \(P\) y es paralela al plano \(\fancyP\) es:
        \end{itemize}
\setcounter{enumi}{20}
\item Sean \(\fancyL_1, \fancyL_2, \fancyP_1, \fancyP_2\) las rectas y planos cuyas ecuaciones son las siguientes:
    \[
        \begin{aligned}
            \fancyL_1 &: -x = \frac{y - 2}{4} = \frac{z + 1}{-2}
            \hspace{1cm}
            \fancyL_2 &: 
            \begin{cases}
                x = -1 -2t \\
                y = 4 +8t \\ 
                z = -2 -4t \\
            \end{cases} \\
            \fancyP_1 &:
            \begin{cases}
                x = -2t \\
                y = 2 - s \\
                z = t - 2s
            \end{cases}
            \hspace{1cm}
            \fancyP_2 &: -x + 4y - 2z = 8
        \end{aligned}
    \]
    \begin{itemize}
        \item Si \(Q = \begin{smallmatrix}1 \\ −2 \\ 1\end{smallmatrix}\) se puede decir que:
            \begin{itemize}
                \item \(Q \in \fancyL_1\)
                \item \(Q \in \fancyL_2\)
                \item \(Q \in \fancyP_1\)
                \item \(Q \in \fancyP_3\)
                \item N.A
            \end{itemize}
        \item La recta \(\fancyL_2\) y el plano \(\fancyP_2\) se interceptan en:
            \begin{itemize}
				\item Ningún punto
				\item Un punto
				\item Infinitos puntos
				\item N.A
            \end{itemize}
        \item De las rectas \(\fancyL_1\) y \(\fancyL_2\) se puede decir que:
            \begin{itemize}
                \item \(\fancyL_1 \bot \fancyL_2\)
                \item \(\fancyL_1 \parallel \fancyL_2\)
                \item \(\fancyL_1 \cap \fancyL_2 \neq \emptyset\)
                \item \(\fancyL_1 \nparallel \fancyL_2\)
                \item N.A
            \end{itemize}
        \item De los planos \(\fancyP_1\) y \(\fancyP_2\) se puede decir que:
            \begin{itemize}
                \item \(\fancyP_1 \bot \fancyP_2\)
                \item \(\fancyP_1 \nparallel \fancyP_2\)
                \item \(\fancyP_1 \cap \fancyP_2 = \emptyset\)
                \item \(\fancyP_1 = \fancyP_2\)
                \item N.A
            \end{itemize}
    \end{itemize}
\end{enumerate}
\end{document}
