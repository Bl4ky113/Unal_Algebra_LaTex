
%%% Template Packages %%%

\usepackage{xifthen} % Template stuff 
\usepackage{graphicx} % Images
\usepackage{tcolorbox} % Color Box
\usepackage[%
vmargin=2cm,%
hmargin=2.25cm,%
headheight=22pt%
]{geometry} % Page Geometry
\usepackage{fancyhdr} % Header / Footer Styles
\usepackage{extramarks} % Header / Footer Marks

\usepackage{ragged2e} % Text Align
\usepackage{amsmath} % Math Align

\usepackage{booktabs} % Tables+ Package
\usepackage{enumitem} % Enumarete+ Package
\usepackage{xparse} % Command Args Split Package

\usepackage{mathtools} % Math General
\usepackage{amsthm} % Math Envs
\usepackage{unicode-math} % Math Symbols

\usepackage{polyglossia} % Language
    \setdefaultlanguage{spanish}%

%%% Variables and Such %%%
\ifthenelse{\isundefined{\TemplatePath}}{%
    \def\TemplatePath{template/}
}{}

%%% Template Styles %%%

% Header / Footer Styles
\pagestyle{fancy}
\RenewDocumentCommand{\headrule}{}{%
    \rule[0.1cm]{\textwidth}{0.1mm}%
}

\ifthenelse{\isundefined{\MyTitle}}{}{%
\fancyhf[HC]{{\slshape \MyTitle{}}}
}%
%\fancyhead[HL]{\firstleftxmark}
%\fancyhead[HR]{\lastleftxmark}

\RenewDocumentCommand{\footrule}{}{%
    \rule[0.1cm]{\textwidth}{0.1mm}%
}

\renewcommand{\thefootnote}{\Roman{footnote}} % Changing footnotes arabic to roman numbers

% Redefine \maketitle
\RenewDocumentCommand{\maketitle}{s}{%
    \begin{@twocolumntrue}%
        \begin{minipage}{0.3\textwidth}%
            \begin{Center}
                \includegraphics[width=0.6\textwidth]{\TemplatePath src/unal_logo.pdf}%
            \end{Center}
        \end{minipage}%
        \begin{minipage}{0.7\textwidth}{%
            \begin{Center}
                \ifthenelse{\isundefined{\MyClass}}{%
                    {\large Set \textbackslash{}def\textbackslash{}MyClass\{your\_class\} in the preamble} \\[0ex]%
                }{%

                    {\large \itshape \MyClass{}} \\[1ex]
                }%
                \ifthenelse{\isundefined{\MyTitle}}{%
                    {\huge Set \textbackslash{}def\textbackslash{}MyTitle\{doc\_title\} in the preamble} \\[4ex]%
                }{%
                    {\huge \slshape \MyTitle{}} \\[4ex]%
                }%
                \ifthenelse{\isundefined{\MyAuthor}}{%
                    {\Large Set \textbackslash{}def\textbackslash{}MyAuthor\{your\_name\} in the preamble} \\%
                }{%
                    {\Large \MyAuthor{}} \\%
                }%
                \ifthenelse{\isundefined{\MyEmail}}{%
                    {\small Set \textbackslash{}def\textbackslash{}MyEmail\{your\_email\} in the preamble} \\[4ex]%
                }{%
                    {\small \MyEmail{}} \\[4ex]%
                }%
                \ifthenelse{\isundefined{\MyDate}}{%
                    Set \textbackslash{}def\textbackslash{}MyDate\{doc\_date\} in the preamble%
                }{%
                    \MyDate{}%
                }%
            \end{Center}
        }%
        \end{minipage}%
    \end{@twocolumntrue}%
    \vspace{0.5cm}%
    \begin{Center}%
        \rule[0cm]{\textwidth}{0.1mm}%
    \end{Center}%
    \vspace{0.5cm}%
}

% Enumerate Lists types
\newcommand\SetItemnumber[1]{\setcounter{enumi}{\numexpr#1-1\relax}}
\newcommand{\listSubscript}[2]{\(#1_{#2}\)}
\newcommand{\listAlph}{\alph*.}
\newcommand{\listALPH}{\Alph*.}

% Fast Supper and Sub -scripts
\NewDocumentCommand{\tsup}{m} {\textsuperscript{#1}}
\NewDocumentCommand{\tsub}{m} {\textsubscript{#1}}

% Set of Numbers
\def\realR{\symbb{R}} % Real
\def\realN{\symbb{N}} % Natural
\def\realZ{\symbb{Z}} % Integers
\def\realQ{\symbb{Q}} % Rational
\def\realC{\symbb{C}} % Complex
\def\realC{\symbb{I}} % Irrational

%%% Template Math stuff %%%

% Theorems
\newtheorem{TMPMathTheorem}{Teorema}
\NewDocumentEnvironment{theorem}{+b} {%
    \begin{TMPMathTheorem}%
        #1 %
    \end{TMPMathTheorem}%
} {}

% Collorary
\newtheorem{TMPMathCorollary}[TMPMathTheorem]{Corolario}
\NewDocumentEnvironment{corollary}{+b} {%
    \begin{TMPMathCorollary}%
        #1 %
    \end{TMPMathCorollary}%
} {}

% Definitions
\newtheorem{TMPMathDefinition}[TMPMathTheorem]{Definición}
\NewDocumentEnvironment{definition}{+b} {%
    \begin{tcolorbox}[left=0mm,right=0mm]%
        \begin{TMPMathDefinition}%
            #1 %
            \hspace{\fill}\(\bigtriangleup\)%
        \end{TMPMathDefinition}%
    \end{tcolorbox}%
} {}

% Cases and stuff
\newtheorem{TMPMathCases}[]{Caso}
\NewDocumentCommand{\ResetCases}{} {\setcounter{TMPMathCases}{0}}
\NewDocumentEnvironment{mathcase}{O{}m} {%
    \IfValueTF{#1}{%
        \begin{TMPMathCases}[#1] % 
            #2%
        \end{TMPMathCases}%
    }{%
        \begin{TMPMathCases} % 
            #2%
        \end{TMPMathCases}%
    }
} {}

% Custom det of a Matrix
\NewDocumentCommand{\pdet}{m} {%
    \text{det}%
    \left(%
    #1 %
    \right)%
}

% Custom Adj of a Matrix
\NewDocumentCommand{\padj}{m} {%
    \text{Adj}%
    \left(%
    #1 %
    \right)%
}

% Custom Parentheses Gen of V.S
\NewDocumentCommand{\pgen}{m} {%
    \text{Gen}%
    \left(%
    #1 %
    \right)%
}

% Custom Bracket Gen of V.S
\NewDocumentCommand{\bgen}{m} {%
    \text{Gen}%
    \left\{%
    #1 %
    \right\}%
}

% Custom Parentheses Dimension of V.S
\NewDocumentCommand{\pdim}{m} {%
    \text{dim}%
    \left(%
    #1 %
    \right)%
}

% Custom Squarebrackets 
\NewDocumentCommand{\veccoord}{mm} {%
    \left[%
    #1 %
    \right]{}_{ #2 }%
}

% Fast Set
\NewDocumentCommand{\fset}{m} {%
    \left\{%
        #1 %
    \right\}%
}

% Fast Vector Proyection

\NewDocumentCommand{\fproy}{mm} {%
    \text{Proy}_{#1} #2
}

% Fast Vector Norm
\NewDocumentCommand{\fnorm}{m} {%
    \left|\left|#1\right|\right|
}

% Fast Vector & Small Vector
\NewDocumentCommand{\fvec}{ > {\SplitList{,}}m } {%
    \begin{pmatrix} %
        \ProcessList{#1}{\handlefvec}
    \end{pmatrix} %
}
\NewDocumentCommand{\fsmvec}{ > {\SplitList{,}}m } {%
    \left(
    \begin{smallmatrix} %
        \ProcessList{#1}{\handlefvec}
    \end{smallmatrix} %
    \right)
}
\NewDocumentCommand{\handlefvec}{m}{%
    #1 \\%
}

% Fast Matrix & Small Matrix
\NewDocumentCommand{\fmatrix}{>{\SplitList{;}}m} {%
    \begin{pmatrix} %
        \ProcessList{#1}{\handlefmatrixline}
    \end{pmatrix} %
}
\NewDocumentCommand{\fsmmatrix}{>{\SplitList{;}}m} {%
    \left(
    \begin{smallmatrix} %
        \ProcessList{#1}{\handlefmatrixline}
    \end{smallmatrix} %
    \right)
}
\NewDocumentCommand{\handlefmatrixline}{>{\SplitList{,}}m}{%
    \ProcessList{#1}{\handlefmatrixitem} \\
}
\NewDocumentCommand{\handlefmatrixitem}{m}{%
    \ifthenelse{\isundefined{\FirstItem}}{%
        #1 
    }{%
        & #1
    }%
    \def\FirstItem{0}
}

