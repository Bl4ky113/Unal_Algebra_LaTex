\item Demuestre, teniendo en cuenta que \(A, B, C, O\) son matrices.
    \begin{enumerate}[label=\listAlph]
        \item Si \(A\) es invertible y \(AB = AC\), entonces \(B = C\).
            \begin{proof}
                Primero podemos ver que \(A, B, C\) son de orden \(n\) con \(n\) siendo un número natural cualesquiera.
                Seguidamente, de manera directa, sabemos por hipotesis que existe \(A^{-1}\) tal que \(A^{-1} \cdot A = I_n\), entonces 
                tomando la igualdad dada en la hipotesis \(A \cdot B = A \cdot C\) la podemos transformar de la siguiente forma.
                \[
                    \begin{aligned}
                        A \cdot B &= A \cdot C \\
                        A^{-1} \cdot (A \cdot B) &= A^{-1} \cdot (A \cdot C) \\
                        \left(A^{-1} \cdot A\right) \cdot B &= \left(A^{-1} \cdot A\right) \cdot C \\
                        I_n \cdot B &= I_n \cdot C \\
                        B &= C
                    \end{aligned}
                \]
                Concluyendo que, por la propiedad de asociatividad del producto entre matrices y definición de inversa, se obtiene que \(B = C\).
            \end{proof}
        \item \(AA^T\) es una matriz simétrica para cualquier matriz \(A\).
            \begin{proof}
                Primero sabemos que una matriz simétrica si y solo si su transpuesta es igual a si misma, o de manera similar \(A = A^T\).
                Seguidamente, de manera directa, si \(A\) tiene \(m\) filas y \(n\) columnas, 
                entonces su transpuesta \(A^T\) va a tener \(n\) filas y \(m\) columnas, 
                haciendo que la pre-multiplicación de \(A^T\) por \(A\) exista. Ahora teniendo esta matriz 
                \(A \cdot A^T\), veamos si su transpuesta es igual a si misma.
                \[
                    \left(A \cdot A^T\right){}^T = \left(A^T\right){}^T \cdot A^T = A \cdot A^T
                \]
                Este resultado se basa fuertemente en la propiedad de distribución de la transpuesta en el producto entre matrices, sin embargo, aún así
                se obtiene que la matriz \(AA^T = \left(AA^T\right){}^T\), es decir, \(AA^T\) es una matriz simétrica para cualquier matriz \(A\).
            \end{proof}
        \item Si la matriz \(A\) es invertible y \(AB = O\), entonces \(B = O\).
            \begin{proof}
                Primero, podemos ver que \(A\) es de orden \(n\), sin embargo solo sabemos que \(B\) tiene \(n\) filas y \(k\) columnas, y de manera similar, 
                \(O\) tiene \(n\) filas y \(k\) columnas, además donde \(n\) y \(k\) son números naturales cualesquiera.
                Posteriormente, de manera directa, sabemos que por hipotesis existe \(A^{-1}\) tal que \(A^{-1} \cdot A = I_n\). 
                Ahora, podemos tomar la igualdad definida en la hipotesis \(A \cdot B = O\), la cual se puede transforma de la siguiente forma.
                \[
                    \begin{aligned}
                        A \cdot B &= O \\
                        A^{-1} \cdot (A \cdot B) &= A^{-1} \cdot O \\
                        \left(A^{-1} \cdot A\right) \cdot B &= O \\
                        I_n \cdot B &= O \\
                        B &= O \\
                    \end{aligned}
                \]
                Cabe aclarar que \(I_n \cdot B\) esta definida ya que \(B\), como anteriormente mencionado, tiene \(n\) columnas. Concluyendo que \(B = O\).
            \end{proof}
    \end{enumerate}
