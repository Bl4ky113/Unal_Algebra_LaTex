\item Las siguientes proposiciones son \textbf{VERDADERAS} justifique el \textbf{POR QUÉ}.
    \begin{enumerate}[label=\listAlph]
        \item Cualquier múltiplo diferente de cero de una matriz invertible es invertible. \\
            Verdadero, para cualquier matriz cuadrada \(A\) y escalar \(\realR - \left\{0\right\}\). 
            Partiendo de una matriz de orden \(n\) \(A\), con \(n \in \realN\), tal que tenga su inversa \(A^{-1}\), 
            tal que \(A \cdot A^{-1} = I_n\). Además sea \(\lambda\) un escalar cualesquiera pero diferente de cero, veamos 
            si la inversa de \(\lambda A\) es \(\frac{1}{\lambda}A^{-1}\). 
            \[
                \left(\lambda A\right) \cdot \left(\frac{1}{\lambda} A^{-1}\right)
                = \left(\lambda \cdot \frac{1}{\lambda}\right) A \cdot A^{-1}
                = 1 \cdot I_n
                = I_n
            \]
        %\item Si una matriz es invertible, cualquier forma escalonada equivalente también es invertible.
        %\item Si la forma escalonada equivalente de una matriz es invertible, la matriz también es invertible.
        \setcounter{enumii}{3}
        \item Si la matriz \(A\) es invertible, el sistema \(Ax = b\) tiene solución única. \\
            Verdadero. Primero podemos ver que al tomar el sistema \(Ax = b\) y transformarlo tenemos \(A^{-1} \cdot (A \cdot x) = I_n \cdot x = x = A^{-1} \cdot b\), 
            es decir \(A^{-1} \cdot b\) es una solución para el sistema. Podemos verificar esto simplemente reemplazando a \(x\) en el sistema. Además podemos verificar si 
            esta solución es única viendo si existe otra solución para el sistema, \(y\)
            \[
                \begin{aligned}
                    Ax &= b \\
                    A \cdot (A^{-1} \cdot b) &= \\
                    (A \cdot A^{-1}) \cdot b &= \\
                    I_n \cdot b &= \\
                    b &= b\\
                \end{aligned}
                \hspace{1cm}
                \begin{aligned}
                    Ay &= b \\
                    A^{-1} \cdot (A \cdot y) &= A^{-1} \cdot b; \text{multiplicamos por } A^{-1} \\
                    (A^{-1} \cdot A) \cdot y &= A^{-1} \cdot b \\
                    I_n \cdot y &= A^{-1} \cdot b \\
                    y &= A^{-1} \cdot b \\
                    y &= x \\
                \end{aligned}
            \]
            Concluyendo que, ya que \(A\) es invertible se tiene que la solución del sistema planteado es \(A^{-1} \cdot b\) y esta solución es única.
        %\item Si la suma de las columnas de una matriz es igual al vector 0, la matriz no es invertible.
        \setcounter{enumii}{5}
        \item Si el sistema \(Ax = b\) tiene infinitas soluciones, la matriz \(A\) no es invertible. \\
            Verdadero, se va a tener que la matriz \(A_{m \times n}\) no es cuadrada o \(A\) no cuenta con \(n\) pivotes, con \(n, m \in \realN\).
            Inicialmente podemos definir a \(E\) como la matriz escalonada equivalente a \(A\), la cual al no tener una única solución no tiene \(n\)
            pivotes, ya que en el sistema de ecuaciones equivalente, almenos debe haber una variable libre para que sus soluciones sean infinitas. 
            Aunque se podrían ver los casos donde \(E\) es una matriz cuadrada, cuenta con una fila de ceros como ultima columna, o donde \(m < n\),
            no teniendo suficientes filas para tener \(n\) pivotes, solo es necesario que no se cuente con \(n\) pivotes en \(E\) para que \(A\) no 
            sea invertible.
        %\item Si el sistema \(Ax = b\) tiene solución única y A es cuadrada, la matriz \(A\) es invertible. (ñapa) \\
        %\item Si la matriz aumentada de un sistema de ecuaciones lineales es invertible, el sistema es inconsistente.
    \end{enumerate}
