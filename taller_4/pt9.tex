\item Demuestre porque son \textbf{FALSAS} las siguientes proposiciones.
    \begin{enumerate}[label=\listAlph]
        \item Toda matriz idempotente es invertible. \\
            \begin{proof}
                Como para demostrar que una afirmación sea falsa solo es necesario un contra ejemplo:
                Sea \(A\) una matriz de orden 2, tal que \(A = A^2\)
                \[
                    A = \begin{pmatrix}
                        3 & -6 \\
                        1 & -2
                    \end{pmatrix}
                    \hspace{0.35cm}
                    A^2 = 
                    \begin{pmatrix}
                        3 & -6 \\
                        1 & -2
                    \end{pmatrix}
                    \cdot 
                    \begin{pmatrix}
                        3 & -6 \\
                        1 & -2
                    \end{pmatrix}
                    =
                    \begin{pmatrix}
                        3 \cdot 3 + (-6) \cdot 1 & 3 \cdot (-6) + (-6) \cdot (-2) \\
                        1 \cdot 3 + (-2) \cdot 1 & 1 \cdot (-6) + (-2) \cdot (-2)
                    \end{pmatrix}
                    =
                    \begin{pmatrix}
                        3 & -6 \\
                        1 & -2
                    \end{pmatrix}
                \]
                Sin embargo al realizar operaciones elementales en \(A\) hasta que se obtenga una matriz escalonada equivalente, se 
                puede ver que no cuenta con 2 pivotes necesarios para que \(A\) cuente con inversa.
                \[
                    \begin{pmatrix}
                        3 & -6 \\
                        1 & -2
                    \end{pmatrix}
                    \sim
                    \begin{aligned}
                        \frac{1}{3}F_3 &\mapsto F_3 \\
                        F_2 - F_3 &\mapsto F_2 \\
                    \end{aligned}
                    \begin{pmatrix}
                        1 & -2 \\
                        0 & 0
                    \end{pmatrix}
                \]
            \end{proof}
        
        \item Si \(A^4 = A^2\), la matriz \(A\) es idempotente. \\
            \begin{proof}
                Como para demostrar que una afirmación sea falsa solo es necesario un contra ejemplo:
                Sea \(A\) de orden \(n\), tal que \(A^2 = A^4\), es decir \(A^2\) sea idempotente, 
                \[
                    A = 
                    \begin{pmatrix}
                        0 & 1 \\
                        1 & 0
                    \end{pmatrix}
                    \hspace{0.5cm}
                \]
                \[
                    A \cdot A =
                    A^2 = 
                    \begin{pmatrix}
                        0 & 1 \\
                        1 & 0
                    \end{pmatrix}
                    \cdot
                    \begin{pmatrix}
                        0 & 1 \\
                        1 & 0
                    \end{pmatrix}
                    =
                    \begin{pmatrix}
                        1 & 0 \\
                        0 & 1
                    \end{pmatrix}
                \]
                \[
                    A^2 \cdot A^2 =
                    A^4 = 
                    \begin{pmatrix}
                        1 & 0 \\
                        0 & 1
                    \end{pmatrix}
                    \cdot
                    \begin{pmatrix}
                        1 & 0 \\
                        0 & 1
                    \end{pmatrix}
                    =
                    \begin{pmatrix}
                        1 & 0 \\
                        0 & 1
                    \end{pmatrix}
                \]
                Es decir tenemos que \(A^2 = A^4\) o \(A^2\) es idempotente, sin embargo, tenemos que \(A \neq A^2\) o que \(A\) no es idempotente.
            \end{proof}
    \end{enumerate}
