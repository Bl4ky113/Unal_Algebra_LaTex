\item Las siguientes proposiciones son \textbf{FALSAS} justifique el \textbf{POR QUÉ}.
    \begin{enumerate}[label=\listAlph]
        %\item Si las columnas de una matriz forman un conjunto de vectores \(l.i\), la matriz es invertible.
        \setcounter{enumii}{1}
        \item La suma de dos matrices invertibles es una matriz invertible. \\
            Falso, ya que se tiene un contraejemplo, la suma entre una matriz y su suma inversa.
            Primero, definiendo a \(A\) y \(-A\) matrices con orden \(n\), con \(n \in \realN\), además existen \(A^{-1}\) y \(-A^{-1}\), sus inversas respectivamente.
            La suma de ambas matrices, \(A + (-A) = O_n\), sigue siendo un cantidato para ser inversa ya que es de orden \(n\). 
            Sin embargo como \(O_n\), al ser escalonada, no tienen ningún pivote, o almenos no tiene \(n\) pivotes, \(A + (-A)\) no va a ser invertible.
        %\item Si la solución de un sistema de ecuaciones lineales homogéneo es única, la matriz de coeficientes del sistema es invertible.
        %\item Si la suma de dos matrices es invertible, cada una de las matrices es invertible.
        %\item El producto de dos matrices cuadradas siempre es invertible.
        \setcounter{enumii}{5}
        \item Si el sistema \(Ax = b\) tiene solución única, la matriz \(A\) es invertible. \\
            Falso, ya que no es necesario que \(A\) sea una matriz cuadrada.
            Inicialmente, sabemos que si \(E_{m \times n}\), con \(m,n \in \realN\), es la matriz escalonada equivalente a \(A\), 
            \(E\) va a tener \(n\) pivotes ya que el sistema \(Ex = b\) debe tener una solución única. 
            Sin embargo, puede que \(m > n\) y que en \(E\) se tenga filas de ceros en la ultima fila de la matriz.
            Haciendo que no afecte el número de soluciones del sistema al no añádir información, pero haciendo que \(E\) no 
            sea una matriz cuadrada y por ende no invertible. De manera similar, se va a poder ver que \(A\) no va a ser 
            cuadrada y por ende tampoco va a ser invertible.
        %\item Para concluir que una matriz es invertible, es necesario calcular su inversa.
        %\item Si la matriz aumentada de un sistema de ecuaciones lineales es invertible, el sistema tiene solucion única.
        \setcounter{enumii}{8}
        \item Si el producto de dos matrices es invertible, cada una de las matrices es invertible. \\
            Falso, sean dos matrices no cuadradas, \(A_{m \times n}\) y \(B_{n \times m}\), con \(m,n \in \realN\), tal que \(AB_{m \times m}\)
            sea invertible. Tenemos que \(AB\) es invertible, pero ya que \(A\) y \(B\) no son matrices cuadradas, estas no pueden ser invertibles.
    \end{enumerate}
