\item Al escalonar la matriz \(A\), se aplicaron las operaciones elementales para obtener la siguiente matriz
    \[
        \begin{aligned}
            F_1 &\leftrightarrow F_3 \\
            F_2 + 2F_1 &\mapsto F_2 \\ 
            F_3 - \frac{5}{2}F_1 &\mapsto F_3 \\
            5F_2 &\mapsto F_2 \\ 
            F_3 + \frac{10}{11}F_2 &\mapsto F_3
        \end{aligned}
        \hspace{1cm}
        \begin{pmatrix}
            -2 & 4 & 4 \\ 
            0 & 11 & 6 \\ 
            0 & 0 & -\frac{17}{11}
        \end{pmatrix}
    \]
    \begin{enumerate}[label=\listAlph]
        \item Calcule \(\det A\) \\
            Vamos a \emph{reconstruir} a la matriz \(A\), siguiendo en orden contrario las operaciones elementales dadas. 
            Y apartir de esta matriz, calcular su determinante. Para esto, vamos a aplicar la inversa de cada operación elemental realizada en 
            \(A\) para obtener la matriz dada en sentido contrario
            \[
                \begin{pmatrix}
                    -2 & 4 & 4 \\ 
                    0 & 11 & 6 \\ 
                    0 & 0 & -\frac{17}{11}
                \end{pmatrix}
                \sim
                \begin{aligned}
                    F_3 - \frac{10}{11} &\mapsto F_3 \\
                \end{aligned}
                \begin{pmatrix}
                    -2 & 4 & 4 \\ 
                    0 & 11 & 6 \\ 
                    0 & -10 & -7
                \end{pmatrix}
                \sim
                \begin{aligned}
                    \frac{1}{5}F_2 &\mapsto F_2 \\ 
                    F_3 + \frac{5}{2}F_1 &\mapsto F_3 \\
                \end{aligned}
                \begin{pmatrix}
                    -2 & 4 & 4 \\ 
                    0 & \frac{11}{5} & \frac{6}{5} \\ 
                    -5 & 0 & 3
                \end{pmatrix}
            \]
            \[
                \sim
                \begin{aligned}
                    F_2 - 2F_1 &\mapsto F_2 \\ 
                \end{aligned}
                \begin{pmatrix}
                    -2 & 4 & 4 \\ 
                    2 & -\frac{9}{5} & -\frac{14}{5} \\ 
                    -5 & 0 & 3
                \end{pmatrix}
                \sim
                \begin{aligned}
                    F_3 &\leftrightarrow F_1 \\
                \end{aligned}
                \begin{pmatrix}
                    -5 & 0 & 3 \\
                    2 & -\frac{9}{5} & -\frac{14}{5} \\ 
                    -2 & 4 & 4 
                \end{pmatrix}
            \]
            Ahora, con \(A\) reconstruida, podemos calcular su determinante mediante \textbf{La Regla de Sarrus}, 
            al ser una matriz de orden 3.
            \[
                \begin{aligned}
                    \pdet{A}&=
                    \begin{aligned}
                    \begin{vmatrix}
                        -5 & 0 & 3 \\
                        2 & -\frac{9}{5} & -\frac{14}{5} \\ 
                        -2 & 4 & 4 
                    \end{vmatrix}
                    \\
                    \begin{array}{ccc}
                        -5 & 0 & 3 \\
                        2 & -\frac{9}{5} & -\frac{14}{5} \\ 
                    \end{array}
                    \end{aligned}
                    \\
                    &=
                    \left(
                        \left(-5 \cdot -\frac{9}{5} \cdot 4\right)
                        +\left(2 \cdot 4 \cdot 3\right)
                        +\left(-2 \cdot 0 \cdot -\frac{14}{5}\right)
                    \right)
                    -
                    \left(
                        \left(3 \cdot -\frac{9}{5} \cdot -2\right)
                        +\left(-\frac{14}{5} \cdot 4 \cdot -5\right)
                        +\left(4 \cdot 0 \cdot 2\right)
                    \right)
                    \\
                    &=
                    \left(
                        36 + 24
                    \right)
                    -
                    \left(
                        \frac{54}{5} + 56
                    \right)
                    =
                    \frac{300}{5}
                    - 
                    \frac{334}{5}
                    \\
                    &=
                    -\frac{34}{5}
                \end{aligned}
            \]
        \item ¿Existe \(A^{-1}\)? \\
            Como la matriz escalonada reducida equivalente a \(A\) cuenta con 3 pivotes, podemos afirmar que existe la matriz inversa para \(A\).
            %Inclusive, como hemos obtenido la matriz escalonada reducida mediante operaciones elementales, podemos aplicar estas operaciones a 
            %una matriz identidad de orden 3 para obtener la inversa de \(A\).
            %\[
                %\begin{aligned}
                    %\begin{aligned}
                        %F_1 &\leftrightarrow F_3 \\
                        %F_2 + 2F_1 &\mapsto F_2 \\ 
                        %F_3 - \frac{5}{2}F_1 &\mapsto F_3 \\
                    %\end{aligned}
                    %\begin{pmatrix}
                        %0 & 0 & 1 \\
                        %0 & 1 & 2 \\ 
                        %1 & 0 & -\frac{5}{2} \\ 
                    %\end{pmatrix}
                    %\hspace{0.15cm}
                    %&\sim
                    %\hspace{0.15cm}
                    %\begin{aligned}
                        %5F_2 &\mapsto F_2 \\ 
                        %F_3 + \frac{10}{11}F_2 &\mapsto F_3 \\
                        %-\frac{1}{2}F_1 &\mapsto F_1 \\
                    %\end{aligned}
                    %\begin{pmatrix}
                        %0 & 0 & -\frac{1}{2} \\
                        %0 & 5 & 10 \\ 
                        %1 & \frac{50}{11} & -\frac{35}{22} \\ 
                    %\end{pmatrix}
                    %\hspace{0.15cm}
                    %&\sim
                    %\hspace{0.15cm}
                    %\\
                    %\begin{aligned}
                        %-\frac{11}{17}F_3 &\mapsto F_3 \\
                        %F_2 - 6F_3 &\mapsto F_2 \\
                        %F_1 + 2F_3 &\mapsto F_1 \\
                    %\end{aligned}
                    %\begin{pmatrix}
                        %-\frac{11}{17} & \frac{50}{17} & -\frac{26}{17} \\
                        %\frac{11}{17} & -\frac{215}{17} & \frac{65}{17} \\ 
                        %-\frac{11}{17} & \frac{50}{17} & -\frac{35}{34} \\ 
                    %\end{pmatrix}
                    %\hspace{0.15cm}
                    %&\sim
                    %\hspace{0.15cm}
                    %\begin{aligned}
                        %\frac{1}{11}F_2 &\mapsto F_2 \\
                        %F_1 + 2F_2 &\mapsto F_1
                    %\end{aligned}
                    %\begin{pmatrix}
                        %-\frac{11}{17} & \frac{50}{17} & -\frac{26}{17} \\
                        %\frac{11}{17} & -\frac{215}{17} & \frac{65}{17} \\ 
                        %-\frac{11}{17} & \frac{50}{17} & -\frac{35}{34} \\ 
                    %\end{pmatrix}
                    %\hspace{0.15cm}
                    %&\sim
                    %\hspace{0.15cm}
                %\end{aligned}
            %\]
        \item Calcule \(\det U\) \\
            Simplemente, vamos a obtener a \(\pdet{U}\) multiplicando las componentes diagonales al ser una matriz triángular superior.
            \[
                \pdet{U} =
                -2 \cdot 11 \cdot -\frac{17}{11} =
                34
            \]
        \item Calcule \(\det \adj\left(A\right)\) \\
            Como contamos con \(\pdet{A}\), por el punto anterior, podemos calcular el \(\pdet{\padj{A}}\) apartir de esté. Obteniendo
            \[
                \pdet{\padj{A}}
                =
                \pdet{A}^{3 - 1}
                =
                \pdet{A}^2
                =
                \left(-\frac{34}{5}\right){}^2
                =
                \frac{1156}{25}
            \]
    \end{enumerate}
