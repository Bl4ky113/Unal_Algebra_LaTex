\item Al escalonar la matriz \(A\), se aplicaron las operaciones elementales para obtener la siguiente matriz
    \[
        \begin{aligned}
            F_1 &\leftrightarrow F_3 \\
            F_2 + 2F_1 &\mapsto F_2 \\ 
            F_3 - \frac{5}{2}F_1 &\mapsto F_3 \\
            5F_2 &\mapsto F_2 \\ 
            F_3 + \frac{10}{11}F_2 &\mapsto F_3
        \end{aligned}
        \hspace{1cm}
        \begin{pmatrix}
            -2 & 4 & 4 \\ 
            0 & 11 & 6 \\ 
            0 & 0 & -\frac{17}{11}
        \end{pmatrix}
    \]
    Primero vamos a continuar reduciendo la matriz con \emph{Gauss-Jordan}.
    \[
        \begin{aligned}
            \begin{aligned}
                -\frac{1}{2}F_1 &\mapsto F_1 \\
                -\frac{11}{17}F_3 &\mapsto F_3 \\
            \end{aligned}
            \begin{pmatrix}
                1 & -2 & -2 \\ 
                0 & 11 & 6 \\ 
                0 & 0 & 1
            \end{pmatrix}
            \hspace{0.15cm}
            &\sim
            \hspace{0.15cm}
            \begin{aligned}
                F_2 - 6F_3 &\mapsto F_2 \\
                F_1 + 2F_3 &\mapsto F_1 
            \end{aligned}
            \begin{pmatrix}
                1 & -2 & 0 \\ 
                0 & 11 & 0\\ 
                0 & 0 & 1
            \end{pmatrix}
            \hspace{0.15cm}
            &\sim
            \hspace{0.15cm}
            \\
            \begin{aligned}
                \frac{1}{11}F_2 &\mapsto F_2
            \end{aligned}
            \begin{pmatrix}
                1 & -2 & 0 \\ 
                0 & 1 & 0 \\ 
                0 & 0 & 1
            \end{pmatrix}
            \hspace{0.15cm}
            &\sim
            \hspace{0.15cm}
            \begin{aligned}
                F_1 + 2F_2 &\mapsto F_1
            \end{aligned}
            \begin{pmatrix}
                1 & 0 & 0 \\ 
                0 & 1 & 0 \\ 
                0 & 0 & 1
            \end{pmatrix}
            \hspace{0.15cm}
            &\phantom{\sim}
            \hspace{0.15cm}
        \end{aligned}
    \]
    \begin{enumerate}[label=\listAlph]
        \item Calcule \(\det A\) \\
            Vamos a \emph{reconstruir} a la matriz \(A\), siguiendo en orden contrario las operaciones elementales dadas. 
            Y apartir de esta matriz, calcular su determinante.
        \item ¿Existe \(A^{-1}\)? \\
            Como la matriz escalonada reducida equivalente a \(A\) cuenta con 3 pivotes, podemos afirmar que existe la matriz inversa para \(A\).
            %Inclusive, como hemos obtenido la matriz escalonada reducida mediante operaciones elementales, podemos aplicar estas operaciones a 
            %una matriz identidad de orden 3 para obtener la inversa de \(A\).
            %\[
                %\begin{aligned}
                    %\begin{aligned}
                        %F_1 &\leftrightarrow F_3 \\
                        %F_2 + 2F_1 &\mapsto F_2 \\ 
                        %F_3 - \frac{5}{2}F_1 &\mapsto F_3 \\
                    %\end{aligned}
                    %\begin{pmatrix}
                        %0 & 0 & 1 \\
                        %0 & 1 & 2 \\ 
                        %1 & 0 & -\frac{5}{2} \\ 
                    %\end{pmatrix}
                    %\hspace{0.15cm}
                    %&\sim
                    %\hspace{0.15cm}
                    %\begin{aligned}
                        %5F_2 &\mapsto F_2 \\ 
                        %F_3 + \frac{10}{11}F_2 &\mapsto F_3 \\
                        %-\frac{1}{2}F_1 &\mapsto F_1 \\
                    %\end{aligned}
                    %\begin{pmatrix}
                        %0 & 0 & -\frac{1}{2} \\
                        %0 & 5 & 10 \\ 
                        %1 & \frac{50}{11} & -\frac{35}{22} \\ 
                    %\end{pmatrix}
                    %\hspace{0.15cm}
                    %&\sim
                    %\hspace{0.15cm}
                    %\\
                    %\begin{aligned}
                        %-\frac{11}{17}F_3 &\mapsto F_3 \\
                        %F_2 - 6F_3 &\mapsto F_2 \\
                        %F_1 + 2F_3 &\mapsto F_1 \\
                    %\end{aligned}
                    %\begin{pmatrix}
                        %-\frac{11}{17} & \frac{50}{17} & -\frac{26}{17} \\
                        %\frac{11}{17} & -\frac{215}{17} & \frac{65}{17} \\ 
                        %-\frac{11}{17} & \frac{50}{17} & -\frac{35}{34} \\ 
                    %\end{pmatrix}
                    %\hspace{0.15cm}
                    %&\sim
                    %\hspace{0.15cm}
                    %\begin{aligned}
                        %\frac{1}{11}F_2 &\mapsto F_2 \\
                        %F_1 + 2F_2 &\mapsto F_1
                    %\end{aligned}
                    %\begin{pmatrix}
                        %-\frac{11}{17} & \frac{50}{17} & -\frac{26}{17} \\
                        %\frac{11}{17} & -\frac{215}{17} & \frac{65}{17} \\ 
                        %-\frac{11}{17} & \frac{50}{17} & -\frac{35}{34} \\ 
                    %\end{pmatrix}
                    %\hspace{0.15cm}
                    %&\sim
                    %\hspace{0.15cm}
                %\end{aligned}
            %\]
        \item Calcule \(\det U\) \\
            Simplemente, vamos a obtener a \(\pdet{U}\) multiplicando las componentes diagonales al ser una matriz triángular superior.
            \[
                \pdet{U} =
                -2 \cdot 11 \cdot -\frac{17}{11} =
                34
            \]
        \item Calcule \(\det \adj\left(A\right)\) \\
    \end{enumerate}
