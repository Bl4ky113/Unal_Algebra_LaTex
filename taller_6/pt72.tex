\begin{definition}
    \textbf{Regla de Cramer}.
    Dado un sistema de ecuaciones lineales \(Ax = b\) de tamaño \(n \times n\), donde \(A\) es invertible.
    Entonces las componentes de la solución \(x^T = \left(x_1 , x_2 , \ldots , x_n \right)\) satisfacen
    \[
        x_i = \frac{\det A_i}{\det A}
    \]
    donde \(A_i\) es la matriz que se obtiene de \(A\) reemplazando la columna \(i\) por el vector \(b\).
\end{definition}
\item Resolver el sistema dado por la \textbf{Regla de Cramer}, siempre que sea posible.
    \begin{enumerate}[label=\listAlph]
        \item 
            \(
                \left\{
                \begin{aligned}
                    &x &+y &+z &-2w &= -4 \\
                    &\phantom{+0x} &2y &+z &+3w &= 1 \\
                    &2x &+y &-z &+2w &= 5 \\
                    &x &-y &\phantom{+0z} &+w &= 4
                \end{aligned}
                \right.
            \) \\
            Primero, creamos la matriz de coeficientes de las incognitas del sistema de ecuaciones. 
            Luego, intercambiamos cada columna de coeficientes por la columan de los terminos independientes.
            Para así calcular los determinantes de cada matriz.
            \[
                A = 
                \begin{pmatrix}
                    1 & 1 & 1 & -2 \\
                    0 & 2 & 1 & 3 \\
                    2 & 1 & -1 & 2 \\
                    1 & -1 & 0 & 1
                \end{pmatrix}
                \hspace{0.5cm}
                A_1 = 
                \begin{pmatrix}
                    -4 & 1 & 1 & -2 \\
                    1 & 2 & 1 & 3 \\
                    5 & 1 & -1 & 2 \\
                    4 & -1 & 0 & 1
                \end{pmatrix}
            \]
            \[
                A_2 = 
                \begin{pmatrix}
                    1 & -4 & 1 & -2 \\
                    0 & 1 & 1 & 3 \\
                    2 & 5 & -1 & 2 \\
                    1 & 4 & 0 & 1
                \end{pmatrix}
                \hspace{0.5cm}
                A_3 = 
                \begin{pmatrix}
                    1 & 1 & -4 & -2 \\
                    0 & 2 & 1 & 3 \\
                    2 & 1 & 5 & 2 \\
                    1 & -1 & 4 & 1
                \end{pmatrix}
                \hspace{0.5cm}
                A_4 = 
                \begin{pmatrix}
                    1 & 1 & 1 & -4 \\
                    0 & 2 & 1 & 1 \\
                    2 & 1 & -1 & 5 \\
                    1 & -1 & 0 & 4
                \end{pmatrix}
            \]
            \[
                \begin{aligned}
                    \pdet{A}
                    &= 
                    \begin{vmatrix}
                        1 & 1 & 1 & -2 \\
                        0 & 2 & 1 & 3 \\
                        2 & 1 & -1 & 2 \\
                        1 & -1 & 0 & 1
                    \end{vmatrix}
                    \sim 
                    \begin{aligned}
                        F_3 - 2F_1 &\mapsto F_3 \\
                        F_4 - F_1 &\mapsto F_4 \\
                    \end{aligned}
                    \begin{vmatrix}
                        1 & 1 & 1 & -2 \\
                        0 & 2 & 1 & 3 \\
                        0 & -1 & -3 & 6 \\
                        0 & -2 & -1 & 3
                    \end{vmatrix}
                    \\
                    &\sim 
                    \begin{aligned}
                        F_3 + \frac{1}{2}F_2 &\mapsto F_3 \\
                        F_4 + F_2 &\mapsto F_4 \\
                    \end{aligned}
                    \begin{vmatrix}
                        1 & 1 & 1 & -2 \\
                        0 & 2 & 1 & 3 \\
                        0 & 0 & -\frac{5}{2} & \frac{15}{2} \\
                        0 & 0 & 0 & 6
                    \end{vmatrix}
                    = 1 \cdot 2 \cdot -\frac{5}{2} \cdot 6
                    = -5 \cdot 6 
                    \\
                    &= -30
                \end{aligned}
            \]
            \[
                \begin{aligned}
                    \pdet{A_1} 
                    &= 
                    \begin{vmatrix}
                        -4 & 1 & 1 & -2 \\
                        1 & 2 & 1 & 3 \\
                        5 & 1 & -1 & 2 \\
                        4 & -1 & 0 & 1
                    \end{vmatrix}
                    \sim
                    \begin{aligned}
                        F_3 + F_1 &\mapsto F_3 \\
                        F_3 - F_2 &\mapsto F_3 \\
                        F_4 + F_1 &\mapsto F_4 \\
                    \end{aligned}
                    \begin{vmatrix}
                        -4 & 1 & 1 & -2 \\
                        1 & 2 & 1 & 3 \\
                        0 & 0 & -1 & -3 \\
                        0 & 0 & 1 & -1
                    \end{vmatrix}
                    \\
                    &\sim
                    \begin{aligned}
                        F_2 + \frac{1}{4}F_1 &\mapsto F_2 \\
                        F_4 + F_3 &\mapsto F_4 \\
                    \end{aligned}
                    \begin{vmatrix}
                        -4 & 1 & 1 & -2 \\
                        0 & \frac{9}{4} & \frac{5}{4} & \frac{5}{2} \\
                        0 & 0 & -1 & -3 \\
                        0 & 0 & 0 & -4
                    \end{vmatrix}
                    =
                    -4 \cdot \frac{9}{4} \cdot -1 \cdot -4
                    = 9 \cdot -4
                    \\
                    &= -36
                \end{aligned}
            \]
            \[
                \begin{aligned}
                    \pdet{A_2}
                    &=
                    \begin{vmatrix}
                        1 & -4 & 1 & -2 \\
                        0 & 1 & 1 & 3 \\
                        2 & 5 & -1 & 2 \\
                        1 & 4 & 0 & 1
                    \end{vmatrix}
                    \sim
                    \begin{aligned}
                        F_3 - 2F_1 &\mapsto F_3 \\
                        F_4 - F_1 &\mapsto F_4 \\
                    \end{aligned}
                    \begin{vmatrix}
                        1 & -4 & 1 & -2 \\
                        0 & 1 & 1 & 3 \\
                        0 & 13 & -3 & 6 \\
                        0 & 8 & -1 & 3
                    \end{vmatrix}
                    \sim
                    \begin{aligned}
                        F_3 - 13F_F2 &\mapsto F_3 \\
                        F_4 - 8F_2 &\mapsto F_4 \\
                    \end{aligned}
                    \begin{vmatrix}
                        1 & -4 & 1 & -2 \\
                        0 & 1 & 1 & 3 \\
                        0 & 0 & -16 & -33 \\
                        0 & 0 & -9 & -21
                    \end{vmatrix}
                    \\
                    &\sim
                    \begin{aligned}
                        F_4 - \frac{9}{16}F_3 &\mapsto F_4
                    \end{aligned}
                    \begin{vmatrix}
                        1 & -4 & 1 & -2 \\
                        0 & 1 & 1 & 3 \\
                        0 & 0 & -16 & -33 \\
                        0 & 0 & 0 & -\frac{39}{16}
                    \end{vmatrix}
                    =
                    1 \cdot 1 \cdot -16 \cdot -\frac{39}{16}
                    \\
                    &= 39
                \end{aligned}
            \]
            \[
                \begin{aligned}
                    \pdet{A_3} 
                    &=
                    \begin{vmatrix}
                        1 & 1 & -4 & -2 \\
                        0 & 2 & 1 & 3 \\
                        2 & 1 & 5 & 2 \\
                        1 & -1 & 4 & 1
                    \end{vmatrix}
                    \sim
                    \begin{aligned}
                        F_3 - 2F_1 &\mapsto F_3 \\
                        F_4 - F_1 &\mapsto F_4 \\
                    \end{aligned}
                    \begin{vmatrix}
                        1 & 1 & -4 & -2 \\
                        0 & 2 & 1 & 3 \\
                        0 & -1 & 13 & 6 \\
                        0 & -2 & 8 & 3
                    \end{vmatrix}
                    \\
                    &\sim
                    \begin{aligned}
                        F_3 + \frac{1}{2}F_2 &\mapsto F_3 \\
                        F_4 + F_2 &\mapsto F_4 \\
                    \end{aligned}
                    \begin{vmatrix}
                        1 & 1 & -4 & -2 \\
                        0 & 2 & 1 & 3 \\
                        0 & 0 & \frac{27}{2} & \frac{15}{2} \\
                        0 & 0 & 9 & 6
                    \end{vmatrix}
                    \sim
                    \begin{aligned}
                        F_4 - \frac{2}{3} &\mapsto F_4
                    \end{aligned}
                    \begin{vmatrix}
                        1 & 1 & -4 & -2 \\
                        0 & 2 & 1 & 3 \\
                        0 & 0 & \frac{27}{2} & \frac{15}{2} \\
                        0 & 0 & 0 & 1
                    \end{vmatrix}
                    =
                    1 \cdot 2 \cdot \frac{27}{2} \cdot 1
                    \\
                    &= 27
                \end{aligned}
            \]
            \[
                \begin{aligned}
                    \pdet{A_4}
                    &=
                    \begin{vmatrix}
                        1 & 1 & 1 & -4 \\
                        0 & 2 & 1 & 1 \\
                        2 & 1 & -1 & 5 \\
                        1 & -1 & 0 & 4
                    \end{vmatrix}
                    \sim
                    \begin{aligned}
                        F_3 - 2F_1 &\mapsto F_3 \\
                        F_4 - F_1 &\mapsto F_4 \\
                    \end{aligned}
                    \begin{vmatrix}
                        1 & 1 & 1 & -4 \\
                        0 & 2 & 1 & 1 \\
                        0 & -1 & -3 & 13 \\
                        0 & -2 & -1 & 8
                    \end{vmatrix}
                    \\
                    &\sim
                    \begin{aligned}
                        F_3 + \frac{1}{2}F_2 \\
                        F_4 + F_2 &\mapsto F_4 \\
                    \end{aligned}
                    \begin{vmatrix}
                        1 & 1 & 1 & -4 \\
                        0 & 2 & 1 & 1 \\
                        0 & 0 & -\frac{5}{2} & \frac{25}{2} \\
                        0 & 0 & 0 & 9
                    \end{vmatrix}
                    = 1 \cdot 2 \cdot -\frac{5}{2} \cdot 9 = -5 \cdot 9
                    \\
                    &= -45
                \end{aligned}
            \]
            Ahora con los determinantes de las matrices obtenidas, además se tiene que \(A\) tiene inversa ya que \(\pdet{A} \neq 0\), podemos usar la 
            \textbf{regla de Cramer} para poder encontrar el valor de cada incognita del sistema de ecuaciones.
            \[
                \begin{aligned}
                    x &= \frac{\pdet{A_1}}{\pdet{A}} \\
                    &= \frac{-36}{-30} \\
                    &= \frac{6}{5}
                \end{aligned}
                \hspace{0.5cm}
                \begin{aligned}
                    y &= \frac{\pdet{A_2}}{\pdet{A}} \\
                    &= \frac{39}{-30} \\
                    &= -\frac{13}{10}
                \end{aligned}
                \hspace{0.5cm}
                \begin{aligned}
                    z &= \frac{\pdet{A_3}}{\pdet{A}} \\
                    &= \frac{27}{-30} \\
                    &= -\frac{9}{10}
                \end{aligned}
                \hspace{0.5cm}
                \begin{aligned}
                    w &= \frac{\pdet{A_4}}{\pdet{A}} \\
                    &= \frac{-45}{-30} \\
                    &= \frac{3}{2}
                \end{aligned}
            \]
        \item 
            \(
                \left\{
                \begin{aligned}
                    &2x &+ 3y &+ 7z &= 2 \\
                    &-2x &\phantom{+0y} &- 4z &= 0 \\
                    &x &+ 2y &+ 4z &= 0
                \end{aligned}
                \right.
            \) \\
            Primero, creamos la matriz de coeficientes equivalente a este sistema de ecuaciones y calculamos su determinante. 
            Para así luego, vamos a intercambiar cada columna de coeficientes por la columna de los terminos independientes,
            la cual tambien se va a obtener sus respectivos determinantes.
            \[
                A = 
                \begin{pmatrix}
                    2 & 3 & 7 \\
                    -2 & 0 & -4 \\
                    1 & 2 & 4
                \end{pmatrix}
                \hspace{0.5cm}
                A_1 =
                \begin{pmatrix}
                    2 & 3 & 7 \\
                    0 & 0 & -4 \\
                    0 & 2 & 4
                \end{pmatrix}
                \hspace{0.5cm}
                A_2 =
                \begin{pmatrix}
                    2 & 2 & 7 \\
                    -2 & 0 & -4 \\
                    1 & 0 & 4
                \end{pmatrix}
                \hspace{0.5cm}
                A_3 =
                \begin{pmatrix}
                    2 & 3 & 2 \\
                    -2 & 0 & 0 \\
                    1 & 2 & 0
                \end{pmatrix}
            \]
            \[
                \begin{aligned}
                    \pdet{A} =
                    \begin{vmatrix}
                        2 & 3 & 7 \\
                        -2 & 0 & -4 \\
                        1 & 2 & 4
                    \end{vmatrix}
                    &=
                    \left(-1\right){}^{1 + 1}\cdot 
                    2 \cdot 
                    \begin{vmatrix}
                        0 & -4 \\
                        2 & 4 
                    \end{vmatrix}
                    + 
                    \left(-1\right){}^{1 + 2} \cdot
                    3 \cdot
                    \begin{vmatrix}
                        -2 & -4 \\
                        1 & 4
                    \end{vmatrix}
                    +
                    \left(-1\right){}^{1 + 3} \cdot
                    7 \cdot
                    \begin{vmatrix}
                        -2 & 0 \\
                        1 & 2
                    \end{vmatrix}
                    \\
                    &=
                    2 \cdot \left(8\right)
                    -3 \cdot \left(-8 + 4\right)
                    +7 \cdot \left(-4\right)
                    \\
                    &= 16 + 12 - 28
                    = 28 - 28
                    \\
                    &= 0
                \end{aligned}
            \]
            \[
                \begin{aligned}
                    \pdet{A_1} =
                    \begin{vmatrix}
                        2 & 3 & 7 \\
                        0 & 0 & -4 \\
                        0 & 2 & 4
                    \end{vmatrix}
                    &=
                    \left(-1\right){}^{1 + 1} \cdot 
                    2 \cdot 
                    \begin{vmatrix}
                        0 & -4 \\
                        2 & 4
                    \end{vmatrix}
                    + \left(-1\right){}^{1 + 2} \cdot 
                    3 \cdot 
                    \begin{vmatrix}
                        0 & -4 \\
                        0 & 4
                    \end{vmatrix}
                    +\left(-1\right){}^{1 + 3} \cdot 
                    7 \cdot 
                    \begin{vmatrix}
                        0 & 0 \\
                        0 & 2
                    \end{vmatrix}
                    \\
                    &=
                    2 \cdot -8 
                    -3 \cdot 0
                    +7 \cdot 0
                    \\
                    &= -16
                \end{aligned}
            \]
            \[
                \begin{aligned}
                    \pdet{A_2} =
                    \begin{vmatrix}
                        2 & 2 & 7 \\
                        -2 & 0 & -4 \\
                        1 & 0 & 4
                    \end{vmatrix}
                    &=
                    \left(-1\right){}^{1 + 1} \cdot 
                    2 \cdot 
                    \begin{vmatrix}
                        0 & -4 \\
                        0 & 4
                    \end{vmatrix}
                    + \left(-1\right){}^{1 + 2} \cdot 
                    2 \cdot 
                    \begin{vmatrix}
                        -2 & -4 \\
                        1 & 4
                    \end{vmatrix}
                    +\left(-1\right){}^{1 + 3} \cdot 
                    7 \cdot 
                    \begin{vmatrix}
                        -2 & 0 \\
                        1 & 0
                    \end{vmatrix}
                    \\
                    &=
                    2 \cdot 0
                    -2 \cdot \left(-8 + 4\right)
                    +7 \cdot 0
                    = -2 \cdot -4
                    \\
                    &= 8
                \end{aligned}
            \]
            \[
                \begin{aligned}
                    \pdet{A_3} =
                    \begin{vmatrix}
                        2 & 3 & 2 \\
                        -2 & 0 & 0 \\
                        1 & 2 & 0
                    \end{vmatrix}
                    &=
                    \left(-1\right){}^{1 + 1} \cdot 
                    2 \cdot 
                    \begin{vmatrix}
                        0 & 0 \\
                        2 & 0
                    \end{vmatrix}
                    + \left(-1\right){}^{1 + 2} \cdot 
                    3 \cdot 
                    \begin{vmatrix}
                        -2 & 0 \\
                        1 & 0
                    \end{vmatrix}
                    +\left(-1\right){}^{1 + 3} \cdot 
                    2 \cdot 
                    \begin{vmatrix}
                        -2 & 0 \\
                        1 & 2
                    \end{vmatrix}
                    \\
                    &=
                    2 \cdot 0
                    -3 \cdot 0
                    +2 \cdot -4
                    \\
                    &= -8
                \end{aligned}
            \]
            Ahora con cada determinante, al tener \(A\) un determinante de cero sabemos que \(A\) es \textbf{no es invertible}. 
            Por lo cual no tenemos lo necesario para poder aplicar la \textbf{Regla de Cramer}.
    \end{enumerate}
