\documentclass{article}

\def\MyClass{Álgebra Lineal}
\def\MyTitle{Taller 2}
\def\MyAuthor{Martín Steven Hernández Ortiz}
\def\MyEmail{mahernandezor@unal.edu.co}
\def\MyDate{\today}

\def\TemplatePath{../template/}


%%% Template Packages %%%

\usepackage{graphicx} % Images
\usepackage{tcolorbox} % Color Box
\usepackage[%
vmargin=2.25cm,%
hmargin=2.25cm%
]{geometry} % Page Geometry
\usepackage{fancyhdr} % Header / Footer Styles

\usepackage{ragged2e} % Text Align
\usepackage{amsmath} % Math Align

\usepackage{booktabs} % Tables Package

\usepackage{mathtools} % Math General
\usepackage{amsthm} % Math Envs
\usepackage{unicode-math} % Math Symbols

\usepackage{polyglossia} % Language
    \setdefaultlanguage{spanish}

\renewcommand{\thefootnote}{\Roman{footnote}} % Changing footnotes arabic to roman numbers

%%% Template Styles %%%

% Header / Footer Styles
\pagestyle{fancy}
\RenewDocumentCommand{\headrule}{}{%
    \rule[0.1cm]{\textwidth}{0.1mm}%
}
\RenewDocumentCommand{\footrule}{}{%
    \rule[0.1cm]{\textwidth}{0.1mm}%
}

\fancyhf[HC]{{\slshape \MyTitle{}}}

% Redefine \maketitle
\RenewDocumentCommand{\maketitle}{s}{%
    \begin{@twocolumntrue}%
        \begin{minipage}{0.3\textwidth}%
            \includegraphics[width=0.85\textwidth]{../template/src/unal_logo.pdf}%
        \end{minipage}%
        \begin{minipage}{0.7\textwidth}{{%
        \begin{Center}%
            {\large \itshape \MyClass{}} \\[1ex]
            {\huge  \slshape \MyTitle{}} \\[4ex]
            {\Large  \MyAuthor{}} \\[0ex]
            {\small  \MyEmail{}} \\[4ex] 
            \MyDate{}
        \end{Center}%
        }}%
        \end{minipage}%
    \end{@twocolumntrue}%
    \vspace{0.5cm}%
    \begin{Center}%
        \rule[0cm]{\textwidth}{0.1mm}%
    \end{Center}%
    \vspace{0.5cm}%
}

%%% Template Math stuff %%%

% Theorems
\newtheorem{TMPMathTheorem}{Teorema}
\NewDocumentEnvironment{theorem}{+b} {%
    \begin{TMPMathTheorem}%
        #1 %
    \end{TMPMathTheorem}%
} {}

% Collorary
\newtheorem{TMPMathCorollary}[TMPMathTheorem]{Corolario}
\NewDocumentEnvironment{corollary}{+b} {%
    \begin{TMPMathCorollary}%
        #1 %
    \end{TMPMathCorollary}%
} {}

% Definitions
\newtheorem{TMPMathDefinition}[TMPMathTheorem]{Definición}
\NewDocumentEnvironment{definition}{+b} {%
    \begin{tcolorbox}[left=0mm,right=0mm]%
        \begin{TMPMathDefinition}%
            #1 %
            \begin{FlushRight}%
                \(\bigtriangleup{}\)%
            \end{FlushRight}%
        \end{TMPMathDefinition}%
    \end{tcolorbox}%
} {}



\begin{document}
\maketitle

\begin{enumerate}
    \setcounter{enumi}{6}
    \item Al resolver el sistema homogéneo de ecuaciones lineales \(Ax = \vec{0}\) se obtuvo la siguiente forma escalonada reducida de \(A\)
        \[
            \begin{bmatrix}
                1 & 0 & 1 & 0 & 0 \\
                0 & 1 & 1 & 0 & 1 \\
                0 & 0 & 0 & 1 & -1 \\
                0 & 0 & 0 & 0 & 0 \\
            \end{bmatrix}
        \]
        Diga cuáles de los siguientes vectores de \(\mathbb{R}^5\) son soluciones del sistema original
        \[
            v_1 = 
            \begin{pmatrix}
                -3 \\ -3 \\ 3 \\ 0 \\ 0
            \end{pmatrix}
            \hspace{1cm}
            v_2 = 
            \begin{pmatrix}
                0 \\ -1 \\ 0 \\ 1 \\ 1
            \end{pmatrix}
            \hspace{1cm}
            v_3 = 
            \begin{pmatrix}
                -\alpha \\ -\alpha - 1 \\ -\alpha \\ 1 \\ 1
            \end{pmatrix}
            \hspace{1cm}
            v_4 = \lambda v_3 \hspace{3ex} \lambda \in \mathbb{R}
        \]
        Veamos el producto de la matriz con cada vector y si su resultado es \(\vec{0}\), entonces sí son una solución para el sistema homogéneo de ecuaciones lineales.
        \ResetCases
        \begin{mathcase}{\(v_1 \cdot A\)}
            \[
                \begin{aligned}
                    Av_1 &= 
                    -3
                    \begin{pmatrix}
                        1 \\ 0 \\ 0 \\ 0
                    \end{pmatrix}
                    -3
                    \begin{pmatrix}
                        0 \\ 1 \\ 0 \\ 0
                    \end{pmatrix}
                    +3
                    \begin{pmatrix}
                        1 \\ 1 \\ 0 \\ 0
                    \end{pmatrix}
                    +0
                    \begin{pmatrix}
                        0 \\ 0 \\ 1 \\ 0
                    \end{pmatrix}
                    +0
                    \begin{pmatrix}
                        0 \\ 1 \\ -1 \\ 0
                    \end{pmatrix} 
                    \\
                    &= 
                    \begin{pmatrix}
                        -3 \\ 0 \\ 0 \\ 0
                    \end{pmatrix}
                    +
                    \begin{pmatrix}
                        0 \\ -3 \\ 0 \\ 0
                    \end{pmatrix}
                    +
                    \begin{pmatrix}
                        3 \\ 3 \\ 0 \\ 0
                    \end{pmatrix}
                    +
                    \vec{0}
                    +
                    \vec{0} 
                    =
                    \begin{pmatrix}
                        0 \\ 0 \\ 0 \\ 0
                    \end{pmatrix}
                    =
                    \vec{0}
                \end{aligned}
            \]
            Para este caso, \(v_1\) es una combinación lineal de las columnas de \(A\).
        \end{mathcase}
        \begin{mathcase}{\(v_2 \cdot A\)}
            \[
                \begin{aligned}
                    Av_2 &= 
                    0
                    \begin{pmatrix}
                        1 \\ 0 \\ 0 \\ 0
                    \end{pmatrix}
                    -1
                    \begin{pmatrix}
                        0 \\ 1 \\ 0 \\ 0
                    \end{pmatrix}
                    +0
                    \begin{pmatrix}
                        1 \\ 1 \\ 0 \\ 0
                    \end{pmatrix}
                    +1
                    \begin{pmatrix}
                        0 \\ 0 \\ 1 \\ 0
                    \end{pmatrix}
                    +1
                    \begin{pmatrix}
                        0 \\ 1 \\ -1 \\ 0
                    \end{pmatrix} 
                    \\
                    &=
                    \vec{0}
                    +
                    \begin{pmatrix}
                        0 \\ -1 \\ 0 \\ 0
                    \end{pmatrix}
                    +
                    \vec{0}
                    +
                    \begin{pmatrix}
                        0 \\ 0 \\ 1 \\ 0
                    \end{pmatrix}
                    +
                    \begin{pmatrix}
                        0 \\ 1 \\ -1 \\ 0
                    \end{pmatrix} 
                    = 
                    \begin{pmatrix}
                        0 \\ 0 \\ 0 \\ 0
                    \end{pmatrix}
                    = 
                    \vec{0}
                \end{aligned}
            \]
            De igual manera para este caso, \(v_2\) es una solución 
            para el sistema de ecuaciones lineales \(\left[A \mid 0\right]\).
        \end{mathcase}
        \begin{mathcase}{\(v_3 \cdot A\)}
            \[
                \begin{aligned}
                    Av_3 &= 
                    -\alpha
                    \begin{pmatrix}
                        1 \\ 0 \\ 0 \\ 0
                    \end{pmatrix}
                    +(-\alpha - 1)
                    \begin{pmatrix}
                        0 \\ 1 \\ 0 \\ 0
                    \end{pmatrix}
                    -\alpha
                    \begin{pmatrix}
                        1 \\ 1 \\ 0 \\ 0
                    \end{pmatrix}
                    +1
                    \begin{pmatrix}
                        0 \\ 0 \\ 1 \\ 0
                    \end{pmatrix}
                    +1
                    \begin{pmatrix}
                        0 \\ 1 \\ -1 \\ 0
                    \end{pmatrix} 
                    \\
                    &=
                    \begin{pmatrix}
                        -\alpha \\ 0 \\ 0 \\ 0
                    \end{pmatrix}
                    +
                    \begin{pmatrix}
                        0 \\ -\alpha - 1 \\ 0 \\ 0
                    \end{pmatrix}
                    +
                    \begin{pmatrix}
                        -\alpha \\ -\alpha \\ 0 \\ 0
                    \end{pmatrix}
                    +
                    \begin{pmatrix}
                        0 \\ 0 \\ 1 \\ 0
                    \end{pmatrix}
                    +
                    \begin{pmatrix}
                        0 \\ 1 \\ -1 \\ 0
                    \end{pmatrix} 
                    = 
                    \begin{pmatrix}
                        -2\alpha \\ -2\alpha \\ 0 \\ 0
                    \end{pmatrix}
                \end{aligned}
            \]
            En el caso que \(\alpha = 0\), se va a tener que \(Av_3 = \vec{0}\), 
            es decir, \(v_3 \in N_{A}\).
        \end{mathcase}
        \begin{mathcase}{\(v_4 \cdot A\)}
            Este caso va a ser similar al caso de \(v_3\)
            \[
                \begin{aligned}
                    Av_4 &= 
                    -\lambda\alpha
                    \begin{pmatrix}
                        1 \\ 0 \\ 0 \\ 0
                    \end{pmatrix}
                    +(-\lambda\alpha - \lambda)
                    \begin{pmatrix}
                        0 \\ 1 \\ 0 \\ 0
                    \end{pmatrix}
                    -\lambda\alpha
                    \begin{pmatrix}
                        1 \\ 1 \\ 0 \\ 0
                    \end{pmatrix}
                    +\lambda
                    \begin{pmatrix}
                        0 \\ 0 \\ 1 \\ 0
                    \end{pmatrix}
                    +\lambda
                    \begin{pmatrix}
                        0 \\ 1 \\ -1 \\ 0
                    \end{pmatrix} 
                    \\
                    &=
                    \begin{pmatrix}
                        -\lambda\alpha \\ 0 \\ 0 \\ 0
                    \end{pmatrix}
                    +
                    \begin{pmatrix}
                        0 \\ -\lambda\alpha - \lambda \\ 0 \\ 0
                    \end{pmatrix}
                    +
                    \begin{pmatrix}
                        -\lambda\alpha \\ -\lambda\alpha \\ 0 \\ 0
                    \end{pmatrix}
                    +
                    \begin{pmatrix}
                        0 \\ 0 \\ \lambda \\ 0
                    \end{pmatrix}
                    +
                    \begin{pmatrix}
                        0 \\ \lambda \\ -\lambda \\ 0
                    \end{pmatrix} 
                    = 
                    \begin{pmatrix}
                        -2\lambda\alpha \\ -2\lambda\alpha \\ 0 \\ 0
                    \end{pmatrix}
                \end{aligned}
            \]
            En este caso es si \(\lambda = 0\) o \(\alpha = 0\) entonces \(Av_4 = \vec{0}\).
        \end{mathcase}
    \setcounter{enumi}{12}
    \item Determine para que valores de \(\alpha\), tal que
        \(
            Gen \left\{
                \begin{pmatrix}
                    0 \\ 1 \\ \alpha
                \end{pmatrix},
                \begin{pmatrix}
                    \alpha \\ 1 \\ 0
                \end{pmatrix},
                \begin{pmatrix}
                    0 \\ \alpha \\ 1
                \end{pmatrix}
            \right\}
            = \mathbb{R}^3
        \) \\
        Para que el conjunto generado sea igual a \(\mathbb{R}^3\) necesitamos que sus vectores o el conjunto generador sean \(l.i\)
    \setcounter{enumi}{22}
    \item Dados los vectores en \(\mathbb{R}^n\), \(v, v, w\) Si \(H = \left\{u, v, w\right\}\) es \(l.i\), determine cuál de la siguientes afirmaciones es falsa.
        \begin{enumerate}[label=\listAlph]
            \item \(Gen\left\{u + v, v − w\right\} \subseteq Gen\left(H\right)\)
            \item \(H \subseteq Gen\left(H\right)\)
            \item \(\vec{0} \in Gen\left(H\right)\)
            \item \(\left\{u, u + v, u + v + w\right\}\) es \(l.d\)
            \item \(w \in H\)
        \end{enumerate}
    \item Considiere los vectores en \(\mathbb{R}^3\) dados por 
        \(
            u = 
            \begin{pmatrix}
                2 \\ -1 \\ 1
            \end{pmatrix},
            v =
            \begin{pmatrix}
                0 \\ 1 \\ 1
            \end{pmatrix}
            \text{ y }
            w = 
            \begin{pmatrix}
                2 \\ 1 \\ 3
            \end{pmatrix}
        \). \\
        Muestre que \(Gen\left\{u + v, v - w\right\} \subseteq Gen\left\{u, v, w\right\}\), y determine si estos conjuntos son iguales.

    \setcounter{enumi}{27}
    \item Sean \(A\) una matriz de \(m\) filas y \(n\) columnas y \(U\) una matriz escalonada equivalente a \(A\). Si \textbf{para cualquier vector \(b\)}
        de \(\mathbb{R}^m\), el sistema de ecuaciones lineales, cuya matriz aumentada es \(\left[A \mid b\right]\), \textbf{tiene solución única}, 
        determine cuales de las siguientes afirmaciones son verdaderas. Justifique su respuesta.
        \begin{enumerate}[label=\listAlph]
            \item El vector \(b\) es combinación lineal de las columnas de \(A\).
            \item El vector \(b\) es combinación lineal de las columnas de \(U\).
            \item Cada fila de \(U\) tiene un pivote.
            \item Cada columna de \(U\) tiene un pivote.
            \item La matriz \(U\) tiene \(n\) pivotes.
            \item La matriz \(U\) tiene \(m\) pivotes.
            \item \(m = n\).
            \item Las columnas de \(A\) generan a \(\mathbb{R}^m\)
        \end{enumerate}
    \item Sean \(A\) una matriz de \(m\) filas y \(n\) columnas y \(U\) una matriz escalonada equivalente a \(A\). 
        Si \textbf{para cualquier vector \(b\) de} \(\mathbb{R}^m\),
        el sistema de ecuaciones lineales, cuya matriz aumentada es \(\left[A \mid b\right]\), 
        \textbf{tiene infinitas soluciones},
        determine cuales de las siguientes afirmaciones son verdaderas. Justifique su respuesta.
        \begin{enumerate}[label=\listAlph]
            \item El vector \(b\) es combinación lineal de las columnas de \(A\).
            \item El vector \(b\) es combinación lineal de las columnas de \(U\).
            \item Cada fila de \(U\) tiene un pivote.
            \item Cada columna de \(U\) tiene un pivote.
            \item La matriz \(U\) tiene \(n\) pivotes.
            \item La matriz \(U\) tiene \(m\) pivotes.
            \item \(m < n\).
            \item Las columnas de \(A\) generan a \(\mathbb{R}^m\).
        \end{enumerate}
    \item Sean \(A\) una matriz de \(m\) filas y \(n\) columnas y \(U\) una matriz escalonada equivalente a \(A\). 
        Si \textbf{para un vector \(b\)} de \(\mathbb{R}^m\),
        el sistema de ecuaciones lineales, cuya matriz aumentada es \(\left[A \mid b\right]\),
        \textbf{es inconsistente}, 
        determine cuales de las siguientes afirmaciones son verdaderas y responda a las preguntas formuladas. 
        Justifique su respuesta.
        \begin{enumerate}[label=\listAlph]
            \item El vector \(b\) es combinación lineal de las columnas de \(A\).
            \item El vector \(b\) puede ser combinación lineal de las columnas de \(U\).
            \item Cada fila de \(U\) tiene un pivote.
            \item El vector \(b\) puede ser \(\vec{0}\)?
            \item El vector \(b\) puede ser un múltiplo de alguna de las columnas de \(A\)?
            \item El vector \(b\) puede ser la suma de las columnas de \(A\)?
            \item Las columnas de \(A\) generan a \(\mathbb{R}^m\).
            \item ¿Qué puede decirse del número de pivotes de \(U\)?
        \end{enumerate}
    \item ¿%
        \(
            Gen
            \left\{
                \begin{pmatrix}
                    1 \\ 0 \\ -3
                \end{pmatrix},
                \begin{pmatrix}
                    4 \\ -10 \\ 2
                \end{pmatrix}
            \right\}
            =
            Gen
            \left\{
                \begin{pmatrix}
                    0 \\ -5 \\ 8
                \end{pmatrix},
                \begin{pmatrix}
                    -1 \\ 5 \\ -5
                \end{pmatrix}
            \right\}
        \)?
\end{enumerate}
\end{document}
