\documentclass{article}

\def\MyClass{Álgebra Lineal}
\def\MyTitle{Taller 2}
\def\MyAuthor{Martín Steven Hernández Ortiz}
\def\MyEmail{mahernandezor@unal.edu.co}
\def\MyDate{\today}

\def\TemplatePath{../template/}


%%% Template Packages %%%

\usepackage{graphicx} % Images
\usepackage{tcolorbox} % Color Box
\usepackage[%
vmargin=2.25cm,%
hmargin=2.25cm%
]{geometry} % Page Geometry
\usepackage{fancyhdr} % Header / Footer Styles

\usepackage{ragged2e} % Text Align
\usepackage{amsmath} % Math Align

\usepackage{booktabs} % Tables Package

\usepackage{mathtools} % Math General
\usepackage{amsthm} % Math Envs
\usepackage{unicode-math} % Math Symbols

\usepackage{polyglossia} % Language
    \setdefaultlanguage{spanish}

\renewcommand{\thefootnote}{\Roman{footnote}} % Changing footnotes arabic to roman numbers

%%% Template Styles %%%

% Header / Footer Styles
\pagestyle{fancy}
\RenewDocumentCommand{\headrule}{}{%
    \rule[0.1cm]{\textwidth}{0.1mm}%
}
\RenewDocumentCommand{\footrule}{}{%
    \rule[0.1cm]{\textwidth}{0.1mm}%
}

\fancyhf[HC]{{\slshape \MyTitle{}}}

% Redefine \maketitle
\RenewDocumentCommand{\maketitle}{s}{%
    \begin{@twocolumntrue}%
        \begin{minipage}{0.3\textwidth}%
            \includegraphics[width=0.85\textwidth]{../template/src/unal_logo.pdf}%
        \end{minipage}%
        \begin{minipage}{0.7\textwidth}{{%
        \begin{Center}%
            {\large \itshape \MyClass{}} \\[1ex]
            {\huge  \slshape \MyTitle{}} \\[4ex]
            {\Large  \MyAuthor{}} \\[0ex]
            {\small  \MyEmail{}} \\[4ex] 
            \MyDate{}
        \end{Center}%
        }}%
        \end{minipage}%
    \end{@twocolumntrue}%
    \vspace{0.5cm}%
    \begin{Center}%
        \rule[0cm]{\textwidth}{0.1mm}%
    \end{Center}%
    \vspace{0.5cm}%
}

%%% Template Math stuff %%%

% Theorems
\newtheorem{TMPMathTheorem}{Teorema}
\NewDocumentEnvironment{theorem}{+b} {%
    \begin{TMPMathTheorem}%
        #1 %
    \end{TMPMathTheorem}%
} {}

% Collorary
\newtheorem{TMPMathCorollary}[TMPMathTheorem]{Corolario}
\NewDocumentEnvironment{corollary}{+b} {%
    \begin{TMPMathCorollary}%
        #1 %
    \end{TMPMathCorollary}%
} {}

% Definitions
\newtheorem{TMPMathDefinition}[TMPMathTheorem]{Definición}
\NewDocumentEnvironment{definition}{+b} {%
    \begin{tcolorbox}[left=0mm,right=0mm]%
        \begin{TMPMathDefinition}%
            #1 %
            \begin{FlushRight}%
                \(\bigtriangleup{}\)%
            \end{FlushRight}%
        \end{TMPMathDefinition}%
    \end{tcolorbox}%
} {}



\begin{document}
\maketitle

\begin{enumerate}
    \setcounter{enumi}{6}
    \item Al resolver el sistema homogéneo de ecuaciones lineales \(Ax = \vec{0}\) se obtuvo la siguiente forma escalonada reducida de \(A\)
        \[
            \begin{bmatrix}
                1 & 0 & 1 & 0 & 0 \\
                0 & 1 & 1 & 0 & 1 \\
                0 & 0 & 0 & 1 & -1 \\
                0 & 0 & 0 & 0 & 0 \\
            \end{bmatrix}
        \]
        Diga cuáles de los siguientes vectores de \(\mathbb{R}^5\) son soluciones del sistema original
        \[
            v_1 = 
            \begin{pmatrix}
                -3 \\ -3 \\ 3 \\ 0 \\ 0
            \end{pmatrix}
            \hspace{1cm}
            v_2 = 
            \begin{pmatrix}
                0 \\ -1 \\ 0 \\ 1 \\ 1
            \end{pmatrix}
            \hspace{1cm}
            v_3 = 
            \begin{pmatrix}
                -\alpha \\ -\alpha - 1 \\ -\alpha \\ 1 \\ 1
            \end{pmatrix}
            \hspace{1cm}
            v_4 = \lambda v_3 \hspace{3ex} \lambda \in \mathbb{R}
        \]
        Veamos el producto de la matriz con cada vector y si su resultado es \(\vec{0}\), entonces sí son una solución para el sistema homogéneo de ecuaciones lineales.
        \ResetCases
        \begin{mathcase}{\(v_1 \cdot A\)}
            \[
                \begin{aligned}
                    Av_1 &= 
                    -3
                    \begin{pmatrix}
                        1 \\ 0 \\ 0 \\ 0
                    \end{pmatrix}
                    -3
                    \begin{pmatrix}
                        0 \\ 1 \\ 0 \\ 0
                    \end{pmatrix}
                    +3
                    \begin{pmatrix}
                        1 \\ 1 \\ 0 \\ 0
                    \end{pmatrix}
                    +0
                    \begin{pmatrix}
                        0 \\ 0 \\ 1 \\ 0
                    \end{pmatrix}
                    +0
                    \begin{pmatrix}
                        0 \\ 1 \\ -1 \\ 0
                    \end{pmatrix} 
                    \\
                    &= 
                    \begin{pmatrix}
                        -3 \\ 0 \\ 0 \\ 0
                    \end{pmatrix}
                    +
                    \begin{pmatrix}
                        0 \\ -3 \\ 0 \\ 0
                    \end{pmatrix}
                    +
                    \begin{pmatrix}
                        3 \\ 3 \\ 0 \\ 0
                    \end{pmatrix}
                    +
                    \vec{0}
                    +
                    \vec{0} 
                    =
                    \begin{pmatrix}
                        0 \\ 0 \\ 0 \\ 0
                    \end{pmatrix}
                    =
                    \vec{0}
                \end{aligned}
            \]
            Para este caso, \(v_1\) es una combinación lineal de las columnas de \(A\).
        \end{mathcase}
        \begin{mathcase}{\(v_2 \cdot A\)}
            \[
                \begin{aligned}
                    Av_2 &= 
                    0
                    \begin{pmatrix}
                        1 \\ 0 \\ 0 \\ 0
                    \end{pmatrix}
                    -1
                    \begin{pmatrix}
                        0 \\ 1 \\ 0 \\ 0
                    \end{pmatrix}
                    +0
                    \begin{pmatrix}
                        1 \\ 1 \\ 0 \\ 0
                    \end{pmatrix}
                    +1
                    \begin{pmatrix}
                        0 \\ 0 \\ 1 \\ 0
                    \end{pmatrix}
                    +1
                    \begin{pmatrix}
                        0 \\ 1 \\ -1 \\ 0
                    \end{pmatrix} 
                    \\
                    &=
                    \vec{0}
                    +
                    \begin{pmatrix}
                        0 \\ -1 \\ 0 \\ 0
                    \end{pmatrix}
                    +
                    \vec{0}
                    +
                    \begin{pmatrix}
                        0 \\ 0 \\ 1 \\ 0
                    \end{pmatrix}
                    +
                    \begin{pmatrix}
                        0 \\ 1 \\ -1 \\ 0
                    \end{pmatrix} 
                    = 
                    \begin{pmatrix}
                        0 \\ 0 \\ 0 \\ 0
                    \end{pmatrix}
                    = 
                    \vec{0}
                \end{aligned}
            \]
            De igual manera para este caso, \(v_2\) es una solución 
            para el sistema de ecuaciones lineales \(\left[A \mid 0\right]\).
        \end{mathcase}
        \begin{mathcase}{\(v_3 \cdot A\)}
            \[
                \begin{aligned}
                    Av_3 &= 
                    -\alpha
                    \begin{pmatrix}
                        1 \\ 0 \\ 0 \\ 0
                    \end{pmatrix}
                    +(-\alpha - 1)
                    \begin{pmatrix}
                        0 \\ 1 \\ 0 \\ 0
                    \end{pmatrix}
                    -\alpha
                    \begin{pmatrix}
                        1 \\ 1 \\ 0 \\ 0
                    \end{pmatrix}
                    +1
                    \begin{pmatrix}
                        0 \\ 0 \\ 1 \\ 0
                    \end{pmatrix}
                    +1
                    \begin{pmatrix}
                        0 \\ 1 \\ -1 \\ 0
                    \end{pmatrix} 
                    \\
                    &=
                    \begin{pmatrix}
                        -\alpha \\ 0 \\ 0 \\ 0
                    \end{pmatrix}
                    +
                    \begin{pmatrix}
                        0 \\ -\alpha - 1 \\ 0 \\ 0
                    \end{pmatrix}
                    +
                    \begin{pmatrix}
                        -\alpha \\ -\alpha \\ 0 \\ 0
                    \end{pmatrix}
                    +
                    \begin{pmatrix}
                        0 \\ 0 \\ 1 \\ 0
                    \end{pmatrix}
                    +
                    \begin{pmatrix}
                        0 \\ 1 \\ -1 \\ 0
                    \end{pmatrix} 
                    = 
                    \begin{pmatrix}
                        -2\alpha \\ -2\alpha \\ 0 \\ 0
                    \end{pmatrix}
                \end{aligned}
            \]
            En el caso que \(\alpha\) sea igual a 0, se va a tener que \(Av_3\) es igual a \(\vec{0}\), 
            es decir, \(v_3 \in N_{A}\).
        \end{mathcase}
        \begin{mathcase}{\(v_4 \cdot A\)}
            Este caso va a ser similar al caso de \(v_3\)
            \[
                \begin{aligned}
                    Av_4 &= 
                    -\lambda\alpha
                    \begin{pmatrix}
                        1 \\ 0 \\ 0 \\ 0
                    \end{pmatrix}
                    +(-\lambda\alpha - \lambda)
                    \begin{pmatrix}
                        0 \\ 1 \\ 0 \\ 0
                    \end{pmatrix}
                    -\lambda\alpha
                    \begin{pmatrix}
                        1 \\ 1 \\ 0 \\ 0
                    \end{pmatrix}
                    +\lambda
                    \begin{pmatrix}
                        0 \\ 0 \\ 1 \\ 0
                    \end{pmatrix}
                    +\lambda
                    \begin{pmatrix}
                        0 \\ 1 \\ -1 \\ 0
                    \end{pmatrix} 
                    \\
                    &=
                    \begin{pmatrix}
                        -\lambda\alpha \\ 0 \\ 0 \\ 0
                    \end{pmatrix}
                    +
                    \begin{pmatrix}
                        0 \\ -\lambda\alpha - \lambda \\ 0 \\ 0
                    \end{pmatrix}
                    +
                    \begin{pmatrix}
                        -\lambda\alpha \\ -\lambda\alpha \\ 0 \\ 0
                    \end{pmatrix}
                    +
                    \begin{pmatrix}
                        0 \\ 0 \\ \lambda \\ 0
                    \end{pmatrix}
                    +
                    \begin{pmatrix}
                        0 \\ \lambda \\ -\lambda \\ 0
                    \end{pmatrix} 
                    = 
                    \begin{pmatrix}
                        -2\lambda\alpha \\ -2\lambda\alpha \\ 0 \\ 0
                    \end{pmatrix}
                \end{aligned}
            \]
            En este caso es si \(\lambda\) es igual a 0 o \(\alpha\) es igual a 0, entonces 
            vamos a tener que \(v_4\) es una solución del sistema homogéneo de ecuaciones lineales,
            o de forma equivalente, \(Av_4 = \vec{0}\).
        \end{mathcase}
    \setcounter{enumi}{12}
    \item Determine para que valores de \(\alpha\), tal que
        \(
            Gen \left\{
                \begin{pmatrix}
                    0 \\ 1 \\ \alpha
                \end{pmatrix},
                \begin{pmatrix}
                    \alpha \\ 1 \\ 0
                \end{pmatrix},
                \begin{pmatrix}
                    0 \\ \alpha \\ 1
                \end{pmatrix}
            \right\}
            = \mathbb{R}^3
        \) \\
        Para que el conjunto generado sea igual a \(\mathbb{R}^3\), o de manera similar, que el generador pueda crear cualquier vector en \(\mathbb{R}^3\). 
        Para lograr esto primero miremos que un generador de los vectores canónicos en \(\mathbb{R}^3\) puede crear cualquier vector en \(\mathbb{R}^3\).
        Sin embargo, solo para confirmar que esto es verdadero, probemos una generalización en \(R^n\) de esta afirmación.
        \begin{proof}
            Debemos probar que si contamos con los vectores canónicos \(e_1, e_2, \ldots, e_n\) en \(\mathbb{R}^n\), el conjunto generado de estos vectores va a ser igual a \(\mathbb{R}^n\),
            es decir:
            \[
                Gen\left\{e_1, e_2, \ldots, e_n\right\} = \mathbb{R}^3
            \]
            Primero, si tenemos un vector \(u\) en \(\mathbb{R}^n\) tal que todos sus componentes sean elementos arbitrarios pero fijos en \(\mathbb{R}\),
            podemos realizar una combinación lineal con el conjunto generador
            \[
                u_1 
                \cdot 
                \begin{pmatrix}
                    1 \\ 0 \\ \vdots \\ 0
                \end{pmatrix}
                +u_2
                \cdot
                \begin{pmatrix}
                    0 \\ 1 \\ \vdots \\ 0
                \end{pmatrix}
                +
                \cdots
                +u_n
                \cdot
                \begin{pmatrix}
                    0 \\ 0 \\ \vdots \\ 1
                \end{pmatrix}
                =
                \begin{pmatrix}
                    u_1 \\ 0 \\ \vdots \\ 0
                \end{pmatrix}
                +
                \begin{pmatrix}
                    0 \\ u_2 \\ \vdots \\ 0
                \end{pmatrix}
                +
                \cdots
                +
                \begin{pmatrix}
                    0 \\ 0 \\ \vdots \\ u_n
                \end{pmatrix}
                =
                u
            \]
            Obteniendo como resultado que al realizar la combinación lineal se obtiene el mismo vector.
            Con lo cual se puede concluir que, al realizar una combinación lineal entre el conjunto generador y cualquier vector se obtiene el mismo vector.
            Es decir que, apartir de este conjunto generador se puede obtener cualquier vector en \(\mathbb{R}^n\), o \(Gen\left\{e_1, e_2, \dots, e_n\right\} = \mathbb{R}^3\)
        \end{proof}
        Además, podemos que el conjunto generador \(\left\{e_1, e_2, \ldots, e_n\right\}\) es \(l.i\) 
        ya que la única combinación lineal que de como resultado a \(\vec{0}\) es con el mismo vector \(\vec{0}\).
        Y por un proceso similar, con \(e_1, e_2, e_3\) en \(\mathbb{R}^3\), tenemos que \(Gen\left\{e_1, e_2, e_3\right\} = \mathbb{R}^3\). Y que \(\left\{e_1, e_2, e_3\right\}\) es \(l.i\).
        \\
        Ahora, volviendo a nuestro problema original, podemos mirar un \(\alpha\) tal que el conjunto generador 
        \(
        \left\{
            \begin{pmatrix}
                0 \\ 1 \\ \alpha
            \end{pmatrix},
            \begin{pmatrix}
                \alpha \\ 1 \\ 0
            \end{pmatrix},
            \begin{pmatrix}
                0 \\ \alpha \\ 1
            \end{pmatrix}
        \right\}
        \)
        sea \(l.i\). Y probar, de forma similar, si este conjunto generador permite obtener a cualquier vector en \(\mathbb{R}^n\).
        Para encontrar el \(\alpha\) tal que el conjunto generador sea \(l.i\) vamos a crear un sistema homogéneo de ecuaciones lineales con el conjunto generador, 
        y tomando a \(\lambda_1, \lambda_2, \lambda_3\) como las variables del sistema. 
        También, por ahora, asuma que \(\alpha \neq 0\), más adelante se explicara el porqué de esa selección
        \[
            \begin{pmatrix}
                0 & \alpha & 0 \\
                1 & 1 & \alpha \\
                \alpha & 0 & 1
            \end{pmatrix}
            \cdot
            \begin{pmatrix}
                \lambda_1 \\ \lambda_2 \\ \lambda_3
            \end{pmatrix}
            =
            \vec{0}
            \hspace{1ex};\hspace{1ex}
            \begin{aligned}
                F_1 &\leftrightarrow F_2 \\
            \end{aligned}
            \begin{pmatrix}
                1 & 1 & \alpha \\
                0 & \alpha & 0 \\
                \alpha & 0 & 1
            \end{pmatrix}
            \sim
            \begin{aligned}
                \frac{1}{\alpha}F_2 &\mapsto F_2 \\
                F_1 - \frac{1}{\alpha}F_3 &\mapsto F_1 \\
            \end{aligned}
            \begin{pmatrix}
                0 & 1 & \alpha - \frac{1}{\alpha} \\
                0 & 1 & 0 \\
                \alpha & 0 & 1
            \end{pmatrix}
        \]
        Apartir de esta matriz equivalente podemos deducir los valores de \(\lambda_1, \lambda_2, \lambda_3\) y \(\alpha\)
        \[
            \begin{aligned}
                \lambda_2 &= 0 \\
                (\lambda_2 = 0) &\rightarrow \left(\alpha - \frac{1}{\alpha}\right) = 0 \vee \lambda_3 = 0 \\
            \end{aligned}
        \]
        \ResetCases{}
        \begin{mathcase}{\(\lambda_3 = 0\)}
            \[
                \lambda_3 = 0 \rightarrow \alpha = 0 \vee \lambda_1 = 0
            \]
            Sin embargo, como \(\alpha\) no puede ser igual a 0 en esta matriz equivalente, ya que tendriamos varias operaciones entre filas
            indefinidas donde tendriamos escalares iguales a \(\frac{1}{0}\). 
            Entonces en el caso que \(\alpha\) sea igual a 0, va a solo ser considerado en la matriz inicial, o en la que no se tiene ninguna operación entre filas.
            Y como por hipotesis del caso, \(\lambda_3\) es igual a 0, para que la implicación sea verdadera, \(\lambda_1\) va a tener que ser igual a 0.
        \end{mathcase}
        \begin{mathcase}{\(\alpha - \frac{1}{\alpha} = 0\)}
            \[
                \begin{aligned}
                    \alpha &= \frac{1}{\alpha} \\
                    \alpha \cdot \alpha = \alpha^2 &= 1 \\
                    \sqrt{\alpha^2} &= \sqrt{1} \\
                    |\alpha| &= 1 \\
                    \alpha = 1 &\vee \alpha = -1
                \end{aligned}
            \]
            Podemos verificar que \(\alpha - \frac{1}{\alpha} = 0\) para cada valor de \(\alpha\)
            \[
                \begin{gathered}
                    1 - \frac{1}{1} = 1 - 1 = 0 \\
                    -1 - \frac{1}{-1} = -1 - (-1) = -1 + 1 = 0
                \end{gathered}
            \]
            Entonces podemos tomar a \(\alpha\) como 1 o -1.
        \end{mathcase} \\
        Cabe aclarar que estos dos casos pueden ser verdaderos al mismo tiempo. Para confirmar que estos si sean los valores de cada variable y de \(\alpha\) 
        hagamos sustitución hacia atrás. Primero con los valores que estamos completamente segudos \((\lambda_2)\)
        \[
            \left\{
                \begin{aligned}
                    0 + 0 \cdot \alpha + 0 &= 0 \\
                    \lambda_1 + 0 + \lambda_3 \cdot \alpha &= 0 \\
                    \lambda_1 \cdot \alpha + 0 + \lambda_3 &= 0
                \end{aligned}
            \right.
        \]
        Ahora veamos los posibles casos de \(\alpha\)
        \[
            \alpha = 1
            \left\{
                \begin{aligned}
                    0 + 0 + 0 &= 0 \\
                    \lambda_1 + 0 + \lambda_3 &= 0 \\
                    \lambda_1 + 0 + \lambda_3 &= 0
                \end{aligned}
            \right.
            \hspace{1cm}
            \alpha = -1
            \left\{
                \begin{aligned}
                    0 + 0 + 0 &= 0 \\
                    \lambda_1 + 0 - \lambda_3 &= 0 \\
                    -\lambda_1 + 0 + \lambda_3 &= 0
                \end{aligned}
            \right.
            \hspace{1cm}
            \alpha = 0
            \left\{
                \begin{aligned}
                    0 + 0 + 0 &= 0 \\
                    \lambda_1 + 0 + 0 &= 0 \\
                    0 + 0 + \lambda_3 &= 0
                \end{aligned}
            \right.
        \]
        \[
            \alpha = 1
            \left\{
                \begin{aligned}
                    0 &= 0 \\
                    \lambda_1 &= -\lambda_3  \\
                    \lambda_1 &= -\lambda_3
                \end{aligned}
            \right.
            \hspace{1cm}
            \alpha = -1
            \left\{
                \begin{aligned}
                    0 &= 0 \\
                    \lambda_1 &= \lambda_3 \\
                    -\lambda_1 &= -\lambda_3
                \end{aligned}
            \right.
            \hspace{1cm}
            \alpha = 0
            \left\{
                \begin{aligned}
                    0 &= 0 \\
                    \lambda_1 &= 0 \\
                    \lambda_3 &= 0
                \end{aligned}
            \right.
        \]
        Podemos ver que para el caso donde \(\alpha\) es igual a 0, tenemos de forma directa que \(\lambda_1\) y \(\lambda_3\) son iguales a 0.
        Sin embargo para los casos donde \(\alpha\) es igual a 1 o -1, podemos retomar los nuevos valores de \(\lambda_1\) y volver a mirar los casos haciendo sustitución hacia atrás
        \[
            \begin{aligned}
                \alpha &= 1 \\
                \lambda_1 &= -\lambda_3 \\
            \end{aligned}
            \left\{
                \begin{aligned}
                    0 &= 0 \\
                    (-\lambda_3) + \lambda_3 &= 0 \\
                    (-\lambda_3) + \lambda_3 &= 0
                \end{aligned}
            \right.
            \hspace{1cm}
            \begin{aligned}
                \alpha &= -1 \\
                \lambda_1 &= \lambda_3 \\
            \end{aligned}
            \left\{
                \begin{aligned}
                    0 &= 0 \\
                    (\lambda_3) - \lambda_3 &= 0 \\
                    -(\lambda_3) + \lambda_3 &= 0
                \end{aligned}
            \right.
        \]
        Finalmente concluyendo que si \(\alpha\) es igual a 1, -1 o 0, se va a tener que \(\lambda_1\), \(\lambda_2\) y \(\lambda_3\) van a ser iguales a 0.
        Es decir, que el sistema homogéneo de ecuaciones lineales equivalente a la matriz del conjunto generador
        \(\left\{
            \begin{pmatrix}
                0 \\ 1 \\ \alpha
            \end{pmatrix},
            \begin{pmatrix}
                \alpha \\ 1 \\ 0
            \end{pmatrix},
            \begin{pmatrix}
                0 \\ \alpha \\ 1
            \end{pmatrix}
        \right\}\)
        va a tener única solución si \(\alpha\) va a ser igual a 1, -1 o 0. \\
        Y por ende, el conjunto generador, 
        \(\left\{
            \begin{pmatrix}
                0 \\ 1 \\ \alpha
            \end{pmatrix},
            \begin{pmatrix}
                \alpha \\ 1 \\ 0
            \end{pmatrix},
            \begin{pmatrix}
                0 \\ \alpha \\ 1
            \end{pmatrix}
        \right\}\), 
        es \(l.i\). Sin embargo, dado a la gran extensión del desarrollo de este punto, se va a afirmar que apartir de 
        este conjunto generador que es \(l.i\) y expresar cualquier vector en \(\mathbb{R}^3\) como una combinación lineal de este conjunto generador.
        Para entender mejor este concepto he leido un poco sobre temas que se veran más adelante, principalmente sobre las \emph{Bases en \(\mathbb{R}^3\)}.
    \setcounter{enumi}{22}
    \item Dados los vectores en \(\mathbb{R}^n\), \(v, v, w\) Si \(H = \left\{u, v, w\right\}\) es \(l.i\), determine cuál de la siguientes afirmaciones es falsa.
        \begin{enumerate}[label=\listAlph]
            \item \(Gen\left\{u + v, v − w\right\} \subseteq Gen\left(H\right)\) \\
                Verdadera, ya que,
                apartir de cualquier vector \(a\) en \(Gen\left\{u + v, v - w\right\}\) vamos a tener que también van a estar en \(Gen\left(H\right)\)
                por la combinación lineal de ambos conjuntos generados.
                Teniendo que, sean \(b, d\) vectores en \(\mathbb{R}^n\) se tiene que
                \[
                    \begin{gathered}
                        b \in Gen\left\{u + v, v - w\right\} \leftrightarrow b = \lambda_1 \cdot \left(u + v\right) + \lambda_2 \cdot \left(v - w\right) \\
                        d \in Gen\left(H\right) \leftrightarrow d = \gamma_1 \cdot u + \gamma_2 \cdot v + \gamma_3 w
                    \end{gathered}
                \]
                Se puede ver que
                \[
                    \begin{aligned}
                        a \in Gen\left\{u + v, v - w\right\} \leftrightarrow
                        a &= \lambda_1 \cdot \left(u + v\right) + \lambda_2 \cdot \left(v - w\right) \\
                        &= \lambda_1u + \lambda_1v + \lambda_2v - \lambda_2w \\
                        &= \lambda_1u + \left(\lambda_1 + \lambda_2\right)v - \lambda_2w \\
                        &= \gamma_1u + \gamma_2v + \gamma_3w = a \leftrightarrow
                        a \in Gen\left(H\right)
                    \end{aligned}
                \]
                Resultando que todo vector en \(Gen\{u + v, v - w\}\) también va a estar en \(Gen(H)\)
            \item \(H \subseteq Gen\left(H\right)\) \\
                Verdadera, ya que para cada vector en \(H\), definido como \(a\), va a tener un vector de escalares, definido como \(a_{\lambda}\), tal que 
                al hacer una combinación lineal de \(H\) por los componentes del vector de escalares se va a obtener a \(a\).
                \[
                    \begin{aligned}
                        u_{\lambda} &= \begin{pmatrix} 1 \\ 0 \\ 0 \end{pmatrix}; \hspace{1cm}
                        1 \cdot u + 0 \cdot v + 0 \cdot w = u \\
                        v_{\lambda} &= \begin{pmatrix} 0 \\ 1 \\ 0 \end{pmatrix}; \hspace{1cm}
                        0 \cdot u + 1 \cdot v + 0 \cdot w = v \\
                        w_{\lambda} &= \begin{pmatrix} 0 \\ 0 \\ 1 \end{pmatrix}; \hspace{1cm}
                        0 \cdot u + 0 \cdot v + 1 \cdot w = w \\
                    \end{aligned}
                \]
                Inclusive, como \(H\) y el conjunto generador son el mismo, para obtener los vectores del conjunto generador, definiendo el número de vectores en \(H\) como \(s_H\),
                vamos a tener que los vectores cuyos componentes generan a un vector del conjunto generador van a ser los vectores canónicos en \(\mathbb{R}^{s_H}\).
            \item \(\vec{0} \in Gen\left(H\right)\) \\
                Verdadera, solo es necesario ver que si hacemos una combinación lineal con las componentes del vector \(\vec{0}\) en \(\mathbb{R}^3\) vamos a tener
                \[
                    0 \cdot u + 0 \cdot v + 0 \cdot w = \vec{0}
                \]
                Por ende, \(\vec{0} \in Gen\left(H\right)\).
            \item \(\left\{u, u + v, u + v + w\right\}\) es \(l.d\) \\
                Antes de iniciar, definamos a \(C\) como el conjunto de los vectores \(u, u + v, u + v + w\) para una mejor explicación.
                Para que \(C\) sea \(l.d\) sabemos que, por definición, debe existir un escalar diferente a 0 tal que la combinación lineal de 
                estos escalares con los vectores en \(C\) sean igual a el vector \(\vec{0}\). Es decir, vamos a tener
                \[
                    \begin{gathered}
                        \lambda_1\left(u\right) + \lambda_2\left(u + v\right) + \lambda_3\left(u + v + w\right) = 0 \\
                        \lambda_1u + \lambda_2u + \lambda_2v + \lambda_3u + \lambda_3v + \lambda_3w = 0 \\
                        \left(\lambda_1 + \lambda_2 + \lambda_3\right)u + \left(\lambda_2 + \lambda_3\right)v + \lambda_3w = 0
                    \end{gathered}
                \]
                Ahora, como \(H\) es \(l.i\), por hipotesis sabemos que los únicos escalares, \(\gamma_1, \gamma_2, \gamma_3\), tal que 
                su combinación lineal con \(H\) es igual a el vector \(\vec{0}\) van a ser \(\gamma_1 = \gamma_2 = \gamma_3 = 0\).
                Entonces si alguno de los escalares \(\left(\lambda_1 + \lambda_2 + \lambda_3\right)\), \(\left(\lambda_2 + \lambda_3\right)\) 
                o \(\lambda_3\) es diferente a 0, vamos a tener una contradicción a la hipotesis. Por ende, 
                \(\left(\lambda_1 + \lambda_2 + \lambda_3\right)\), \(\left(\lambda_2 + \lambda_3\right)\) y \(\lambda_3\) deben ser iguales a 0. \\
                En conclusión, \(C\) es \(l.i\), entonces la afirmación es \textbf{falsa}.
            \item \(w \in H\) \\
                Verdadera, por definición de \(H\) tiene como elemento a \(w\).
        \end{enumerate}
    \item Considiere los vectores en \(\mathbb{R}^3\) dados por 
        \(
            u = 
            \begin{pmatrix}
                2 \\ -1 \\ 1
            \end{pmatrix},
            v =
            \begin{pmatrix}
                0 \\ 1 \\ 1
            \end{pmatrix}
            \text{ y }
            w = 
            \begin{pmatrix}
                2 \\ 1 \\ 3
            \end{pmatrix}
        \). \\
        Muestre que \(Gen\left\{u + v, v - w\right\} \subseteq Gen\left\{u, v, w\right\}\), y determine si estos conjuntos son iguales. \\
        Primero definamos ambos conjuntos generados como las combinaciones lineales de unos escalares con los vectores del respectivo conjunto generador, 
        sean \(b, d\) vectores en \(\mathbb{R}^3\), escalares \(\lambda_i, \gamma_i\) para \(i \in \mathbb{N}\) y \(1 \leq i \leq 3\) entonces
        \[
            \begin{aligned}
                b \in Gen\left\{u + v, v - w\right\} &\leftrightarrow b = \lambda_1\left(u + v\right) + \lambda_2\left(v - w\right) \\
                d \in Gen\left\{u, v, w\right\} &\leftrightarrow d = \gamma_1 \cdot u + \gamma_2 \cdot v + \gamma_3 \cdot w 
            \end{aligned}
        \]
        Además, antes de iniciar, verifiquemos que \(\left\{u + v, v - w\right\}\) y \(\left\{u, v, w\right\}\) son \(l.i\)
        \[
            Gen\left\{u + v, v - w\right\} = 
            Gen\left\{
                \begin{pmatrix}
                    2 \\ -1 \\ 1
                \end{pmatrix}
                +
                \begin{pmatrix}
                    0 \\ 1 \\ 1
                \end{pmatrix},
                \begin{pmatrix}
                    0 \\ 1 \\ 1
                \end{pmatrix}
                -
                \begin{pmatrix}
                    2 \\ 1 \\ 3
                \end{pmatrix}
            \right\}
            =   
            Gen\left\{
                \begin{pmatrix}
                    2 \\ 0 \\ 2
                \end{pmatrix},
                \begin{pmatrix}
                    -2 \\ 0 \\ -2
                \end{pmatrix}
            \right\}
            =
            Gen\left\{
                \begin{pmatrix}
                    2 \\ 0 \\ 2
                \end{pmatrix}
            \right\}
        \]
        Al \(Gen\left\{\begin{pmatrix}2 \\ 0 \\ 2\end{pmatrix}\right\}\) tener un conjunto generador de solo un vector, esté va a ser \(l.i\).
        A continuación, para ver si \(\left\{u, v, w\right\}\) es \(l.i\) vamos a tener mirar si su sistema homogéneo de ecuaciones lineales tiene solo una solución.
        Partiendo desde una matriz que tenga como columnas a \(u, v, \text{ y  } w\), vamos a tener que
        \[
            \begin{pmatrix}
                2 & 0 & 2 \\
                -1 & 1 & 1 \\ 
                1 & 1 & 3
            \end{pmatrix}
            \cdot
            \begin{pmatrix}x_1 \\ x_2 \\ x_3\end{pmatrix}
            = 
            \vec{0}
            ;\hspace{0.5cm}
            \begin{pmatrix}
                2 & 0 & 2 \\
                -1 & 1 & 1 \\ 
                1 & 1 & 3
            \end{pmatrix}
            \sim
            \begin{aligned}
                \frac{1}{2}F_1 &\mapsto F_1 \\
                F_3 + F_2 &\mapsto F_3
            \end{aligned}
            \begin{pmatrix}
                1 & 0 & 1 \\
                -1 & 1 & 1 \\ 
                0 & 2 & 4
            \end{pmatrix}
            \sim
            \begin{aligned}
                \frac{1}{2}F_3 &\mapsto F_3 \\
                F_2 + F_1 &\mapsto F_2 \\
                F_3 - F_2 &\mapsto F_3
            \end{aligned}
            \begin{pmatrix}
                1 & 0 & 1 \\
                0 & 1 & 2 \\ 
                0 & 0 & 0
            \end{pmatrix}
        \]
        Ahora con esta matriz equivalente vamos a hacer sustitución hacia atrás
        \[
            \left\{
                \begin{aligned}
                    x_1 + x_3 &= 0 \\
                    x_2 + x_3 &= 0 
                \end{aligned}
            \right.
            \hspace{0.5cm}
            \left\{
                \begin{aligned}
                    x_1 &= -x_3 \\
                    x_2 &= -x_3
                \end{aligned}
            \right.
            \hspace{0.5cm}
            \left\{
                \begin{aligned}
                    x_1 &= x_2 \\
                \end{aligned}
            \right.
        \]
        Aunque pareciera que \(x_3\) puede ser cualquier número real, si verificamos en el sistema homogéneo de ecuaciones lineales inicial vamos a 
        obtener que si \(x_1\) es igual a \(x_2\), y ambos son iguales a \(-x_3\)
        \[
            \left\{
            \begin{aligned}
                2x_1 + x_3 = 2\left(-x_3\right) + x_3 = -x_3 &= 0 \\
                -x_1 + x_2 + x_3 = -x_1 + x_1 + x_3 = x_3 &= 0 \\
                x_1 + x_2 + 3x_3 = -x_3 + \left(-x_3\right) + 3x_3 = x_3 &= 0
            \end{aligned}
            \right.
        \]
        El único valor real de \(x_3\) que puede solucionar el sistema homogéneo de ecuaciones lineales es cuando \(v_3\) es igual a 0.
        Por ende, el sistema homogéneo de ecuaciones lineales equivalente a el conjunto de vectores \(u, v, w\) solo tiene una solución, y además,
        \(\{u, v, w\}\) es \(l.i\).

                
        Como sabemos que ambos conjuntos generadores son \(l.i\) vamos a poder hacer un proceso similar al del punto 23--a.
        Partiendo de este punto, vamos a ver cada caso de contenencia para ver si ambos conjuntos generados son iguales.
        \ResetCases{}
        \begin{mathcase}{\(Gen\left\{u + v, v - w\right\} \subseteq Gen\left\{u, v, w\right\}\)}
            Sea \(a\) un vector elemento del conjunto generado \(Gen\left\{u + v, v - w\right\}\) cualesquiera, por definición es igual a
            \[
                a = \lambda_1\left(u + v\right) + \lambda_2\left(v - w\right)
            \]
            Distribuimos amobs escalares respecto a la suma de vectores.
            \[
                a = \lambda_1u + \lambda_1v + \lambda_2v - \lambda_2w
            \]
            Por propiedad asociativa de la suma entre vectores, \emph{juntamos} el vector \(v\)
            \[
                a = \lambda_1u + \left(\lambda_1v + \lambda_2v\right) - \lambda_2w
            \]
            Por dristribuidad de los escalares entre suma de escalares, factorizamos a \(v\) dentro de la suma de los vectores \(v\)
            \[
                a = \lambda_1u + \left(\lambda_1 + \lambda_2\right)v - \lambda_2w
            \]
            Reemplazamos \(\lambda_1 = \gamma_1\), \(\left(\lambda_1 + \lambda_2\right) = \gamma_2\) y \(-\lambda_2 = \gamma_3\)
            \[
                a = \gamma_1u + \gamma_2v + \gamma_3w
            \]
            Obteniendo, por definición que \(a\) es un elemento del conjunto generado \(Gen\left(u, v, w\right)\).
            Por ende, \(Gen\left\{u + v, v - w\right\}\) va a estar contenido en \(Gen\left(u, v, w\right)\).
        \end{mathcase}
        \begin{mathcase}{\(Gen\left\{u + v, v - w\right\} \supseteq Gen\left\{u, v, w\right\}\)}
            Partiendo desde, \(c \in Gen\left\{u, v, w\right\}\)
            \[
                \begin{aligned}
                    c &= \gamma_1u + \gamma_2v + \gamma_3w \\
                    &= \lambda_1u + \left(\lambda_1 + \lambda_2\right)v - \lambda_2w \\
                    &= \lambda_1u + \left(\lambda_1v + \lambda_2v\right) - \lambda_2w \\
                    &= \lambda_1u + \lambda_1v + \lambda_2v - \lambda_2w \\ 
                    &= \lambda_1\left(u + v\right) + \lambda_2\left(v - w\right) \\
                \end{aligned}
            \]
            Se tiene que, por definición, \(c \in Gen\{u + v, v - w\}\). 
            Por ende, \(Gen\{u + v, v - w\} \supseteq Gen\{u, v, w\}\).
        \end{mathcase}
        \\
        Conluyendo por ambos pasos de contenencia, por definición de igualdad de dos conjuntos, se obtiene que
        \[
            Gen\left\{u + v, v - w\right\} = Gen\left\{u, v, w\right\}
        \]

    \setcounter{enumi}{27}
    \item Sean \(A\) una matriz de \(m\) filas y \(n\) columnas y \(U\) una matriz escalonada equivalente a \(A\). Si \textbf{para cualquier vector \(b\)}
        de \(\mathbb{R}^m\), el sistema de ecuaciones lineales, cuya matriz aumentada es \(\left[A \mid b\right]\), \textbf{tiene solución única}, 
        determine cuales de las siguientes afirmaciones son verdaderas. Justifique su respuesta.
        \begin{enumerate}[label=\listAlph]
            %\item El vector \(b\) es combinación lineal de las columnas de \(A\).
                %Verdadera,
            %\item El vector \(b\) es combinación lineal de las columnas de \(U\).
                %Verdadera
            \setcounter{enumii}{2}
            \item Cada fila de \(U\) tiene un pivote. \\
                Falsa, ya que puede que exista un sistema de ecuaciones lineales tal que, en su matriz escalonada, cada variable, y por definición columna, tenga un pivote; 
                pero qué cuente con una ecuación donde cada coeficiente es igual a 0. En este caso, cada fila de la matriz escalonada va a contar con pivote excepto esta última fila.
            \item Cada columna de \(U\) tiene un pivote. \\
                Verdadera, ya que por hipotesis tenemos que el sistema de ecuaciones lineales tiene una única solución, o de manera similar, 
                que la matriz escalonada del sistema de ecuaciones lineales va tener un pivote para cada variable, dado a que no hay variables libres.
                Y por definición de la matriz de un sistema de ecuaciones lineales, cada columna corresponde a una variable, por ende cada columna tiene un pivote.
            %\item La matriz \(U\) tiene \(n\) pivotes.
                %Verdadera
            %\item La matriz \(U\) tiene \(m\) pivotes.
                %%
            %\item \(m = n\).
                %%
            %\item Las columnas de \(A\) generan a \(\mathbb{R}^m\)
                %%
        \end{enumerate}
    \item Sean \(A\) una matriz de \(m\) filas y \(n\) columnas y \(U\) una matriz escalonada equivalente a \(A\). 
        Si \textbf{para cualquier vector \(b\) de} \(\mathbb{R}^m\),
        el sistema de ecuaciones lineales, cuya matriz aumentada es \(\left[A \mid b\right]\), 
        \textbf{tiene infinitas soluciones},
        determine cuales de las siguientes afirmaciones son verdaderas. Justifique su respuesta.
        \begin{enumerate}[label=\listAlph]
            %\item El vector \(b\) es combinación lineal de las columnas de \(A\).
                %Falso, existen vectores tal que no sean una combinación lineal de las columnas de \(A\), o equivalente,
                %que las componentes del vector no sean una solución para el sistema de ecuaciones lineales. 
                %Por ejemplo, si el conjunto solución de un sistema de ecuaciones lineales es 
                %\[
                    %\left\{
                        %x_1 \in \mathbb{R} 
                        %\mid 
                        %\left(x_1, 2x_1, 4x_1, 8x_1, \ldots, 2^{m}x_1\right)
                    %\right\}
                %\]
            %\item El vector \(b\) es combinación lineal de las columnas de \(U\).
            %\item Cada fila de \(U\) tiene un pivote.
            \setcounter{enumii}{3}
            \item Cada columna de \(U\) tiene un pivote. \\
                Falso, ya que puede existir un sistema de ecuaciones lianes tal que su matriz cuyas columnas sean \(c_1, c_2, \ldots, \vec{0}, \ldots, c_n\),
                donde \(c_i\) y \(\vec{0}\) son vectores en \(\mathbb{R}^m\) para \(i \in \mathbb{N} \mid 1 \leq i \leq n\),
                donde la columna del vector \(\vec{0}\) no va a tener pivote. O, en general, cuando una variable sea libre, esta no va a 
                tener pivote, y para que un sistema de ecuaciones lineales tenga infinitas soluciones vamos a tener que esta va a tener almenos una variable libre.
            %\item La matriz \(U\) tiene \(n\) pivotes.
            %\item La matriz \(U\) tiene \(m\) pivotes.
            \setcounter{enumii}{6}
            \item \(m < n\). \\
                Falso, ya que apartir de un sistema de ecuaciones lineales, denotado \(Ec\), tal que la diferencia entre columnas y filas de su matriz sea de una columna mayor a las filas,
                es decir \(m = n + 1\). Y además, en \(Ec\) no se cuenta con la ecuación de la solución trivial: \(0x_1 + 0x_2 + \cdots + 0x_n = 0\). 
                Se va a poder crear un nuevo sistema de ecuaciones linales, denotado \(Ec^+\), tal que tenga todas las ecuaciones de \(Ec\) 
                y se agregue la ecuación solución trivial. Entonces, la matriz de \(Ec^+\) va a tener el mismo número de columnas y filas.
            %\item Las columnas de \(A\) generan a \(\mathbb{R}^m\).
        \end{enumerate}
    \item Sean \(A\) una matriz de \(m\) filas y \(n\) columnas y \(U\) una matriz escalonada equivalente a \(A\). 
        Si \textbf{para un vector \(b\)} de \(\mathbb{R}^m\),
        el sistema de ecuaciones lineales, cuya matriz aumentada es \(\left[A \mid b\right]\),
        \textbf{es inconsistente}, 
        determine cuales de las siguientes afirmaciones son verdaderas y responda a las preguntas formuladas. 
        Justifique su respuesta.
        \begin{enumerate}[label=\listAlph]
            \item El vector \(b\) es combinación lineal de las columnas de \(A\). \\
                Falso, por definición de sistema de ecuaciones lineales inconsistente. Si \(b\) es una combinación lineal de las columnas de \(A\) va a existir un vector de escalares
                \(x = \begin{pmatrix}x_1 \\ x_2 \\ \vdots \\ x_n\end{pmatrix}\) tal que, siendo \(A_i\) una columna de \(A\) para \(i \in \mathbb{N} \mid 1 \leq i \leq n\) 
                    que es un vector en \(\mathbb{R}^m\) se va a tener que
                \[
                    A \cdot x = b
                \]
                Sin embargo, al ser el sistema de ecuaciones lineales inconsistente para \(b\), se va a tener que alguna componente de \(x\), para 
                que se cumpla la igualdad, va a tener que ser dos valores al mismo tiempo, o se va a tener una contradicción en su valor. 
                Entonces esta combinación lineal no puede existir.
            %\item El vector \(b\) puede ser combinación lineal de las columnas de \(U\).
            \setcounter{enumii}{2}
            \item Cada fila de \(U\) tiene un pivote. \\
                Falso, esto necesariamente no se cumple para todo sistema de ecuaciones lineales. Ya que en una matriz expandida escalonada tal que su ultima fila sea
                \[
                    \left(
                    \begin{array}{cccc|c}
                        \vdots & \vdots & \ddots & \vdots & \vdots \\
                        0 & 0 & \cdots & 0 & p
                    \end{array}
                    \right)
                    \hspace{1cm}
                    p \neq 0
                \]
                En esta fila la matriz escalonada no va a contar con pivote.
            %\item El vector \(b\) puede ser \(\vec{0}\)?
            %\item El vector \(b\) puede ser un múltiplo de alguna de las columnas de \(A\)?
            %\item El vector \(b\) puede ser la suma de las columnas de \(A\)?
            %\item Las columnas de \(A\) generan a \(\mathbb{R}^m\).
            %\item ¿Qué puede decirse del número de pivotes de \(U\)? \\
        \end{enumerate}
    \item ¿%
        \(
            Gen
            \left\{
                \begin{pmatrix}
                    1 \\ 0 \\ -3
                \end{pmatrix},
                \begin{pmatrix}
                    4 \\ -10 \\ 2
                \end{pmatrix}
            \right\}
            =
            Gen
            \left\{
                \begin{pmatrix}
                    0 \\ -5 \\ 8
                \end{pmatrix},
                \begin{pmatrix}
                    -1 \\ 5 \\ -5
                \end{pmatrix}
            \right\}
        \)? \\
        Para demostrar esto simplemente tenemos que ver si cada vector en el conjunto generador 
        \(
            \left\{
                \begin{pmatrix}
                    0 \\ -5 \\ 8
                \end{pmatrix},
                \begin{pmatrix}
                    -1 \\ 5 \\ -5
                \end{pmatrix}
            \right\}
        \)
        esta en el conjunto generado 
        \[
            Gen
            \left\{
                \begin{pmatrix}
                    1 \\ 0 \\ -3
                \end{pmatrix},
                \begin{pmatrix}
                    4 \\ -10 \\ 2
                \end{pmatrix}
            \right\}
        \]
        lo cual es posible mediante un par de sistemas de ecuaciones lineales. Veamos si ambos vectores del segundo conjunto generador
        estan en el primer conjunto generado, creando un sistema de ecuaciones lineales tal que cada 
        columna sea un vector en el primer conjunto generador, las variables sean los escalares necesarios para 
        que se obtenga los vectores a buscar.
        \[
            \left\{
            \begin{aligned}
                \lambda_1 + 4\lambda_2 &= 1 \\
                10\lambda_2 &= 5 \\
                -3\lambda_1 + 2\lambda_2 &= -5 \\
            \end{aligned}
            \right.
            \hspace{0.5cm}
            \left\{
            \begin{aligned}
                \lambda_1 + 4\lambda_2 &= 0 \\
                10\lambda_2 &= -5 \\
                -3\lambda_1 + 2\lambda_2 &= 8 \\
            \end{aligned}
            \right.
        \]
        \[
            \left(\begin{array}{cc|cc}
                1 & 4 & 1 & 0\\
                0 & 10 & 5 & -5 \\
                -3 & 2 & -5 & 8
            \end{array}\right)
            \sim
            \begin{aligned}
                \frac{1}{5}F_2 &\mapsto F_2 \\
            \end{aligned}
            \left(\begin{array}{cc|cc}
                1 & 4 & 1 & 0\\
                0 & 2 & 1 & -1 \\
                -3 & 2 & -5 & 8
            \end{array}\right)
            \sim
            \begin{aligned}
                F_3 - F_2 &\mapsto F_3 \\
                \frac{1}{3}F_3 &\mapsto F_3 \\
            \end{aligned}
            \left(\begin{array}{cc|cc}
                1 & 4 & 1 & 0\\
                0 & 2 & 1 & -1 \\
                -1 & 0 & -2 & 3
            \end{array}\right)
        \]
        \[
            \sim
            \begin{aligned}
                F_1 + F_3 &\mapsto F_1 \\
            \end{aligned}
            \left(\begin{array}{cc|cc}
                0 & 4 & -1 & 3\\
                0 & 2 & 1 & -1 \\
                -1 & 0 & -2 & 3
            \end{array}\right)
            \sim
            \begin{aligned}
                F_1 - 2F_2 &\mapsto F_1 \\
            \end{aligned}
            \left(\begin{array}{cc|cc}
                0 & 0 & 1 & 1 \\
                0 & 2 & 1 & -1 \\
                -1 & 0 & -2 & 3
            \end{array}\right)
        \]
        Y si hacemos sustitución hacia atrás, podemos ver que
        \[
            \left\{
            \begin{aligned}
                0 &= 1 \\
                2\lambda_2 &= 1 \\
                -\lambda_1 &= -2 \\
            \end{aligned}
            \right.
            \hspace{0.5cm}
            \left\{
            \begin{aligned}
                0 &= 1 \\
                2\lambda_2 &= -1 \\
                -\lambda_1 &= 3 \\
            \end{aligned}
            \right.
        \]
        Es decir, ningún de los vectores en el segundo conjunto generador no estan en el primer conjunto generado. 
        Por ende, 
        \(
            Gen
            \left\{
                \begin{pmatrix}
                    1 \\ 0 \\ -3
                \end{pmatrix},
                \begin{pmatrix}
                    4 \\ -10 \\ 2
                \end{pmatrix}
            \right\}
            \neq
            Gen
            \left\{
                \begin{pmatrix}
                    0 \\ -5 \\ 8
                \end{pmatrix},
                \begin{pmatrix}
                    -1 \\ 5 \\ -5
                \end{pmatrix}
            \right\}
        \)
        %Para demostrar esto, simplemente tenemos que demostrar doble contenencia entre ambos conjuntos generados.
        %\ResetCases{}
        %\begin{mathcase}{%
            %\(%
                %Gen\left\{%
                    %\begin{pmatrix}1 \\ 0 \\ -3\end{pmatrix},%
                    %\begin{pmatrix}4 \\ -10 \\ 2\end{pmatrix}%
                %\right\}%
                %\subseteq
                %Gen\left\{%
                    %\begin{pmatrix}0 \\ -5 \\ 8\end{pmatrix},%
                    %\begin{pmatrix}-1 \\ 5 \\ -5\end{pmatrix}
                %\right\}%
            %\)%
        %}
            %Sea \(a\) un vector en \(\mathbb{R}^3\) tal que este en \(Gen\left\{\begin{pmatrix}1 \\ 0 \\ -3\end{pmatrix},\begin{pmatrix}4 \\ -10 \\ 2\end{pmatrix}\right\}\), entonces
            %\[
                %a = \lambda_1 \begin{pmatrix}1 \\ 0 \\ -3\end{pmatrix} + \lambda_2 \begin{pmatrix}4 \\ -10 \\ 2\end{pmatrix}
            %\]
        %\end{mathcase}
        %\begin{mathcase}{%
                %\(%
                    %Gen\left\{%
                        %\begin{pmatrix}1 \\ 0 \\ -3\end{pmatrix},%
                        %\begin{pmatrix}4 \\ -10 \\ 2\end{pmatrix}%
                    %\right\}%
                    %\supseteq
                    %Gen\left\{%
                        %\begin{pmatrix}0 \\ -5 \\ 8\end{pmatrix},%
                        %\begin{pmatrix}-1 \\ 5 \\ -5\end{pmatrix}
                    %\right\}%
                %\)%
            %}
        %\end{mathcase}
\end{enumerate}
\end{document}
