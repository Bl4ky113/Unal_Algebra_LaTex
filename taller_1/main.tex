\documentclass{article}

\def\MyAuthor{Martín Steven Hernández Ortiz}
\def\MyEmail{mahernandezor@unal.edu.co}
\def\MyClass{Álgebra Lineal}
\def\MyTitle{Taller 1}
\def\MyDate{\today}

\author{\MyAuthor{}}
\title{\MyClass{} \-- \MyTitle{}}

\usepackage{csquotes}


%%% Template Packages %%%

\usepackage{graphicx} % Images
\usepackage{tcolorbox} % Color Box
\usepackage[%
vmargin=2.25cm,%
hmargin=2.25cm%
]{geometry} % Page Geometry
\usepackage{fancyhdr} % Header / Footer Styles

\usepackage{ragged2e} % Text Align
\usepackage{amsmath} % Math Align

\usepackage{booktabs} % Tables Package

\usepackage{mathtools} % Math General
\usepackage{amsthm} % Math Envs
\usepackage{unicode-math} % Math Symbols

\usepackage{polyglossia} % Language
    \setdefaultlanguage{spanish}

\renewcommand{\thefootnote}{\Roman{footnote}} % Changing footnotes arabic to roman numbers

%%% Template Styles %%%

% Header / Footer Styles
\pagestyle{fancy}
\RenewDocumentCommand{\headrule}{}{%
    \rule[0.1cm]{\textwidth}{0.1mm}%
}
\RenewDocumentCommand{\footrule}{}{%
    \rule[0.1cm]{\textwidth}{0.1mm}%
}

\fancyhf[HC]{{\slshape \MyTitle{}}}

% Redefine \maketitle
\RenewDocumentCommand{\maketitle}{s}{%
    \begin{@twocolumntrue}%
        \begin{minipage}{0.3\textwidth}%
            \includegraphics[width=0.85\textwidth]{../template/src/unal_logo.pdf}%
        \end{minipage}%
        \begin{minipage}{0.7\textwidth}{{%
        \begin{Center}%
            {\large \itshape \MyClass{}} \\[1ex]
            {\huge  \slshape \MyTitle{}} \\[4ex]
            {\Large  \MyAuthor{}} \\[0ex]
            {\small  \MyEmail{}} \\[4ex] 
            \MyDate{}
        \end{Center}%
        }}%
        \end{minipage}%
    \end{@twocolumntrue}%
    \vspace{0.5cm}%
    \begin{Center}%
        \rule[0cm]{\textwidth}{0.1mm}%
    \end{Center}%
    \vspace{0.5cm}%
}

%%% Template Math stuff %%%

% Theorems
\newtheorem{TMPMathTheorem}{Teorema}
\NewDocumentEnvironment{theorem}{+b} {%
    \begin{TMPMathTheorem}%
        #1 %
    \end{TMPMathTheorem}%
} {}

% Collorary
\newtheorem{TMPMathCorollary}[TMPMathTheorem]{Corolario}
\NewDocumentEnvironment{corollary}{+b} {%
    \begin{TMPMathCorollary}%
        #1 %
    \end{TMPMathCorollary}%
} {}

% Definitions
\newtheorem{TMPMathDefinition}[TMPMathTheorem]{Definición}
\NewDocumentEnvironment{definition}{+b} {%
    \begin{tcolorbox}[left=0mm,right=0mm]%
        \begin{TMPMathDefinition}%
            #1 %
            \begin{FlushRight}%
                \(\bigtriangleup{}\)%
            \end{FlushRight}%
        \end{TMPMathDefinition}%
    \end{tcolorbox}%
} {}



\begin{document}
\maketitle

\begin{enumerate}
    \setcounter{enumi}{4}
    \extramarks{Punto 5}{}
    \item Dar condiciones sobre el parámetro \(\alpha\), para que el sistema:
        \[
            \left\{
                \begin{aligned}
                    x + y - z = 2 \\
                    x + 2y + z = 3 \\
                    x + y + \left(\alpha^2 - 5\right)z = \alpha
                \end{aligned}
            \right.
        \]
        Para un mejor desarrollo del punto, primero se va a convertir el sistema de ecuaciones a su matriz aumentada, 
        para después realizar parcialmente la eliminación de \emph{Gauss}, ya que no sabemos si el valor de \(\alpha\) nos favorezca para que el pivote sea \(1\).
        \[
            \text{Base}
            \left(
            \begin{array}{ccc|c}
                1 & 1 & -1 & 2 \\
                1 & 2 & 1 & 3 \\
                1 & 1 & \alpha^2 -5 & \alpha \\
            \end{array}
            \right)
            \sim
            F_2 - F_1 \mapsto F_2
            \left(
            \begin{array}{ccc|c}
                1 & 1 & -1 & 2 \\
                0 & 1 & 2 & 1 \\
                1 & 1 & \alpha^2 - 5 & \alpha \\
            \end{array}
            \right)
        \]
        \[
            \sim
            F_3 - F_1 \mapsto F_3
            \left(
            \begin{array}{ccc|c}
                1 & 1 & -1 & 2 \\
                0 & 1 & 2 & 1 \\
                0 & 0 & \alpha^2 - 4 & \alpha - 2 \\
            \end{array}
            \right)
        \]
        Desde este punto podemos deducir, mediante la sustitución hacia atrás, que:
        \[
            \left\{
                \begin{aligned}
                    x + y - z = 2 \\
                    y + 2z = 1 \\
                    \left(\alpha^2 - 4\right)z = \alpha - 2
                \end{aligned}
            \right.
            \hspace{1cm}
            \begin{aligned}
                x &= 1 + \frac{3\alpha - 6}{\alpha^2 - 4} \text{,} \\
                y &= 1 - \frac{2\alpha - 4}{\alpha^2 - 4} \hspace{1ex} \text{y}  \\
                z &= \frac{\alpha - 2}{\alpha^2 - 4}
            \end{aligned}
        \]
        \begin{enumerate}[label=\listAlph]
            \item No tenga solución. \\
                Tomando a \(\alpha = -2\), podemos ver que en la \(3^{\text{ra}}\) ecuación del sistema se tiene: \(0x + 0y + 0z = -4\). 
                Haciendo el sistema de ecuaciones inconsistente y por definición, no se tiene ninguna solución o que el conjunto solución es \(\emptyset\).
            \item Tenga infinitas soluciones (dar la solución general). \\
                Se debe tomar que \(\alpha = 2\), para que en la \(3^{\text{ra}}\) ecuación del sistema se tenga: \(0x + 0y + 0z = 0\).
                Lo cual se puede interpretar como  que \(x \text{, } y \text{ y } z\) tienen infinitas soluciones, 
                sin embargo \(x \text{ y } y\) se pueden explicar en terminos de \(z\).
                \[
                    \begin{aligned}
                        y &= 1 - 2z \\
                        x &= 2 + z - y \\
                        &= 1 + 3z \\
                    \end{aligned}
                \]
                Obteniendo el conjunto solución
                \[
                    \left\{\left(1 + 3z, 1 - 2z, z\right) \vert\hspace{0.5ex} z \in \mathbb{R} \right\}
                \]

            \item Tenga solución única (dar la solución). \\
                Si se toma \(\alpha = 0\), en la matriz del sistema de ecuaciones se podría modificar de tal forma que
                \[
                    \left(
                    \begin{array}{ccc|c}
                        1 & 1 & -1 & 2 \\
                        0 & 1 & 2 & 1 \\
                        0 & 0 & -4 & -2 \\
                    \end{array}
                    \right)
                    \sim
                    F_3 + F_1 \mapsto F_3
                    \left(
                    \begin{array}{ccc|c}
                        1 & 1 & -1 & 2 \\
                        0 & 1 & 2 & 1 \\
                        0 & 0 & -5 & 0 \\
                    \end{array}
                    \right)
                    \sim
                    -\frac{1}{5}F_3 \mapsto F_3
                    \left(
                    \begin{array}{ccc|c}
                        1 & 1 & -1 & 2 \\
                        0 & 1 & 2 & 1 \\
                        0 & 0 & 1 & 0 \\
                    \end{array}
                    \right)
                \]
                Realizando sustitución hacia atrás obtenemos
                \[
                    \left\{
                        \begin{aligned}
                            x + y - z = 2 \\
                            y + 2z = 1 \\
                            z = 0
                        \end{aligned}
                    \right.
                    \hspace{1cm}
                    \begin{aligned}
                        x &= 2 + z - y = 1 \\
                        y &= 1 -2z = 1 \\
                        z &= 0 \\
                    \end{aligned}
                \]
                \[
                    \text{Siendo el conjunto solución: }
                    \left\{\left(1, 1, 0\right)\right\}
                \]
                El cual podemos verificar en el sistema de ecuaciones:
                \[
                    \left\{
                        \begin{aligned}
                            1 + 1 - 0 = 2 &= 2 \\
                            1 + 2\left(0\right) = 1 &= 1 \\
                            0 &= 0
                        \end{aligned}
                    \right.
                \]
        \end{enumerate}

    \extramarks{Punto 6}{}
    \item Si al escalonar la matriz aumentada de un sistema de ecuaciones lineales, se obtiene
        \[
            \left(
            \begin{array}{ccccc|c}
                \sqrt{2}    & 0        & 3        & -1       & 4        & 0 \\
                0           & 0        & 2        & \pi      & -1       & 1 \\
                0           & 0        & 0        & \alpha   & 1        & -5 \\
                0           & 0        & 0        & 0        & b^2 - b  & b \\
            \end{array}
            \right)
        \]
        \begin{itemize}
            \item Es el sistema consistente cuando a = b = 0? En caso de serlo, es la solución única? \\
                Tomando a \(a = b = 0\), tendriamos la matriz:
                \[
                    \left(
                    \begin{array}{ccccc|c}
                        \sqrt{2}    & 0        & 3        & -1       & 4        & 0 \\
                        0           & 0        & 2        & \pi      & -1       & 1 \\
                        0           & 0        & 0        & 0        & 1        & -5 \\
                        0           & 0        & 0        & 0        & 0        & 0 \\
                    \end{array}
                    \right)
                \]
                Por la \(4^{\text{ta}}\) fila o ecuación podemos ver el sistema de ecuaciones que va a tener infinitas soluciones. 
                Y por ende, el sistema va a ser consistente.
            \item Es el sistema consistente cuando a = 1 y b = 0? En caso de serlo, es la solución única? \\
                Si \(a = 1 \text{y} b = 0\), tendríamos la matriz:
                \[
                    \left(
                    \begin{array}{ccccc|c}
                        \sqrt{2}    & 0        & 3        & -1       & 4        & 0 \\
                        0           & 0        & 2        & \pi      & -1       & 1 \\
                        0           & 0        & 0        & 1        & 1        & -5 \\
                        0           & 0        & 0        & 0        & 0        & 0 \\
                    \end{array}
                    \right)
                \]
                Obteniendo que el sistema es consistente y tiene infinitas soluciones. 
                Ya que la \(4^{\text{ta}}\) ecuación del sistema implica que las variables del sistema pueden tener infinitas soluciones.
            \item Es el sistema consistente cuando a = 0 y b = 1? En caso de serlo, es la solución única? \\
                Si \(a = 0 \text{y} b = 1\), tendríamos la matriz:
                \[
                    \left(
                    \begin{array}{ccccc|c}
                        \sqrt{2}    & 0        & 3        & -1       & 4        & 0 \\
                        0           & 0        & 2        & \pi      & -1       & 1 \\
                        0           & 0        & 0        & 0        & 1        & -5 \\
                        0           & 0        & 0        & 0        & 0        & 1 \\
                    \end{array}
                    \right)
                \]
                Se va ver que el sistema de ecuaciones va a ser inconsistente, ya que en la \(4^{\text{ta}}\) ecuación del sistema se va a 
                tener que los variables con coeficiente de 0, va a ser igual a un número diferente a 0.
            \item Si b = 2 y a \(\neq\) 0, qué puede decirse del conjunto solución? \\
                Si \(a \neq 0 \text{ y } b = 2\), tendriamos:
                \[
                    \left(
                    \begin{array}{ccccc|c}
                        \sqrt{2}    & 0        & 3        & -1       & 4        & 0 \\
                        0           & 0        & 2        & \pi      & -1       & 1 \\
                        0           & 0        & 0        & \alpha   & 1        & -5 \\
                        0           & 0        & 0        & 0        & 2        & 2 \\
                    \end{array}
                    \right)
                    \sim
                    \begin{aligned}
                        \frac{1}{2}F_4 &\mapsto F_4 \\
                        F_3 - F_4 &\mapsto F_3 \\
                        F_2 + F_4 &\mapsto F_2 \\
                        F_1 - 4F_4 &\mapsto F_1
                    \end{aligned}
                    \left(
                    \begin{array}{ccccc|c}
                        \sqrt{2}    & 0        & 3        & -1       & 0        & 4 \\
                        0           & 0        & 2        & \pi      & 0        & 2 \\
                        0           & 0        & 0        & \alpha   & 0        & -6 \\
                        0           & 0        & 0        & 0        & 1        & 1 \\
                    \end{array}
                    \right)
                \]
                Como \(\alpha \neq 0\) podemos usarlo como denominador en una fracción:
                \[
                    \sim
                    \begin{aligned}
                        \frac{1}{2}F_2 \mapsto F_2
                    \end{aligned}
                    \left(
                    \begin{array}{ccccc|c}
                        \sqrt{2}    & 0        & 3        & -1              & 0        & 4 \\
                        0           & 0        & 1        & \frac{\pi}{2}   & 0        & 1 \\
                        0           & 0        & 0        & \alpha          & 0        & -6 \\
                        0           & 0        & 0        & 0               & 1        & 1 \\
                    \end{array}
                    \right)
                    \sim
                    \begin{aligned}
                        \frac{1}{\alpha}F_3 &\mapsto F_3 \\
                        F_1 + F_3 &\mapsto F_1 
                    \end{aligned}
                    \left(
                    \begin{array}{ccccc|c}
                        \sqrt{2}    & 0        & 3        & 0               & 0        & 4 - \frac{6}{\alpha} \\
                        0           & 0        & 1        & \frac{\pi}{2}   & 0        & 1 \\
                        0           & 0        & 0        & 1               & 0        & -\frac{6}{\alpha} \\
                        0           & 0        & 0        & 0               & 1        & 1 \\
                    \end{array}
                    \right)
                \]
                Apartir de la matriz expandida podemos realizar sustitución hacia atrás y obtener el conjunto solución:
                \[
                    \left\{
                        \begin{aligned}
                            \sqrt{2}x_1 + 0x_2 + 3x_3 + 0x_4 + 0x_5 = 4 - \frac{6}{\alpha} \\
                            0x_1 + 0x_2 + x_3 + \frac{pi}{2}x_4 + 0x_5 = 1 \\
                            0x_1 + 0x_2 + 0x_3 + x_4 + 0x_5 = - \frac{6}{\alpha} \\
                            0x_1 + 0x_2 + 0x_3 + 0x_4 + x_5 = 1
                        \end{aligned}
                    \right.
                    \hspace{3ex}
                    \begin{aligned}
                        x_1 &= \frac{\alpha + 9\pi - 6}{\sqrt{2}\alpha} \\
                        x_2 &\in \mathbb{R} \\
                        x_3 &= \frac{\alpha - 3\pi}{\alpha} \\
                        x_4 &= -\frac{6}{\alpha} \\
                        x_5 &= 1
                    \end{aligned}
                \]
                \[
                    \left\{\left(\frac{\alpha + 9\pi - 6}{\sqrt{2}\alpha}, x_2, \frac{\alpha - 3\pi}{\alpha}, -\frac{6}{\alpha}, 1\right) \vert\hspace{0.5ex} x_2 \in \mathbb{R} \right\}
                \]
                Aunque pareciera que se podría obtener una única solución, por \(x_2\) el conjunto solución termina teniendo infinitos elementos.

            %\item Si b = 1 y a \(\neq\) 0, qué puede decirse del conjunto solución? \\
            %\item Si a \(\neq\) 0, dé un valor de b (diferente de 0), en caso de que exista, para que el sistema sea consistente. \\
            %\item Si a \(\neq\) 0, para que valores de b el sistema tiene infinitas soluciones? \\
            %\item Si a \(\neq\) 0, para que valores de b el sistema tiene solución única? \\
            %\item Si a \(\neq\) 0, para que valores de b el sistema es inconsistente? \\
        \end{itemize}

    \setcounter{enumi}{7}
    \extramarks{Punto 8}{}
    \item Justifique POR QUÉ cada una de las siguientes afirmaciones son VERDADERAS
        \begin{enumerate}[label=\listAlph]
			\item Si un sistema de ecuaciones lineales tiene solución única, su sistema de ecuaciones lineales homogéneo asociado también tiene solución única. \\
                Primero bozquejemos los sistemas preguntados como sistemas de ecuaciones \(m \times n\)
                \[
                    S_1 =
                    \left\{
                        \begin{aligned}
                            a_{11}x_{11} + a_{12}x_{12} + \cdots + a_{1n}x_{1n} &= b_1 \\
                            a_{21}x_{21} + a_{22}x_{22} + \cdots + a_{2n}x_{2n} &= b_2 \\
                            & \hspace{0.75ex}\vdots \\
                            a_{m1}x_{m1} + a_{m2}x_{m2} + \cdots + a_{mn}x_{mn} &= b_m \\
                        \end{aligned}
                    \right.
                    \hspace{0.5cm}
                    S_h = 
                    \left\{
                        \begin{aligned}
                            a_{11}x_{11} + a_{12}x_{12} + \cdots + a_{1n}x_{1n} &= 0 \\
                            a_{21}x_{21} + a_{22}x_{22} + \cdots + a_{2n}x_{2n} &= 0 \\
                            & \hspace{0.75ex}\vdots \\
                            a_{m1}x_{m1} + a_{m2}x_{m2} + \cdots + a_{mn}x_{mn} &= 0 \\
                        \end{aligned}
                    \right.
                \]
                Como sabemos que la implicación presentada: \(\text{`soluciónes de }S_1 = 1\text{`} \Rightarrow \text{`soluciones de }S_h = 1\text{`}\); es una tautolgía por la hipotesis del problema, 
                Además, cabe aclarar, que por definición de un sistema de ecuaciones homogéneo este siempre va a tener almenos una o infinitas soluciones. Siendo la primera, y trivial, solución donde 
                todas las variables simplemente sean 0: \(x_1 = x_2 = \cdots = x_{n - 1} = x_n = 0\). \\ 
                Tambien, analizando que el sistema de ecuaciones \(S_1\) solo tenga una solución implica que no tenga ningún elemento libre.
                Concluyendo que, con esta información de los sistemas, la única solución de \(S_h\) sería la trivial. 
                Ya que al no tener ningún elemento libre en ambos sistemas solo puede existir la solución trivial para \(S_h\).
                Dado que cada valor tiene que tomar un valor fijo. Y, los únicos valores que resuelven a \(S_h\) es la solución trival: \(x_1 = x_2 = \cdots = x_{n - 1} = x_n = 0\).
			\item Un sistema de ecuaciones lineales con 14 variables y 10 ecuaciones no tiene solución única. \\
                Esta afrimación es bastante directa y sencilla de explicar, ya que, se divide entre el caso donde el sistema es consistente y en el caso contrario.
                \ResetCases{}
                \begin{mathcase}{Sistema consistente}
                    En este caso al pasar el sistema de ecuaciones a su matriz expandida y escalonar esta matriz para realizar el método de \emph{Gauss}, 
                    se va a poder ver que se cuenta con 10 o menos pivotes para las variables del sistema. 
                    Consecuentemente, se va a contar con algunas variables libres o con infinitas soluciones, pero con las cuales podrían ser definidas en terminos de estas variables libres.
                \end{mathcase}
                \begin{mathcase}{Sistema inconsistente}
                    En el caso que el sistema sea inconsistente, ya sea porqué se tenga que en la matriz expandida del sistema una fila de coeficientes 0 para todas las variables y un valor 
                    diferente a 0 como termino independiente. O, el sistema cuente con incongruencias entre las ecuaciones. En general, por definición, no se va a tener ninguna solución para el sistema.
                \end{mathcase}
            \setcounter{enumii}{6}
			%\item Un sistema de ecuaciones lineales con 27 variables y 13 ecuaciones puede no tener solución. \\
			%\item Un sistema de ecuaciones lineales con 100 variables y 300 ecuaciones puede tener solución única. \\
			%\item Un sistema de ecuaciones lineales homogéneo con 10 variables y 7 ecuaciones tiene infinitas soluciones. \\
			%\item Un sistema de ecuaciones lineales homogéneo con 14 variables y 10 ecuaciones no tiene solución única. \\
			\item Un sistema de ecuaciones lineales homogéneo con 100 variables y 300 ecuaciones puede tener solución única. \\
                Esta afrimación es bastante sencilla, ya que por definición, un sistema de ecuaciones homogéneo va a tener cómo mínimo una 
                solución. Esta solución va a ser la solución trivial, el cual para el sistema propuesto sería: \(x_1 = x_2 = x_3 = \cdots = x_{98} = x_{99} = x_{100} = 0\).
                Además, como la pregunta solo pide explicar la existencia de esta respuesta trivial, no es necesario analizar más información sobre este sistema.
			\item Un sistema de ecuaciones lineales consistente con 10 variables y 7 ecuaciones tiene infinitas soluciones. \\
                Dado a que el sistema es consistente, al contar con un número mayor de variables que de variables al momento de escribir la matriz expandida 
                del sistema y escalonarla, se va a tener 10 o menos pivotes. 
                Cabe aclarar que, el número de pivotes puede ser menor a 10, a causa de que puede que existan filas en la matriz donde los coeficientes sean 0. 
                De igual forma, al no poder tener un pivote en cada variable en la forma escalonada de la matriz expandida, se va a tener que las variables sin 
                pivote van a ser variables libres o variables con infinitas soluciones.
            %\item Un sistema de ecuaciones lineales consistente con 14 variables y 10 ecuaciones no tiene solución única. \\
			%\item Un sistema de ecuaciones lineales consistente con 100 variables y 300 ecuaciones puede tener solución única. \\
        \end{enumerate}

    \extramarks{Punto 9}{}
    \item Justifique POR QUÉ cada una de las siguientes afirmaciones son FALSAS
        \begin{enumerate}[label=\listAlph]
			\item Si un sistema de ecuaciones lineales tiene solución, cualquier otro sistema de ecuaciones lineales con la misma matriz de coeficientes también tiene solución. \\
                Esta proposición es bastante simple de demostrar falsa mediante un contra--ejemplo.
                Consideremos \(S_1\) y \(S_2\), sistemas de ecuaciones lineales asociados. Definidos como:
                \[
                    S_1 =
                    \left\{
                        \begin{aligned}
                            x - y &= 5 \\
                            2x - 2y &= 10
                        \end{aligned}
                    \right.
                    \hspace{1cm}
                    S_2 =
                    \left\{
                        \begin{aligned}
                            x - y &= 1 \\
                            2x - 2y &= 10
                        \end{aligned}
                    \right.
                \]
                Las matrices expandidas de cada sistema de ecuaciones lineales es
                \[
                    M_{S_1\text{,}S_2} =
                    \left(
                    \begin{array}{cc|cc}
                        1 & -1 & 5  & 1 \\
                        2 & -2 & 10 & 10
                    \end{array}
                    \right)
                    \sim
                    \begin{aligned}
                        \frac{1}{2}F_2 &\mapsto F_2
                    \end{aligned}
                    \left(
                    \begin{array}{cc|cc}
                        1 & -1 & 5  & 1 \\
                        1 & -1 & 5  & 5
                    \end{array}
                    \right)
                \]
                \[
                    \sim
                    \begin{aligned}
                        F_2 - F_1 &\mapsto F_2
                    \end{aligned}
                    \left(
                    \begin{array}{cc|cc}
                        1 & -1 & 5  & 1 \\
                        0 & 0 & 0  & 4
                    \end{array}
                    \right)
                \]
                Despues de simples operaciones entre filas, se puede ver que \(S_1\) va a tener infinitas soluciones mientras que 
                \(S_2\), al ser un sistema inconsistente, no va a tener solución alguna.

			\item Un sistema de ecuaciones lineales tiene solución siempre que su sistema de ecuaciones lineales homogéneo asociado tenga solución. \\
                Esta proposición es algo curiosa, ya que la implicación que tiene es bastante parecida, y igual de erronea, a otras en el ambito de las matemáticas como:
                \enquote{Si una función es continua, es derivable}. Donde aunque suena bastante, si no es que completamente, correcto, el caso es totalmente lo contrario.
                Igualmente, por definición el sistema de ecuaciones lineales homogéneo asociado a otro sistema siempre va a tener una solución, la solución trivial donde todas
                las variables van a ser igual a 0. Sin embargo, esto no implica que el sistema de ecuaciones lineales tenga una solución. Un contra--ejemplo para esta proposición es:
                \[
                    S_{\text{base}} =
                    \left\{
                        \begin{aligned}
                            x + y &= 5 \\
                            2x + 2y &= 20 
                        \end{aligned}
                    \right.
                    \hspace{1cm}
                    S_{h} = 
                    \left\{
                        \begin{aligned}
                            x + y &= 0 \\
                            2x + 2y &= 0 
                        \end{aligned}
                    \right.
                \]
                Donde \(S_{\text{base}}\) es un sistema de ecuaciones lineales inconsistente y por ende sin soluciones.
                Pero \(S_h\), el sistema de ecuaciones lineales homogéneo asociado, tiene como mínimo la solución la dupla: \(\left(0, 0\right)\).
			\item El tipo de conjunto solución de un sistema de ecuaciones lineales y el del sistema de ecuaciones lineales homogéneo asociado siempre es el mismo. \\
                Esta proposición no es tan sencilla de ver que es falsa, sin embargo, solo es necesario describir un contra--ejemplo para mostrar que la proposición es falsa.
                Consideremos el sistema de ecuaciones lineales \(S_1\) y su sistema de ecuaciones lineales homogéneo asociado \(S_h\):
                \[
                    S_1 = 
                    \left\{
                        \begin{aligned}
                            x_1 + x_3 = 10 \\
                            x_2 + x_3 = 20 \\
                            x_3 = 30 \\
                        \end{aligned}
                    \right.
                    \hspace{1cm}
                    S_h = 
                    \left\{
                        \begin{aligned}
                            x_1 + x_3 = 0 \\
                            x_2 + x_3 = 0 \\
                            x_3 = 0 \\
                        \end{aligned}
                    \right.
                \]
                Podemos convertir a \(S_1\) a su matriz expandida y obtener su conjunto solución.
                \[
                    M_{S_{1}} =
                    \left(
                    \begin{array}{ccc|c}
                        1 & 0 & 1 & 10 \\
                        0 & 1 & 1 & 20 \\
                        0 & 0 & 1 & 30 \\
                    \end{array}
                    \right)
                    \sim
                    \begin{aligned}
                        F_1 - F_3 &\mapsto F_1 \\
                        F_2 - F_3 &\mapsto F_2 \\
                    \end{aligned}
                    \left(
                    \begin{array}{ccc|c}
                        1 & 0 & 0 & -20 \\
                        0 & 1 & 0 & -10 \\
                        0 & 0 & 1 & 30 \\
                    \end{array}
                    \right)
                \]
                \[
                    \left\{\left(-20, -10, 30\right)\right\}
                \]
                Ahora, podemos tomar la tupla \(-20, -10, 30\) y verificar si es una solución para \(S_h\), y concluir si \(S_1\) y \(S_h\) tienen el mismo conjunto solución:
                \[
                    \left\{
                        \begin{aligned}
                            \left(-20\right) + \left(30\right) = 0 \\
                            \left(-10\right) + \left(30\right) = 0 \\
                            \left(30\right) = 0 \\
                        \end{aligned}
                    \right.
                \]
                Solo es necesario evaluar la \(3^{\text{ra}}\) ecuación del sistema para verificar que, no, el conjunto solucíón de un sistema de ecuaciónes lineales 
                y el sistema de ecuaciones lineales homogéneo asociado \textbf{no siempre} es el mismo.

			\item Un sistema de ecuaciones lineales homogéneo con 27 variables y 13 ecuaciones puede no tener solución. \\
                Esta proposición es explicitamente falsa por faltar a la definición de un sistema de ecuaciones lineales homogéneo. 
                Ya que como \textbf{mínimo} se debe tener una solución para un sistema homogéneo, y este va a ser donde todas sus variables son 0.
			%\item Un sistema de ecuaciones lineales con 10 variables y 7 ecuaciones tiene infinitas soluciones. \\
			%\item Un sistema de ecuaciones lineales con 20 variables y 20 ecuaciones tiene solución única. \\
			%\item Un sistema de ecuaciones lineales homogéneo con 20 variables y 20 ecuaciones tiene solución única \\
			%\item Un sistema de ecuaciones lineales consistente con 20 variables y 20 ecuaciones tiene solución única. \\
        \end{enumerate}
    \item APLICACIONES
        \begin{itemize}
            \item \textbf{Fabricación de muebles} \\
                Un mueblero fabrica sillas, mesas para café y mesas para comedor. Se necesitan 10 minutos para lijar
                una silla, 6 para pintarla y 12 para barnizarla. Se necesitan 12 minutos para lijar una mesa para café,
                ocho para pintarla y 12 para barnizarla. Se necesitan 15 minutos para lijar una mesa para comedor,
                12 para pintarla y 18 para barnizarla. La mesa de lijado está disponible 16 horas a la semana, la mesa
                de pintura 11 horas a la semana y la mesa de barnizado 18 horas. 
                ¿Cuántas unidades de cada mueble deben fabricarse por semana de modo que las mesas de trabajo se ocupen todo el tiempo disponible?
                \\[1ex]
                Primero, vamos a tener que llenar el horario disponible de las mesas de trabajo, para eso vamos a mirar que tantas sillas, y tipos de 
                mesas tenemos que hacer en una semana para completar su horario. Luego, definamos las variables \(s = \text{número de sillas; } c = \text{número de mesas para café; } m = \text{número de mesas para comedor}\).
                Y, como solo vamos a tener 3 ecuaciones, para las 3 mesas de trabajo, con las mismas variables solo vamos a tener un sistema de ecuaciones.
                \[
                    S =
                    \left\{
                    \begin{aligned}
                        10s + 12c + 15m &= 960 \text{; Mesa de lijado} \\
                        6s + 8c + 12m &= 660 \text{; Mesa de pinado} \\
                        12s + 12c + 18m &= 1080 \text{; Mesa de barnizado}
                    \end{aligned}
                    \right.
                    M_S =
                    \left(
                    \begin{array}{ccc|c}
                        10 & 12 & 15 & 960 \\
                        6 & 8 & 12 & 660 \\
                        12 & 12 & 18 & 1080
                    \end{array}
                    \right)
                \]
                Tomamos la matriz expandida del sistema de ecuaciones lineales y le aplicamos el método de \emph{Gauss--Jordan}.
                \[
                    \sim
                    \begin{aligned}
                        \frac{1}{2}F_2 &\mapsto F_2 \\
                        \frac{1}{6}F_3 &\mapsto F_3
                    \end{aligned}
                    \left(
                    \begin{array}{ccc|c}
                        10 & 12 & 15 & 960 \\
                        3 & 4 & 6 & 330 \\
                        2 & 2 & 3 & 180
                    \end{array}
                    \right)
                    \sim
                    \begin{aligned}
                        F_1 - 5F_3 &\mapsto F_1 \\
                        \frac{1}{2}F_1 &\mapsto F_1 \\
                        F_1 &\leftrightarrow F_2
                    \end{aligned}
                    \left(
                    \begin{array}{ccc|c}
                        3 & 4 & 6 & 330 \\
                        0 & 1 & 0 & 30 \\
                        2 & 2 & 3 & 180
                    \end{array}
                    \right)
                \]
                \[
                    \sim
                    \begin{aligned}
                        F_1 - F_3 &\mapsto F_1 \\
                        F_1 &\leftrightarrow F_3
                    \end{aligned}
                    \left(
                    \begin{array}{ccc|c}
                        2 & 2 & 3 & 180 \\
                        0 & 1 & 0 & 30 \\
                        1 & 2 & 3 & 150 
                    \end{array}
                    \right)
                    \sim
                    \begin{aligned}
                        F_1 - F_3 &\mapsto F_1 \\
                        F_3 - 2F_2 &\mapsto F_3 \\
                        F_3 - F_1 &\mapsto F_3 \\
                        \frac{1}{3}F_3 &\mapsto F_3 \\
                    \end{aligned}
                    \left(
                    \begin{array}{ccc|c}
                        1 & 0 & 0 & 30 \\
                        0 & 1 & 0 & 30 \\
                        0 & 0 & 1 & 20
                    \end{array}
                    \right)
                \]
                Ahora, hacemos sustitución hacia atrás, obtenemos, y verificamos que estos valores sean correctos
                \[
                    \left\{
                        \begin{aligned}
                            s &= 30 \\
                            c &= 30 \\
                            m &= 20 \\
                        \end{aligned}
                    \right.
                    \hspace{1cm}
                    \left\{
                    \begin{aligned}
                        10(30) + 12(30) + 15(20) &= 300 + 360 + 300 =& 960 = 960 \\
                        6(30) + 8(30) + 12(20) &= 180 + 240 + 240 =& 660 = 660 \\
                        12(30) + 12(30) + 18(20) &= 360 + 360 + 360 =& 1080 = 1080
                    \end{aligned}
                    \right.
                \]
                \[
                    S = \left\{\left(30, 30, 20\right)\right\}
                \]
                Es decir, vamos a tener que hacer 30 sillas, 30 mesas para café y 20 mesas para comedor, para ocupar todo el horario de uso de las mesas de trabajo.

            \item \textbf{Hacer insecticidas} \\
                Para fabricar insecticidas se utilizan tres clases de compuestos. 
                Una unidad del insecticida Magnon requiere 
                    10mls de Nuvan, 
                    30mls de Citronela B y 
                    60mls de petróleo. 
                Una unidad del Baygon requiere
                    20mls de Nuvan, 
                    30mls de Citronela y 
                    50mls de petróleo. 
                Una unidad del insecticida Nocaut, requiere
                    50mls de Nuvan y
                    50mls de petróleo. 
                Si se disponen de 
                    1600mls de Nuvan,
                    1200mls de Citronela y
                    3200mls de petróleo. 
                Determine cuántas unidades de los tres insecticidas pueden producirse usando todos los componentes disponibles.
                \\[1ex]
                Este problema es bastante similar al anterior, tenemos que encontrar el número de diferentes elementos que podemos realizar con 
                una cantidad dada de ingredientes o componentes de los elementos.
                Definamos las variables \(m = \text{in. Magnon; } b = \text{in. Baygon; } n = \text{in. Nocaut}\), y el sistema de ecuaciones junto a su matriz expandida:
                \[
                    S = 
                    \left\{
                        \begin{aligned}
                            10m + 20b + 50n &= 1600 \text{; Nuvan}\\
                            30m + 30b &= 1200 \text{; Citronela}\\
                            60m + 50b + 50n &= 3200 \text{; Petróleo}
                        \end{aligned}
                    \right.
                    \hspace{1cm}
                    M_S = 
                    \left(
                    \begin{array}{ccc|c}
                        10 & 20 & 50 & 1600 \\
                        30 & 30 & 0 & 1200 \\
                        60 & 50 & 50 & 3200
                    \end{array}
                    \right)
               \]
               De igual forma que el problema anterior, vamos a aplicar a esta matriz el método de \emph{Gauss}, encontrar las soluciones de las variables, su conjunto solución y verificar si es correcto el resultado.
               \[
                    \sim
                    \begin{aligned}
                        \frac{1}{10}F_1 &\mapsto F_1 \\
                        \frac{1}{30}F_2 &\mapsto F_2 \\
                        \frac{1}{10}F_3 &\mapsto F_3
                    \end{aligned}
                    \left(
                    \begin{array}{ccc|c}
                        1 & 2 & 5 & 160 \\
                        1 & 1 & 0 & 40 \\
                        6 & 5 & 5 & 320
                    \end{array}
                    \right)
                    \sim
                    \begin{aligned}
                        F_3 - F_1 &\mapsto F_3 \\
                        F_3 - 3F_2 &\mapsto F_3 \\
                        \frac{1}{2}F_3 &\mapsto F_3 \\
                    \end{aligned}
                    \left(
                    \begin{array}{ccc|c}
                        1 & 2 & 5 & 160 \\
                        1 & 1 & 0 & 40 \\
                        1 & 0 & 0 & 20
                    \end{array}
                    \right)
               \]
               \[  
                   \sim
                    \begin{aligned}
                        F_1 - F_3 &\mapsto F_1 \\
                        F_2 - F_3 &\mapsto F_2 \\
                        F_1 - 2F_2 &\mapsto F_2 \\
                        \frac{1}{5}F_1 &\mapsto F_1 \\
                        F_1 &\leftrightarrow F_3 
                    \end{aligned}
                    \left(
                    \begin{array}{ccc|c}
                        1 & 0 & 0 & 20 \\
                        0 & 1 & 0 & 20 \\
                        0 & 0 & 1 & 20
                    \end{array}
                    \right)
                    \hspace{1cm}
                    \left\{
                        \begin{aligned}
                            10(20) + 20(20) + 50(20) &= 200 + 400 + 1000 &= 1600 \\
                            30(20) + 30(20) &= 600 + 600 &= 1200 \\
                            60(20) + 50(20) + 50(20) &= 1200 + 1000 + 1000 &= 3200
                        \end{aligned}
                    \right.
               \]
               Resultando que, vamos a tener que 20 unidades de los 3 insecticidas para usar todos los componentes químicos disponibles.
        \end{itemize}
\end{enumerate}
\end{document}
