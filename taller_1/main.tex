\documentclass{article}

\def\MyAuthor{Martín Steven Hernández Ortiz}
\def\MyEmail{mahernandezor@unal.edu.co}
\def\MyClass{Álgebra Lineal}
\def\MyTitle{Taller 1}
\def\MyDate{\today}

\author{\MyAuthor{}}
\title{\MyClass{} \-- \MyTitle{}}


%%% Template Packages %%%

\usepackage{graphicx} % Images
\usepackage{tcolorbox} % Color Box
\usepackage[%
vmargin=2.25cm,%
hmargin=2.25cm%
]{geometry} % Page Geometry
\usepackage{fancyhdr} % Header / Footer Styles

\usepackage{ragged2e} % Text Align
\usepackage{amsmath} % Math Align

\usepackage{booktabs} % Tables Package

\usepackage{mathtools} % Math General
\usepackage{amsthm} % Math Envs
\usepackage{unicode-math} % Math Symbols

\usepackage{polyglossia} % Language
    \setdefaultlanguage{spanish}

\renewcommand{\thefootnote}{\Roman{footnote}} % Changing footnotes arabic to roman numbers

%%% Template Styles %%%

% Header / Footer Styles
\pagestyle{fancy}
\RenewDocumentCommand{\headrule}{}{%
    \rule[0.1cm]{\textwidth}{0.1mm}%
}
\RenewDocumentCommand{\footrule}{}{%
    \rule[0.1cm]{\textwidth}{0.1mm}%
}

\fancyhf[HC]{{\slshape \MyTitle{}}}

% Redefine \maketitle
\RenewDocumentCommand{\maketitle}{s}{%
    \begin{@twocolumntrue}%
        \begin{minipage}{0.3\textwidth}%
            \includegraphics[width=0.85\textwidth]{../template/src/unal_logo.pdf}%
        \end{minipage}%
        \begin{minipage}{0.7\textwidth}{{%
        \begin{Center}%
            {\large \itshape \MyClass{}} \\[1ex]
            {\huge  \slshape \MyTitle{}} \\[4ex]
            {\Large  \MyAuthor{}} \\[0ex]
            {\small  \MyEmail{}} \\[4ex] 
            \MyDate{}
        \end{Center}%
        }}%
        \end{minipage}%
    \end{@twocolumntrue}%
    \vspace{0.5cm}%
    \begin{Center}%
        \rule[0cm]{\textwidth}{0.1mm}%
    \end{Center}%
    \vspace{0.5cm}%
}

%%% Template Math stuff %%%

% Theorems
\newtheorem{TMPMathTheorem}{Teorema}
\NewDocumentEnvironment{theorem}{+b} {%
    \begin{TMPMathTheorem}%
        #1 %
    \end{TMPMathTheorem}%
} {}

% Collorary
\newtheorem{TMPMathCorollary}[TMPMathTheorem]{Corolario}
\NewDocumentEnvironment{corollary}{+b} {%
    \begin{TMPMathCorollary}%
        #1 %
    \end{TMPMathCorollary}%
} {}

% Definitions
\newtheorem{TMPMathDefinition}[TMPMathTheorem]{Definición}
\NewDocumentEnvironment{definition}{+b} {%
    \begin{tcolorbox}[left=0mm,right=0mm]%
        \begin{TMPMathDefinition}%
            #1 %
            \begin{FlushRight}%
                \(\bigtriangleup{}\)%
            \end{FlushRight}%
        \end{TMPMathDefinition}%
    \end{tcolorbox}%
} {}



\begin{document}
\maketitle

\begin{enumerate}
    \setcounter{enumi}{4}
    \extramarks{Punto 5}{}
    \item Dar condiciones sobre el parámetro \(\alpha\), para que el sistema:
        \[
            \left\{
                \begin{aligned}
                    x + y - z = 2 \\
                    x + 2y + z = 3 \\
                    x + y + \left(\alpha^2 - 5\right)z = \alpha
                \end{aligned}
            \right.
        \]
        Para un mejor desarrollo del punto, primero se va a convertir el sistema de ecuaciones a su matriz aumentada, 
        para después realizar parcialmente la eliminación de \emph{Gauss}, ya que no sabemos si el valor de \(\alpha\) nos favorezca para que el pivote sea \(1\).
        \[
            \text{Base}
            \left(
            \begin{array}{ccc|c}
                1 & 1 & -1 & 2 \\
                1 & 2 & 1 & 3 \\
                1 & 1 & \alpha^2 -5 & \alpha \\
            \end{array}
            \right)
            \sim
            F_2 - F_1 \mapsto F_2
            \left(
            \begin{array}{ccc|c}
                1 & 1 & -1 & 2 \\
                0 & 1 & 2 & 1 \\
                1 & 1 & \alpha^2 - 5 & \alpha \\
            \end{array}
            \right)
        \]
        \[
            \sim
            F_3 - F_1 \mapsto F_3
            \left(
            \begin{array}{ccc|c}
                1 & 1 & -1 & 2 \\
                0 & 1 & 2 & 1 \\
                0 & 0 & \alpha^2 - 4 & \alpha - 2 \\
            \end{array}
            \right)
        \]
        Desde este punto podemos deducir, mediante la sustitución hacia atrás, que:
        \[
            \left\{
                \begin{aligned}
                    x + y - z = 2 \\
                    y + 2z = 1 \\
                    \left(\alpha^2 - 4\right)z = \alpha - 2
                \end{aligned}
            \right.
            \hspace{1cm}
            \begin{aligned}
                x &= 1 + \frac{3\alpha - 6}{\alpha^2 - 4} \text{,} \\
                y &= 1 - \frac{2\alpha - 4}{\alpha^2 - 4} \hspace{1ex} \text{y}  \\
                z &= \frac{\alpha - 2}{\alpha^2 - 4}
            \end{aligned}
        \]
        \begin{enumerate}[label=\listAlph]
            \item No tenga solución. \\

            \item Tenga infinitas soluciones (dar la solución general). \\
                Solo es necesario que \(\alpha = 2\), para que en la \(3^{\text{ra}}\) ecuación del sistema se tenga: \(0x + 0y + 0z = 0\).
                Lo cual se puede interpretar como  que \(x \text{, } y \text{ y } z\) tienen infinitas respuestas, 
                sin embargo \(x \text{ y } y\) se pueden explicar en terminos de \(z\).
                \[
                    \begin{aligned}
                        y &= 1 - 2z \\
                        x &= 2 + z - y \\
                        &= 1 + 3z \\
                    \end{aligned}
                \]
                Obteniendo el conjunto solución
                \[
                    \left\{\left(1 + 3z, 1 - 2z, z\right) | z \in \mathbb{R} \right\}
                \]

            \item Tenga solución única (dar la solución). \\
                Si se toma \(\alpha = 0\), en la matriz del sistema de ecuaciones se podría modificar de tal forma que
                \[
                    \left(
                    \begin{array}{ccc|c}
                        1 & 1 & -1 & 2 \\
                        0 & 1 & 2 & 1 \\
                        0 & 0 & -4 & -2 \\
                    \end{array}
                    \right)
                    \sim
                    F_3 + F_1 \mapsto F_3
                    \left(
                    \begin{array}{ccc|c}
                        1 & 1 & -1 & 2 \\
                        0 & 1 & 2 & 1 \\
                        0 & 0 & -5 & 0 \\
                    \end{array}
                    \right)
                    \sim
                    -\frac{1}{5}F_3 \mapsto F_3
                    \left(
                    \begin{array}{ccc|c}
                        1 & 1 & -1 & 2 \\
                        0 & 1 & 2 & 1 \\
                        0 & 0 & 1 & 0 \\
                    \end{array}
                    \right)
                \]
                Realizando sustitución hacia atras obtenemos
                \[
                    \left\{
                        \begin{aligned}
                            x + y - z = 2 \\
                            y + 2z = 1 \\
                            z = 0
                        \end{aligned}
                    \right.
                    \hspace{1cm}
                    \begin{aligned}
                        x &= 2 + z - y = 1 \\
                        y &= 1 -2z = 1 \\
                        z &= 0 \\
                    \end{aligned}
                \]
                \[
                    \text{Siendo el conjunto solución: }
                    \left\{\left(1, 1, 0\right)\right\}
                \]
                El cual podemos verificar en el sistema de ecuaciones:
                \[
                    \left\{
                        \begin{aligned}
                            1 + 1 - 0 = 2 &= 2 \\
                            1 + 2\left(0\right) = 1 &= 1 \\
                            0 &= 0
                        \end{aligned}
                    \right.
                \]
        \end{enumerate}

    \extramarks{Punto 6}{}
    \item Si al escalonar la matriz aumentada de un sistema de ecuaciones lineales, se obtiene
        \[
            \left(
            \begin{array}{ccccc|c}
                \sqrt{2}    & 0        & 3        & -1       & 4        & 0 \\
                0           & 0        & 2        & \pi      & -1       & 1 \\
                0           & 0        & 0        & \alpha   & 1        & -5 \\
                0           & 0        & 0        & 0        & b^2 - b  & b \\
            \end{array}
            \right)
        \]
        \begin{itemize}
            \item Es el sistema consistente cuando a = b = 0? En caso de serlo, es la solución única?
            \item Es el sistema consistente cuando a = 1 y b = 0? En caso de serlo, es la solución única?
            \item Es el sistema consistente cuando a = 0 y b = 1? En caso de serlo, es la solución única?
            \item Si b = 2 y a \(\neq\) 0, qué puede decirse del conjunto solución?
            \item Si b = 1 y a \(\neq\) 0, qué puede decirse del conjunto solución?
            \item Si a \(\neq\) 0, dé un valor de b (diferente de 0), en caso de que exista, para que el sistema sea consistente.
            \item Si a \(\neq\) 0, para que valores de b el sistema tiene infinitas soluciones?
            \item Si a \(\neq\) 0, para que valores de b el sistema tiene solución única?
            \item Si a \(\neq\) 0, para que valores de b el sistema es inconsistente?
        \end{itemize}

    \setcounter{enumi}{7}
    \extramarks{Punto 8}{}
    \item Justifique POR QUÉ cada una de las siguientes afirmaciones son VERDADERAS:
        \begin{enumerate}[label=\listAlph]
			\item Si un sistema de ecuaciones lineales tiene solución única, su sistema de ecuaciones lineales homogéneo asociado también tiene solución única.
			\item Un sistema de ecuaciones lineales con 14 variables y 10 ecuaciones no tiene solución única.
			\item Un sistema de ecuaciones lineales con 27 variables y 13 ecuaciones puede no tener solución.
			\item Un sistema de ecuaciones lineales con 100 variables y 300 ecuaciones puede tener solución única.
			\item Un sistema de ecuaciones lineales homogéneo con 10 variables y 7 ecuaciones tiene infinitas soluciones.
			\item Un sistema de ecuaciones lineales homogéneo con 14 variables y 10 ecuaciones no tiene solución única.
			\item Un sistema de ecuaciones lineales homogéneo con 100 variables y 300 ecuaciones puede tener solución única.
			\item Un sistema de ecuaciones lineales consistente con 10 variables y 7 ecuaciones tiene infinitas soluciones.
			\item Un sistema de ecuaciones lineales consistente con 14 variables y 10 ecuaciones no tiene solución única.
			\item Un sistema de ecuaciones lineales consistente con 100 variables y 300 ecuaciones puede tener solución única.
        \end{enumerate}

    \extramarks{Punto 9}{test?}
    \item Justifique POR QUÉ cada una de las siguientes afirmaciones son FALSAS
        \begin{enumerate}[label=\listAlph]
			\item Si un sistema de ecuaciones lineales tiene solución, cualquier otro sistema de ecuaciones lineales con la misma matriz de coeficientes también tiene solución
			\item Un sistema de ecuaciones lineales tiene solución siempre que su sistema de ecuaciones lineales homogéneo asociado tenga solución.
			\item El tipo de conjunto solución de un sistema de ecuaciones lineales y el del sistema de ecuaciones lineales homogéneo asociado siempre es el mismo
			\item Un sistema de ecuaciones lineales homogéneo con 27 variables y 13 ecuaciones puede no tener solución.
			\item Un sistema de ecuaciones lineales con 10 variables y 7 ecuaciones tiene infinitas soluciones
			\item Un sistema de ecuaciones lineales con 20 variables y 20 ecuaciones tiene solución única.
			\item Un sistema de ecuaciones lineales homogéneo con 20 variables y 20 ecuaciones tiene solución única
			\item Un sistema de ecuaciones lineales consistente con 20 variables y 20 ecuaciones tiene solución única.
        \end{enumerate}
\end{enumerate}
\end{document}
