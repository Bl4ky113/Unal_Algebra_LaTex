\item Encuentre el vector de coordenadas de u con respecto a la base \(\fb\) del espacio \(V\), es decir, \(\veccoord{u}{\fb}\)
    \begin{enumerate}[label=\listAlph]
        \item Para \(V = \realR^3, \hspace{0.25cm} \fb = \fset{\fsmvec{3,2,2}, \fsmvec{-1,2,1}, \fsmvec{0,1,0}}\), \(u = \fsmvec{5,3,1}\). \\
            Para obtener a \(\veccoord{u}{\fb}\), primero debemos recordar que un vector de coordenadas son los escalares de la combinación lineal de la base dada para obtener al vector.
            Entonces, simplemente vamos a realizar un sistema de ecuaciones apartir de los escalares, es decir:
            \[
                u = \lambda_1 \fvec{3,2,2} + \lambda_2 \fvec{-1,2,1} + \lambda_3 \fvec{0,1,0} 
                = \fvec{3\lambda_1, 2\lambda_1, 2\lambda_1} + \fvec{-\lambda_2, 2\lambda_2, \lambda_2} + \fvec{0, \lambda_3, 0}
                = \fvec{3\lambda_1 -\lambda_2, 2\lambda_1 + 2\lambda_2 + \lambda_3,  2\lambda_1}
            \]
            Ahora, con cada componente del vector obtenido vamos a igualarlos a cada componente del vector \(u\), 
            para así apartir de la matriz de coeficientes poder obtener el valor de los escalares.
            \[
                \left(
                \begin{array}{ccc|c}
                    3 & -1 & 0 & 5 \\
                    2 & 2 & 1 & 3 \\
                    2 & 1 & 0 & 1
                \end{array}
                \right)
                \sim
                \begin{aligned}
                    F_1 + F_3 &\mapsto F_1 \\
                    F_2 - F_3 &\mapsto F_2 \\
                \end{aligned}
                \left(
                \begin{array}{ccc|c}
                    5 & 0 & 0 & 6 \\
                    0 & 1 & 1 & 2 \\
                    2 & 1 & 0 & 1
                \end{array}
                \right)
                \sim
                \begin{aligned}
                    \frac{1}{5}F_1 &\mapsto F_1 \\
                    F_2 &\leftrightarrow F_3 \\
                \end{aligned}
                \left(
                \begin{array}{ccc|c}
                    1 & 0 & 0 & \frac{6}{5} \\
                    2 & 1 & 0 & 1 \\
                    0 & 1 & 1 & 2
                \end{array}
                \right)
            \]
            \[
                \sim
                \begin{aligned}
                    F_2 - 2F_1 &\mapsto F_2
                \end{aligned}
                \left(
                \begin{array}{ccc|c}
                    1 & 0 & 0 & \frac{6}{5} \\
                    0 & 1 & 0 & -\frac{7}{5} \\
                    0 & 1 & 1 & 2
                \end{array}
                \right)
                \sim
                \begin{aligned}
                    F_3 - F_2 &\mapsto F_3 
                \end{aligned}
                \left(
                \begin{array}{ccc|c}
                    1 & 0 & 0 & \frac{6}{5} \\
                    0 & 1 & 0 & -\frac{7}{5} \\
                    0 & 0 & 1 & \frac{17}{5}
                \end{array}
                \right)
            \]
            Por \emph{Gauss-Jordan}, obtenemos que \(\lambda_1 = \frac{6}{5}, \lambda_2 = -\frac{7}{5}, \lambda_3 = \frac{17}{5}\), 
            cuales son los componentes del vector de coordenadas de \(u\) respecto a \(\fb\).
            \[
                \veccoord{u}{\fb} = \fvec{\frac{6}{5},-\frac{7}{5},\frac{17}{5}}
            \]
        \item Para \(V = \fp_2, \hspace{0.25cm} \fb = \fset{t^2 + t, t - 1, t + 1}\), \(u = 3t^2 - t + 2\). \\
            Se va a realizar el mismo proceso que el numeral anterior, para evitar explicar detalladamente el mismo proceso varias veces. 
            Primero, vamos a obtener el sistema de ecuaciones de los escalares de \(u\) en la base \(\fb\), 
            y apartir de este su matriz de coeficientes de donde se va a realizar \emph{Gauss-Jordan} 
            para obtener el valor de cada escalar, y por ende, las componentes del vector coordenada de \(u\) respecto a \(B\).
            \[
                u = \lambda_1t^2 + (\lambda_1 + \lambda_2 + \lambda_3)t + (-\lambda_2 + \lambda_3)
            \]
            \[
                \begin{aligned}
                    &\left(
                    \begin{array}{ccc|c}
                        1 & 0 & 0 & 3 \\
                        1 & 1 & 1 & -1 \\
                        0 & -1 & 1 & 2 
                    \end{array}
                    \right)
                    \sim
                    \begin{aligned}
                        F_2 - F_1 &\mapsto F_2 \\
                        F_3 + F_2 &\mapsto F_3 \\
                        \frac{1}{2} F_3 &\mapsto F_3 \\
                    \end{aligned}
                    \left(
                    \begin{array}{ccc|c}
                        1 & 0 & 0 & 3 \\
                        0 & 1 & 1 & -4 \\
                        0 & 0 & 1 & -1
                    \end{array}
                    \right)
                    \sim
                    \begin{aligned}
                        F_2 - F_3 &\mapsto F_2
                    \end{aligned}
                    \left(
                    \begin{array}{ccc|c}
                        1 & 0 & 0 & 3 \\
                        0 & 1 & 0 & -3 \\
                        0 & 0 & 1 & -1
                    \end{array}
                    \right)
                \end{aligned}
            \]
            \[
                \veccoord{u}{\fb} = \fvec{3, -3, -1}
            \]
        \item Para \(V = \fm_{2 \times 2}, \hspace{0.25cm} \fb = \fset{\fsmmatrix{1,1;0,0}, \fsmmatrix{0,0;1,1}, \fsmmatrix{1,0;0,1}, \fsmmatrix{0,1;1,1}} u = \fsmmatrix{3,-1;4,2} \) \\
            De manera similar a los puntos anteriores. Se va obtener un sistema de ecuaciones de los escalares de la base \(\fb\) respecto a \(u\), 
            para luego obtener la matriz escalonada reducida equivalente de la matriz de coeficientes y finalmente obtener el valor de los escalares 
            y, por ende, los componentes del vector coordenada de \(u\) respecto a \(\fb\).
            \[
                u = \fmatrix{
                    \lambda_1 + \lambda_3, \lambda_1 + \lambda_4;
                    \lambda_2 + \lambda_4, \lambda_2 + \lambda_3 + \lambda_4
                }
            \]
            \[
                \left(
                \begin{array}{cccc|c}
                    1 & 0 & 1 & 0 & 3 \\
                    1 & 0 & 0 & 1 & -1 \\
                    0 & 1 & 0 & 1 & 4 \\
                    0 & 1 & 1 & 1 & 2
                \end{array}
                \right)
                \sim
                \begin{aligned}
                    F_2 - F_1 &\mapsto F_2 \\
                    F_3 &\leftrightarrow F_4 \\
                \end{aligned}
                \left(
                \begin{array}{cccc|c}
                    1 & 0 & 1 & 0 & 3 \\
                    0 & 0 & -1 & 1 & -4 \\
                    0 & 1 & 1 & 1 & 2 \\
                    0 & 1 & 0 & 1 & 4
                \end{array}
                \right)
                \sim
                \begin{aligned}
                    F_1 - F_3 &\mapsto F_1 \\
                    F_2 &\leftrightarrow F_4
                \end{aligned}
                \left(
                \begin{array}{cccc|c}
                    1 & 0 & 0 & 0 & 5 \\
                    0 & 1 & 0 & 1 & 4 \\
                    0 & 0 & 1 & 0 & -2 \\
                    0 & 0 & -1 & 1 & -4
                \end{array}
                \right)
            \]
            \[
                \sim
                \begin{aligned}
                    F_4 + F_3 &\mapsto F_4 \\
                \end{aligned}
                \left(
                \begin{array}{cccc|c}
                    1 & 0 & 0 & 0 & 5 \\
                    0 & 1 & 0 & 1 & 4 \\
                    0 & 0 & 1 & 0 & -2 \\
                    0 & 0 & 0 & 1 & -6
                \end{array}
                \right)
                \sim
                \begin{aligned}
                    F_2 - F_4 &\mapsto F_2 \\
                \end{aligned}
                \left(
                \begin{array}{cccc|c}
                    1 & 0 & 0 & 0 & 5 \\
                    0 & 1 & 0 & 0 & 10 \\
                    0 & 0 & 1 & 0 & -2 \\
                    0 & 0 & 0 & 1 & -6
                \end{array}
                \right)
            \]
            \[
                \veccoord{u}{\fb} = \fvec{5,10,-2,-6}
            \]
    \end{enumerate}
