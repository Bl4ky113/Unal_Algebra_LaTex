\item Justifique que \(\fb\) y \(\fb′\) son bases de \(V\), calcule el vector \(u\) cuyas coordenadas en una de las dos bases se dan y calcule las coordenadas del vector \(u\) en la otra base.
    \begin{enumerate}[label=\listAlph]
        \item \(V = \realR^3 \hspace{0.25cm} \fb = \fset{\fsmvec{1,2,-1}, \fsmvec{1,1,5}, \fsmvec{-2,1,-1}} \hspace{0.25cm} \fb' = \fset{\fsmvec{1,0,-1}, \fsmvec{1,1,0}, \fsmvec{0,1,-1}} \hspace{0.25cm} \veccoord{u}{\fb} = \fsmvec{2,1,-2}\).
        \item \(V = \fm_{2 \times 2} \hspace{0.25cm} \begin{aligned}\fb = \fset{\fsmmatrix{-1,0;1,0}, \fsmmatrix{-1,-1;1,0}, \fsmmatrix{-1,0;1,1}, \fsmmatrix{-1,1;1,1}}\\\fb' = \fset{\fsmmatrix{-1,0;1,0}, \fsmmatrix{-1,1;1,0}, \fsmmatrix{-1,0;1,1}, \fsmmatrix{-1,1;1,1}}\end{aligned} \hspace{0.25cm} \veccoord{u}{\fb'}=\fsmvec{0,1,-2,-1}\).
        \item \(V = \fp_2 \hspace{0.25cm} \fb = \fset{2 - x, 1 - x^2, 1 + x}, \fb' = \fset{1 + x, 2 - x, 1 - x^2} \hspace{0.25cm} \veccoord{u}{\fb'}=\fsmvec{-3,0,-2}\).
        \item \(V = \fp_3 \hspace{0.25cm} \fb = \fset{1, x, x^2, x^3}, \fb' = \fset{1 + x, x + x^2, x^2 + x^3, 1 - x^3} \hspace{0.25cm} \veccoord{u}{\fb'} = \fsmvec{0,1,-2,-1}\)
        \item \(V = \fset{\fsmvec{x,y,z}: x + y + z = 0} \hspace{0.25cm} \fb = \fset{\fsmvec{2,-1,-1}, \fsmvec{3,0,-3}}, \fb' = \fset{\fsmvec{-3,-1,4}, \fsmvec{0,2,-2}}, \veccoord{u}{\fb} = \fsmvec{1,-1}\)
    \end{enumerate}
