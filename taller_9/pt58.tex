\item Justifique que \(\fb\) y \(\fb′\) son bases de \(V\), calcule el vector \(u\) cuyas coordenadas en una de las dos bases se dan y calcule las coordenadas del vector \(u\) en la otra base.
    \begin{enumerate}[label=\listAlph]
        \item \(V = \realR^3 \hspace{0.25cm} \fb = \fset{\fsmvec{1,2,-1}, \fsmvec{1,1,5}, \fsmvec{-2,1,-1}} \hspace{0.25cm} \fb' = \fset{\fsmvec{1,0,-1}, \fsmvec{1,1,0}, \fsmvec{0,1,-1}} \hspace{0.25cm} \veccoord{u}{\fb} = \fsmvec{2,1,-2}\). \\
            Primero, ya sabemos que las bases de \(\realR^3\) es de 3 elementos, o es de dimensión 3, entonces como\(\fb\) y \(\fb'\) tienen solo 3 elementos son \(l.i\) 
            y al ser un subconjunto por el teorema de máximal y mínimal del conjunto \(l.i\) también son una base.
            Ahora, debemos mirar el valor de \(u\) para así pasarlo a \(\veccoord{u}{\fb'}\), lo cual se va a hacer solamente evalualando los escalares dados del vector coordenada respecto a \(\fb\) con la misma base.
            \[
                u 
                = 2 \fvec{1,2,-1} + \fvec{1,1,5} - 2\fvec{-2,1,-1} 
                = \fvec{2,4,-2} + \fvec{1,1,5} - \fvec{-4,2,-2}
                = \fvec{7,3,5}
            \]
            Consecuentemente, vamos a igualar cada componente de \(u\) por cada componente que se obtiene al hacer una combinación lineal por escalares fijos pero arbitrarios de \(\fb'\),
            para así obtener a \(\veccoord{u}{\fb'}\).
            \[
                \fvec{7,3,5} = \lambda_1 \fvec{1,0,-1} + \lambda_2 \fvec{1,1,0} + \lambda_3 \fvec{0,1,-1}
                = \fvec{\lambda_1 + \lambda_2, \lambda_2 + \lambda_3, -\lambda_1 - \lambda_3}
            \]
            \[
                \left(
                \begin{array}{ccc|c}
                    1 & 1 & 0 & 7 \\
                    0 & 1 & 1 & 3 \\
                    -1 & 0 & -1 & 5 
                \end{array}
                \right)
                \sim
                \begin{aligned}
                    F_3 + F_1 &\mapsto F_3 \\
                \end{aligned}
                \left(
                \begin{array}{ccc|c}
                    1 & 1 & 0 & 7 \\
                    0 & 1 & 1 & 3 \\
                    0 & 1 & -1 & 12
                \end{array}
                \right)
                \sim
                \begin{aligned}
                    F_3 - F_2 &\mapsto F_3 \\
                    -\frac{1}{2}F_3 &\mapsto F_3 
                \end{aligned}
                \left(
                \begin{array}{ccc|c}
                    1 & 1 & 0 & 7 \\
                    0 & 1 & 1 & 3 \\
                    0 & 0 & 1 & -\frac{9}{2}
                \end{array}
                \right)
            \]
            \[
                \sim
                \begin{aligned}
                    F_2 - F_3 &\mapsto F_2
                \end{aligned}
                \left(
                \begin{array}{ccc|c}
                    1 & 1 & 0 & 7 \\
                    0 & 1 & 0 & \frac{15}{2} \\
                    0 & 0 & 1 & -\frac{9}{2}
                \end{array}
                \right)
                \sim
                \begin{aligned}
                    F_1 - F_2 &\mapsto F_1
                \end{aligned}
                \left(
                \begin{array}{ccc|c}
                    1 & 0 & 0 & -\frac{1}{2} \\
                    0 & 1 & 0 & \frac{15}{2} \\
                    0 & 0 & 1 & -\frac{9}{2}
                \end{array}
                \right)
            \]
            \[
                \veccoord{u}{\fb'} = \fvec{-\frac{1}{2},\frac{15}{2},-\frac{9}{2}}
            \]
        %\item \(V = \fm_{2 \times 2} \hspace{0.25cm} \begin{aligned}\fb &= \fset{\fsmmatrix{-1,0;1,0}, \fsmmatrix{-1,-1;1,0}, \fsmmatrix{-1,0;1,1}, \fsmmatrix{-1,1;1,1}}\\\fb' &= \fset{\fsmmatrix{-1,0;1,0}, \fsmmatrix{-1,1;1,0}, \fsmmatrix{-1,0;1,1}, \fsmmatrix{-1,1;1,1}}\end{aligned} \hspace{0.25cm} \veccoord{u}{\fb'}=\fsmvec{0,1,-2,-1}\).
        \setcounter{enumii}{2}
        \item \(V = \fp_2 \hspace{0.25cm} \fb = \fset{2 - x, 1 - x^2, 1 + x}, \fb' = \fset{1 + x, 2 - x, 1 - x^2} \hspace{0.25cm} \veccoord{u}{\fb'}=\fsmvec{-3,0,-2}\). \\
            Inicialmente, sabemos que la dimensión de \(\fp_2\) es de 3, entonces \(\fb\) y \(fb'\) al tener 3 elementos y ser un subconjunto son bases de \(\fp_2\).
        %\item \(V = \fp_3 \hspace{0.25cm} \fb = \fset{1, x, x^2, x^3}, \fb' = \fset{1 + x, x + x^2, x^2 + x^3, 1 - x^3} \hspace{0.25cm} \veccoord{u}{\fb'} = \fsmvec{0,1,-2,-1}\)
        \setcounter{enumii}{4}
        \item \(V = \fset{\fsmvec{x,y,z}: x + y + z = 0} \hspace{0.25cm} \fb = \fset{\fsmvec{2,-1,-1}, \fsmvec{3,0,-3}}, \fb' = \fset{\fsmvec{-3,-1,4}, \fsmvec{0,2,-2}}, \veccoord{u}{\fb} = \fsmvec{1,-1}\) \\
            Como la dimensión de un hiperplano en \(\realR^n\) es de \(n - 1\), para nuestro caso 2, entonces al tener \(\fb\) y \(\fb'\) solo dos elementos, y ser subconjuntos sabemos que ambas son bases.
            Ahora, obtengamos a \(u\) como en los puntos anteriores y consecuentemente a \(\veccoord{u}{\fb'}\)
            \[
                u = \fvec{2,-1,-1} - \fvec{3,0,-3} = \fvec{-1,-1,2}
            \]
            \[
                \left(
                \begin{array}{cc|c}
                    -3 & 0 & -1 \\
                    -1 & 2 & -1  \\
                    4 & -2 & 2
                \end{array}
                \right)
                \sim
                \begin{aligned}
                    F_1 + F_3 &\mapsto F_1
                \end{aligned}
                \left(
                \begin{array}{cc|c}
                    1 & -2 & 1 \\
                    -1 & 2 & -1  \\
                    4 & -2 & 2
                \end{array}
                \right)
                \sim
                \begin{aligned}
                    F_2 + F_1 &\mapsto F_2 \\
                    F_3 - 4F_1 &\mapsto F_3
                \end{aligned}
                \left(
                \begin{array}{cc|c}
                    1 & -2 & 1 \\
                    0 & 0 & 0  \\
                    0 & 6 & -2
                \end{array}
                \right)
            \]
            \[
                \sim
                \begin{aligned}
                    F_3 &\leftrightarrow F_2 \\
                    \frac{1}{6}F_2 &\mapsto F_2 \\
                \end{aligned}
                \left(
                \begin{array}{cc|c}
                    1 & -2 & 1 \\
                    0 & 1 & -\frac{1}{3} \\
                    0 & 0 & 0 
                \end{array}
                \right)
                \sim
                \begin{aligned}
                    F_1 + 2F_2 \mapsto F_1
                \end{aligned}
                \left(
                \begin{array}{cc|c}
                    1 & 0 & \frac{1}{3} \\
                    0 & 1 & -\frac{1}{3} \\
                    0 & 0 & 0 
                \end{array}
                \right)
            \]
            \[
                \veccoord{u}{\fb'} = \fvec{\frac{1}{3}, -\frac{1}{3}}
            \]
    \end{enumerate}
