\item Determine \textbf{POR QUÉ} las siguientes afirmaciones son \textbf{VERDADERAS}
    \begin{enumerate}[label=\listAlph]
        \item Si el determinante de una matriz \(5 \times 5\) es 3, la matriz tiene máximo 3 columnas \(l.i\).
        \item Si \(\fset{u_1, u_2, u_3, u_4}\) es un conjunto de vectores \(l.d\)., \(\fset{u_1, u_2, u_3}\) también es un conjunto de vectores \(l.d\).
        \item Si \(\fset{u_1, u_2, u_3, u_4}\) es un conjunto \(l.i\)., \(\fset{u_1, u_2, u_3, u_4, u_5}\) también es un conjunto \(l.i\).
        \item La unión de dos subespacios vectoriales es un subespacio vectorial.
        \item Si \(\bgen{u_1, u_2, u_3, u_4} = \bgen{u_1, u_2, u_4}\), entonces \(\fset{u_1, u_3, u_4}\) es \(l.i\).
        \item El conjunto de puntos dentro de un circulo alrededor del origen de radio 1 es un subespacio de \(\realR^2\).
        \item El conjunto de matrices triangulares inferiores \(5 \times 5\) con unos en la diagonal es un subespacio de \(\fm_{2 \times 2}\).
        \item El conjunto de matrices elementales \(10 \times 10\) es un subespacio de \(\fm_{10 \times 10}\).
        \item El conjunto de polinomios de grado menor o igual a 5, con coeficientes enteros, es un subespacio de \(\fp_5\).
        \item El conjunto de polinomios de grado igual a 3, es un subespacio de \(\fp_3\).
    \end{enumerate}
