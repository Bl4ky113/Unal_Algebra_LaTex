\item Determine \textbf{POR QUÉ} las siguientes afirmaciones son \textbf{VERDADERAS}
    \begin{enumerate}[label=\listAlph]
            \setcounter{enumii}{6}
        \item Si \(\veccoord{u}{\fb} \in \realR^5\) entonces \(\pdim{\pgen{\fb}} = 5\). \\
            Efectivamente, es verdadera por definición, ya que el número de componentes 
            de un vector coordenada es dictado por el número de escalares de la combinación lineal de 
            un vector respecto a una base. Por ende, si un vector coordenada tiene \(n\) o en este caso 5 
            componentes, la dimensión de la base o el número de elementos de esta debe ser \(n\) o para nuestro caso 5.
            \setcounter{enumii}{7}
        \item La dimensión del espacio de las matrices diagonales \(4 \times 4\) es 4.
            Completamente verdadero, ya que una base del espacio de las matrices tales que \(\fset{a_{ij} = 0: i \neq j}\)
            es la base 
            \[
                \fb = \fset{
                    \fmatrix{1,0,0,0;0,0,0,0;0,0,0,0;0,0,0,0},
                    \fmatrix{0,0,0,0;0,1,0,0;0,0,0,0;0,0,0,0},
                    \fmatrix{0,0,0,0;0,0,0,0;0,0,1,0;0,0,0,0},
                    \fmatrix{0,0,0,0;0,0,0,0;0,0,0,0;0,0,0,1}
                }
            \]
            Esta es base se puede ver casi inmediatamente que es la base canónica del espacio vectorial de las matrices diagonales \(4 \times 4\).
            Ya que es \(l.i\) y puede generar cualquier cualquier matriz diagonal \(4 \times 4\), con los escalares iguales a cada componente,
            es decir
            \[
                \fmatrix{a,0,0,0;0,b,0,0;0,0,c,0;0,0,0,d} = 
                \lambda_1 
                \fmatrix{1,0,0,0;0,0,0,0;0,0,0,0;0,0,0,0}
                +
                \lambda_2
                \fmatrix{0,0,0,0;0,1,0,0;0,0,0,0;0,0,0,0}
                +
                \lambda_3
                \fmatrix{0,0,0,0;0,0,0,0;0,0,1,0;0,0,0,0}
                +
                \lambda_4
                \fmatrix{0,0,0,0;0,0,0,0;0,0,0,0;0,0,0,1}
            \]
            Con \(\lambda_1 = a, \lambda_2 = b, \lambda_3 = c, \lambda_4 = d\).
            Entonces, al tener 4 elementos esta base, todas las demás también han de tener 4 elementos, y por ende, la dimensión del espacio es de 4.
        \item La dimensión de un hiperplano en \(\realR^5\) es 4. \\
            Verdadero ya que un hiperplano es definido apartir de \(n - 1\) vectores ortogonales a un
            vector normal, entonces para un hiperplano de \(\realR^5\) al ser un espacio vectorial, debe contar 
            con una base y poder generar cualquier vector ortogonal al vector normal, por lo que se necesitan 4 vectores; 
            es decir. Se tiene que, los hiperplanos en \(\realR^n\), o en este caso \(\realR^5\), tienen una dimensión de \(n - 1\) o \(4\) para este caso.
        \item La dimensión de \(\bgen{u_1,u_2,u_3,u_4}\), cuando \(\fset{u_1,u_2,u_3,u_4}\) es \(l.i\), es 4. \\
            Verdadero; Como \(\fset{u_1,u_2,u_3,u_4}\) es \(l.i\) vamos a tener que verificar que el espacio vectorial dado es de 4. 
            Cabe aclarar que se sabe por la proposición de que \(\bgen{u_1,u_2,u_3,u_4}\) es un espacio vectorial ya que 
            nos preguntan por la dimensión de esté, lo cual si el conjunto generado no fuera un espacio vectorial entonces esté no contaria con dimensión.
            Ahora, como el conjunto generado por \(\bgen{u_1,u_2,u_3,u_4}\) es un espacio vectorial y \(\fset{u_1,u_2,u_3,u_4}\) es \(l.i\), vamos a 
            tener que \(\fset{u_1,u_2,u_3,u_4}\) es una base de este espacio vectorial. Con lo cual al tener 4 elementos esta base, vamos a tener que
            el espacio vectorial tiene una dimensión de 4.
            \setcounter{enumii}{14}
        \item Una matriz \(A\) de tamaño \(4 \times 7\) no puede tener una nulidad igual a cero. \\
            Esta afirmación es verdadera. Ya que partiendo del teorema de la nulidad, se tiene que \(\rho(A) + \nu(A) = 7\), 
            además a lo sumo esta matriz posee 4 variables pivotales en cada fila. Entonces podemos considerar que en el caso que se 
            tengan estas 4 variables pivotales, es decir \(\rho(A) = 4\), vamos a tener que \(\nu(A) = 3 \neq 0\).
    \end{enumerate}
