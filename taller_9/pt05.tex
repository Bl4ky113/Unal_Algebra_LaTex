\item De las siguientes afirmaciones, señale \textbf{DOS VERDADERAS}.
    \begin{enumerate}[label=\listAlph]
        \item Las coordenadas de un vector de un plano en \(\realR^5\), en una base del plano, es un vector de \(\realR^2\).
        \item Las coordenadas de un vector de un plano en \(\realR^5\), en una base de \(\realR^5\), es un vector de \(\realR^2\).
        \item Un conjunto de 5 vectores de un hiperplano en \(\realR^5\) que pasa por el origen puede ser \(l.i\).
        \item Si \(\nu(A) = 0\) y \(A\) es una matriz \(7 \times 4\), el sistema de ecuaciones lineales \(Ax = b\) tiene solución única para todo vector \(b\).
        \item Si \(\rho(A) = 5\) y \(A\) es una matriz \(5 \times 9\), el sistema de ecuaciones lineales \(Ax = b\) tiene infinitas soluciones para todo vector \(b\).
    \end{enumerate}
