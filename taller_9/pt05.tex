\item De las siguientes afirmaciones, señale \textbf{DOS VERDADERAS}.
    \begin{enumerate}[label=\listAlph]
        \item Las coordenadas de un vector de un plano en \(\realR^5\), en una base del plano, es un vector de \(\realR^2\). \\
            Esta afirmación es verdadera. Ya que se sabe que los planos tienen una dimensión de 2 y son un subespacio de \(\realR^n\), 
            entonces cualquier representación de un vector en el plano como vector coordenada respecto a alguna base siempre va a tener 
            2 componentes o ser un elemento de \(\realR^2\).
        %\item Las coordenadas de un vector de un plano en \(\realR^5\), en una base de \(\realR^5\), es un vector de \(\realR^2\).
        %\item Un conjunto de 5 vectores de un hiperplano en \(\realR^5\) que pasa por el origen puede ser \(l.i\).
        %\item Si \(\nu(A) = 0\) y \(A\) es una matriz \(7 \times 4\), el sistema de ecuaciones lineales \(Ax = b\) tiene solución única para todo vector \(b\).
        \setcounter{enumii}{4}
        \item Si \(\rho(A) = 5\) y \(A\) es una matriz \(5 \times 9\), el sistema de ecuaciones lineales \(Ax = b\) tiene infinitas soluciones para todo vector \(b\). \\
            Esta afirmación es verdadera. Dado que como \(\rho(A) = 5\) o que se tienen 5 variables, para la matrix \(5 \times 9\), no se va a contar con un pivote 
            para cada columna de la matriz. Lo cual es condición necesaria para que tenga una solución única. Además sabemos que este sistema si tiene solución ya que 
            su nulidad es \(\nu(A) = 4\), en lugar de \(\nu(A) = 0\), por el teorema de rango-nulidad.
    \end{enumerate}
