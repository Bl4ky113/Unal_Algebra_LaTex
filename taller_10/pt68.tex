\item Determine \textbf{POR QUÉ} las siguientes afirmaciones son \textbf{FALSAS}.
    \begin{enumerate}[label=\listAlph]
        \item Si el rango de una matriz \(A \in \fm_{7 \times 9}\) es 7, su nulidad es cero. \\
            Recordando el teorema de la relación entre la nulidad y rango de una matriz, podemos ver que 
            \[
                \rho(A) + \nu(A) = n \Rightarrow 7 + \nu(A) = 9
            \]
            Donde se puede ver que \(\nu = 2 \neq 0\), es decir, la matriz escalonada equivalente a \(A\) cuenta con 2 variables libres y 7 pivotes.
            Concluyendo que la nulidad de \(A\) no es cero.
        \item Si el rango de una matriz \(A \in \fm_{17 \times 9}\) es 9, el sistema \(Ax = b\) tiene solución única. \\
            Inicialmente, debemos ver que como \(A\) es una matriz con 9 columnas y rango de 9 vamos a tener 9 pivotes en la matriz escalonada equivalente a \(A\), 
            la cual se notara como \(U\).
            Entonces, las primeras 9 filas de \(U\) van a contar con pivotes en sus respectivas columnas, 
            y desde la 10\tsup{ma} fila de \(U\) vamos a contar con filas de ceros. Como cualquier valor real al ser multiplicado por cero es igual a cero puede 
            que exista un vector \(b\) tal que 
            \[
                b = \fvec{b_1, b_2, \vdots, b_9, b_{10}, \vdots, b_{17}}
                \hspace{1cm}
                \text{Donde existe } b_{i} \neq 0 \text{ para } i = 10, 11, \ldots, 17
            \]
            Entonces para el sistema \(Ax = b\) se tendría la matriz del sistema
            \[
                \left(
                \begin{array}{cccc|c}
                    a_{11} & a_{12} & \cdots & a_{19} & b_1 \\
                    0 & a_{22} & \cdots & a_{29} & b_2 \\
                    \vdots & \vdots & \ddots & \vdots & \vdots \\
                    0 & 0 & \cdots & a_{99} & b_9 \\
                    0 & 0 & \cdots & 0 & b_{10} \\
                    0 & 0 & \cdots & 0 & b_{11} \\
                    \vdots & \vdots & \ddots & \vdots & \vdots \\
                    0 & 0 & \cdots & 0 & b_{17} \\
                \end{array}
                \right)
            \]
            Como existe en \(b\) un \(b_i\) tal que sea diferente de 0 se tiene que 
            \(0x_1 + 0x_2 + 0x_3 + \cdots + 0x_9 = b_i \neq 0\) lo cual hace que el sistema sea inconsistente, es decir  el sistema \(Ax = b\) no tendría solución.
        %\item La dimensión del espacio de los polinomios de grado menor o igual a 4, que evaluados en 1 es 0, es 3.
        \setcounter{enumii}{3}
        \item La dimensión de \(\bgen{u_1, u_2, u_3, u_4}\) es 4. \\
            Inicialmente, debemos tener en cuenta que \(u_1, u_2, u_3, u_4\) pertenecen a un espacio vectorial \(V\) y además que por la falta 
            de información puede que el conjunto de vectores dados sea o no sea \(l.i\) y además que \(\bgen{u_1, u_2, u_3, u_4}\) puede que genere o no a \(V\),
            por ende, no podriamos directamente sobre la dimensión del espacio vectorial \(V\).
            Ahora, podemos ver cada caso descrito para \(u_1, u_2, u_3, u_4\) y \(\bgen{u_1, u_2, u_3, u_4}\), para así poder ver que en algunos casos su dimensión va a ser diferente de 4.
            \ResetCases{}
            \begin{mathcase}{\(\fset{u_1, u_2, u_3, u_4}\) es \(l.i\) y \(\bgen{u_1, u_2, u_3, u_4} = V\)}
                Como \(\fset{u_1, u_2, u_3, u_4}\) es \(l.i\) y genera a \(V\), va a ser una base de \(V\), entonces podemos afirmar que la dimensión de \(V\) sí es 4, 
                sin embargo, vamos a ver que esto no se tiene para todo caso.
            \end{mathcase}
            \begin{mathcase}{\(\fset{u_1, u_2, u_3, u_4}\) es \(l.d\) y \(\bgen{u_1, u_2, u_3, u_4} = V\)}
                Como \(\fset{u_1, u_2, u_3, u_4}\) es \(l.d\) y genera a \(V\), entonces va a existir un subconjunto de \(\fset{u_1, u_2, u_3, u_4}\) tal que 
                este sea \(l.i\) y genere a \(V\). Pero esto indica que la dimensión de \(V\) va a tener que ser menor que \(4\), y por ende diferente de 4.
            \end{mathcase}
            \begin{mathcase}{\(\fset{u_1, u_2, u_3, u_4}\) es \(l.i\) y \(\bgen{u_1, u_2, u_3, u_4} \neq V\)}
                Apartir de que \(\fset{u_1, u_2, u_3, u_4}\) es \(l.i\) y no genera a \(V\) podemos tomar uno o varios vectores de \(V\) 
                tal que no sean combinación lineal de \(u_1, u_2, u_3, u_4\) así que para el generador de estos vectores con \(u_1, u_2, u_3, u_4\) pueda 
                generar al espacio vectorial \(V\). Pero, esto también implica que la dimensión de \(V\) va a tener que ser mayor que 4, y por ende diferente de 4.
                \\
                Sin embargo, puede que exista un subespacio vectorial de \(V\), que denotaremos \(S\), al cual \(u_1, u_2, u_3, u_4\) pertenecen y al ser estos \(l.i\)
                van a generar a \(S\). Entonces la dimensión de \(S\) sí sería 4. 
                Pero en el caso que alguno de los vectores \(u_1, u_2, u_3, u_4\) no este en \(S\), \(\bgen{u_1, u_2, u_3, u_4}\) no generaría a \(S\) y por ende su dimensión sería diferente a 4.
            \end{mathcase}
            \begin{mathcase}{\(\fset{u_1, u_2, u_3, u_4}\) es \(l.d\) y \(\bgen{u_1, u_2, u_3, u_4} \neq V\)}
                En este caso no se puede asegurar que la dimensión de \(V\) sea diferente de 4. 
                Ya que \(\fset{u_1, u_2, u_3, u_4}\) al ser \(l.d\) y que \(\bgen{u_1, u_2, u_3, u_4}\) no genere a \(V\) no nos brinda la suficiente información sobre \(V\). \\
                Además de que podemos remover los vectores que son combinaciones lineales de \(\fset{u_1, u_2, u_3, u_4}\) 
                así para obtener un conjunto \(l.i\) con \(m\) elementos, el cual tampoco va a generar a \(V\) ya que al eliminar los elementos que son combinación lineal no cambia su conjunto generado.
                Pero apartir de esto, para que este conjunto \(l.i\) si genere a \(V\), se puede agregar \(n\) vectores que no sean combinación lineal de tal forma que 
                el conjunto resultante sí genere a \(V\), pero al ser arbitrario el número de vectores en el conjunto \(l.i\) obtenido apartir de \(\fset{u_1, u_2, u_3, u_4}\), \(m\),
                y el número de elementos que se deben agregar a este conjunto para que esté genere a \(V\), \(n\), puede que exista el caso que \(m + n = 4\)
                y por ende la dimensión de \(V\) puede que sea 4. Sin embargo, en general se va a tener que la dimensión de este espacio vectorial va a ser de \(m + n\).
            \end{mathcase}
        %\item Las coordenadas de una matriz \(3 \times 5\) en una base de \(\fm_{3 \times 5}\) es un vector de \(\realR^8\). 
        %\item Las coordenadas de un vector de un hiperplano en \(\realR^5\) en una base de \(\realR^5\) es un vector de \(\realR^4\).
        \setcounter{enumii}{6}
        \item Un conjunto de 5 matrices \(3 \times 2\) puede generar a \(\fm_{3 \times 2}\). \\
            Esta afirmación se puede notar que es falsa al considerar la dimensión del espacio \(\fm_{3 \times 2}\), ya que su base canónica cuenta con 6 elementos y 
            por ende la dimensión del espacio vectorial \(\fm_{3 \times 2}\) es 6. Entonces, por definición, se necesita como mínimo un conjunto de 6 matrices para 
            poder generar a \(\fm_{3 \times 2}\). Para asegurarnos más de que la dimensión de \(\fm_{3 \times 2}\) es 6, veamos la base canónica \(\fb\)
            \[
                \fb = \fset{
                    \fmatrix{1,0,0;0,0,0},
                    \fmatrix{0,1,0;0,0,0},
                    \fmatrix{0,0,1;0,0,0},
                    \fmatrix{0,0,0;1,0,0},
                    \fmatrix{0,0,0;0,1,0},
                    \fmatrix{0,0,0;0,0,1},
                }
            \]
            Donde \(\fb\) es \(l.i\) ya que al tomar la matriz de los coeficientes de los escalares de las combinaciones lineales de \(\fb\) vamos a tener una matriz 
            escalonada \(6 \times 6\) con 6 pivotes.
            \[
                \lambda_1 
                \fmatrix{1,0,0;0,0,0}
                +
                \lambda_2 
                \fmatrix{0,1,0;0,0,0}
                +
                \lambda_3
                \fmatrix{0,0,1;0,0,0}
                +
                \lambda_4 
                \fmatrix{0,0,0;1,0,0}
                +
                \lambda_5 
                \fmatrix{0,0,0;0,1,0}
                +
                \lambda_6 
                \fmatrix{0,0,0;0,0,1}
            \]
            \[
                \fmatrix{\lambda_1,0,0;0,0,0}
                +
                \fmatrix{0,\lambda_2,0;0,0,0}
                +
                \fmatrix{0,0,\lambda_3;0,0,0}
                +
                \fmatrix{0,0,0;\lambda_4,0,0}
                +
                \fmatrix{0,0,0;0,\lambda_5,0}
                +
                \fmatrix{0,0,0;0,0,\lambda_6}
            \]
            \[
                \fmatrix{
                    \lambda_1,
                    \lambda_2,
                    \lambda_3;
                    \lambda_4,
                    \lambda_5,
                    \lambda_6
                }
                \hspace{1cm}
                \fmatrix{
                    1,0,0,0,0,0;
                    0,1,0,0,0,0;
                    0,0,1,0,0,0;
                    0,0,0,1,0,0;
                    0,0,0,0,1,0;
                    0,0,0,0,0,1;
                }
            \]
            Además, podemos ver que \(\fb\) genera a \(\fm_{3 \times 2}\) ya que si tomamos cualquier matriz de \(\pgen{\fb}\) va a 
            estar contenida en \(\fm_{3 \times 2}\), y si tomamos cualquier matriz \(A\) de \(\fm_{3 \times 2}\) la podemos expresar como combinación lineal de \(\fb\).
            \[
                A 
                = \fmatrix{a,b,c;d,e,f} 
                = 
                \fmatrix{a,0,0;0,0,0}
                +
                \fmatrix{0,b,0;0,0,0}
                +
                \fmatrix{0,0,c;0,0,0}
                +
                \fmatrix{0,0,0;d,0,0}
                +
                \fmatrix{0,0,0;0,e,0}
                +
                \fmatrix{0,0,0;0,0,f}
            \]
            \[
                a
                \fmatrix{1,0,0;0,0,0}
                +
                b
                \fmatrix{0,1,0;0,0,0}
                +
                c
                \fmatrix{0,0,1;0,0,0}
                +
                d
                \fmatrix{0,0,0;1,0,0}
                +
                e
                \fmatrix{0,0,0;0,1,0}
                +
                f
                \fmatrix{0,0,0;0,0,1}
                \in \pgen{\fb}
            \]
            Concluyendo que \(\fb\) es una base de \(\fm_{3 \times 2}\) y que la dimensión de \(\fm_{3 \times 2}\) es 6.
            %\item Cualquier conjunto de 5 polinomios de grado menor o igual 3 genera a \(\fp_3\).
        %\item Un conjunto de 5 vectores de un hiperplano en \(\realR^5\) que pasa por el origen puede ser \(l.i\).
        \setcounter{enumii}{9}
        \item Cualquier conjunto de 5 matrices diagonales \(6 \times 6\) es \(l.i\). \\
            Primero, recordemos la definición de la matriz \(6 \times 6\) diagonal. Donde una matriz \(A\), definida como \(A = \left(a_{ij}\right)\), 
            \(A\) va a ser una matriz diagonal si y solo si \(a_{ij} = 0\) si \(i \neq j\). 
            Ahora, bajo esta definición podemos ver que la matriz \(6 \times 6\) nula es una matriz diagonal. 
            Y por definición si el elemento nulo esta en un conjunto de 5 elementos, esté es \(l.d\) ya que existe un escalar diferente a 0 tal que. 
            \[
                \lambda_1v_1 \lambda_2v_2 + \lambda_3v_3 + \lambda_4v_4 + \lambda_5v_5 = 0
            \]
            Entonces, al considerar cualquier conjunto, ya sea de 5, menos o más elementos, con la matriz nula el conjunto siempre va a ser \(l.d\).
        %\item Si \(S = \fset{p, q, r, t} \subset \fp_3\), \(S\) puede ser un conjunto ortogonal.
        %\item Para calcular la proyección ortogonal de un vector en un subespacio, se requiere una base ortogonal del subespacio.
        %\item La proyección ortogonal de un vector sobre un subespacio es ortogonal al subespacio.
        %\item Si \(\eta(A) = 0\) y \(A\) es una matriz \(7 \times 4\), el sistema de ecuaciones lineales \(Ax = b\) tiene solución única para todo vector b.
        \setcounter{enumii}{14}
        \item Si el rango de una matriz \(5 \times 8\) es 5, la nulidad de su transpuesta es 3. \\ 
            Partiendo desde el teorema de la relación entre el rango de una matriz y el de su transpuesta, tenemos que \(\rho\left(A\right) = \rho\left(A^T\right)\),
            vamos a obtener que \(\rho(A) = 5 = \rho\left(A^T\right)\), entonces ahora por el teorema de la relación entre la nulidad y rango de una matriz además tenemos que 
            \[  
                \begin{aligned}
                    \rho\left(A^T\right) + \nu\left(A^T\right) &= 5 \\
                    5 + \nu\left(A^T\right) &= 5 \\
                    \nu\left(A^T\right) &= 0 \\
                \end{aligned}
            \]
            Confirmando que la nulidad de \(A^T\) es 0 en lugar de 3 como se había propuesto.
        %\item Una base del espacio columna de una matriz es la conformada por las columnas pivotales de una matriz escalonada equivalente.
    \end{enumerate}
