\item Sea \(W = \fset{\fsmvec{x,y,z,w}: x - 2y + z - w  = 0}\) un subespacio de \(\realR^4\). Encuentre una base ortonormal de \(W\).
    Exprese \(v = \fsmvec{1,1,2,1} \in W\) como combinación lineal de la base ortonormal hallada. \\
    Primero, para obtener una base ortonormal de \(W\) debemos partir de una base, \(\fb\), de \(W\). De lo cual podemos partir que la dimensión de un hiperplano en \(\realR^n\) es de \(n - 1\).
    Entonces para este plano en \(\realR^4\) vamos a tener una dimensión de 3. Ahora, vamos a tomar la ecuación normal del hiperplano y despejar a \(w\). 
    Para así poder encontrar los coeficientes de las variables restantes en cada componente del punto en el hiperplano, resultando en una base del hiperplano.
    \[
        \begin{aligned}
            x - 2y + z - w &= 0 \\
            x - 2y + z &= w \\
        \end{aligned}
        \hspace{0.25cm}
        \Rightarrow
        \hspace{0.25cm}
        \fset{\fvec{x,y,z,x -2y + z}}
        \hspace{0.25cm}
        \Rightarrow
        \hspace{0.25cm}
        \fset{
            x
            \fvec{1,0,0,1}
            +
            y
            \fvec{0,1,0,-2}
            +
            z
            \fvec{0,0,1,1}
        }
    \]
    \[
        \fb = \fset{
            \fvec{1,0,0,1},
            \fvec{0,1,0,-2},
            \fvec{0,0,1,1}
        } 
    \]
    Sin embargo verifiquemos que \(\fb\) si sea una base de \(W\). Primero, veamos que \(\fb\) es \(l.i\) apartir del 
    sistema de ecuaciones lineales de las componentes de \(\fb\) y su matriz de coeficientes, para luego escalonar esta matriz 
    y verificar que cuente con 3 pivotes para que, por el teorema de equivalencias, sus columnas sean \(l.i\).
    \[
        \begin{cases}
            \fb_1 = x \\
            \fb_2 = y \\
            \fb_3 = z \\
            \fb_4 = x - 2y + z
        \end{cases}
        \hspace{0.25cm}
        \Rightarrow
        \hspace{0.25cm}
        \fmatrix{
            1, 0, 0;
            0, 1, 0;
            0, 0, 1;
            1, -2, 1
        }
        \sim
        \begin{aligned}
            F_4 - F_1 &\mapsto F_4 \\
            F_4 + 2F_2 &\mapsto F_4 \\
            F_4 - F_3 &\mapsto F_4
        \end{aligned}
        \fmatrix{
            1, 0, 0;
            0, 1, 0;
            0, 0, 1;
            0, 0, 0
        }
    \]
    Ahora verifiquemos que \(\pgen{\fb} = W\) mediante una doble contenencia. 
    Sea \(u'\) un elemento cualquiera en \(\pgen{\fb}\), entonces 
    \(
        u'
        = 
        \lambda_1 
        \fsmvec{1,0,0,1}
        +
        \lambda_2
        \fsmvec{0,1,0,-2}
        +
        \lambda_3
        \fsmvec{0,0,1,1}
        =
        \fsmvec{
            \lambda_1,
            \lambda_2,
            \lambda_3,
            \lambda_1 - 2\lambda_2 + \lambda_3
        }
    \)
    si reemplazamos cada componente en la ecuación normal de \(W\) tendremos
    \(
        \lambda_1 -2(\lambda_2) + \lambda_3 - (\lambda_1 -2\lambda_2 + \lambda_3) 
        =
        \lambda_1 - \lambda_1 -2\lambda_2 +2\lambda_2 + \lambda_3 - \lambda_3
        =
        0
    \)
    concluyendo que \(u'\) pertenece a \(W\).
    Ahora veamos que \(v'\) un punto cualquiera en el hiperplano \(W\) es contenido por \(\pgen{\fb}\), entonces podemos ver que
    \(v' = \fsmvec{x,y,z,x -2y + z} = x \fsmvec{1,0,0,1} + y\fsmvec{0,1,0,-2} + z\fsmvec{0,0,1,1}\), como \(x, y, z\) son 
    escalares reales entonces \(v'\) es una combinación lineal de \(\fb\), es decir, \(\pgen{\fb}\) contiene a \(v'\).
    Y podemos afirmar que \(\pgen{\fb} = W\).
    \\
    Ahora con esta base \(\fb\) de \(W\) vamos a usar el proceso de ortogonalización de \emph{Gram-Schmidt}, donde 
    vamos a tener nuestros vectores 
    \(
        v_1 = \fsmvec{1,0,0,1}, 
        v_2 = \fsmvec{0,1,0,-2},
        v_3 = \fsmvec{0,0,1,1}
    \)
    y los vamos a usar para formar el conjunto de vectores ortogonales \(\fset{u_1, u_2, u_3}\) y apartir de estos, normalizarlos para 
    obtener al conjunto de los vectores ortonormales \(\fset{w_1, w_2, w_3}\).
    Primero, vamos a definir a \(u_1 = v_1\) ya que almenos necesitamos 2 vectores para determinar que sean ortogonales entre sí,
    apartir de este vector vamos a definir a \(u_2\) y \(u_3\) como \(u_2 = v_2 - \fproy{u_1}{v_2}\) y \(u_3 = v_3 - \fproy{u_2}{v_3} - \fproy{u_1}{v_3}\) respectivamente.
    \[
        \begin{aligned}
            u_2 &= v_2 - \fproy{u_1}{v_2} 
                = v_2 - \frac{v_2 \cdot u_1}{u_1 \cdot u_1}u_1 \\
            &= \fvec{0,1,0,-2} + \frac{2}{2}\fvec{1,0,0,1} \\
            &= \fvec{0,1,0,-2} + \fvec{1,0,0,1} \\
            &= \fvec{1,1,0,-1} \\
        \end{aligned}
        \hspace{1cm}
        \begin{aligned}
            u_3 &= v_3 - \fproy{u_2}{v_3} - \fproy{u_3}{v_3}
                = v_3 -\frac{v_3 \cdot u_2}{u_2 \cdot u_2}u_2 - \frac{v_3 \cdot u_1}{u_1 \cdot u_1}u_1 \\
            &= \fvec{0,0,1,1} +\frac{1}{3}\fvec{1,1,0,-1} -\frac{1}{2}\fvec{1,0,0,1}\\
            &= \fvec{0,0,1,1} +\fvec{\frac{1}{3},\frac{1}{3},0,-\frac{1}{3}} -\fvec{\frac{1}{2},0,0,\frac{1}{2}}
                = \fvec{\frac{1}{3},\frac{1}{3},1,\frac{2}{3}} - \fvec{\frac{1}{2},0,0,\frac{1}{2}} \\
            &= \fvec{-\frac{1}{6},\frac{1}{3},1,\frac{1}{6}}
        \end{aligned}
    \]
    Ahora, con \(u_1, u_2, u_3\) vamos a crear a \(w_1, w_2, w_3\) simplemente conociendo que cada vector \(u_i\) con \(i = 1, 2, 3\) es diferente del vector nulo
    vamos a tener que \(\frac{u_i}{\fnorm{u_i}}\) va a ser un vector unitario. 
    Esto ya que, si tomamos cualquier vector \(a\) de \(\realR^n\), sabemos que va a ser unitario si y solo si \(\fnorm{a} = 1\), 
    además, la norma de un vector esta definida como \(\fnorm{a} = \sqrt{a \cdot a} = \sqrt{a_1^2 + a_2^2 + \cdots a_n^2}\). 
    Entonces si mutliplicamos este vector por el inverso multiplicativo de la norma de sí mismo vamos a tener que
    \[
        \frac{1}{\fnorm{a}}a = 
        \fvec{
            \frac{a_1}{\sqrt{a_1^2 + \cdots + a_n^2}},
            \vdots,
            \frac{a_n}{\sqrt{a_1^2 + \cdots + a_n^2}}
        }
    \]
    \[
        \begin{aligned}
            \fnorm{\frac{1}{\fnorm{a}}a} 
            &= 
            \sqrt{
                \left(
                    \frac{a_1}{\sqrt{a_1^2 + \cdots + a_n^2}}
                \right){}^2
                +
                \cdots
                +
                \left(
                    \frac{a_n}{\sqrt{a_1^2 + \cdots + a_n^2}}
                \right){}^2
            }\\
            &=
            \sqrt{
                \frac{a_1^2}{a_1^2 + \cdots + a_n^2}
                +
                \cdots
                +
                \frac{a_n^2}{a_1^2 + \cdots + a_n^2}
            }\\
            &=
            \sqrt{\frac{a_1^2 + \cdots + a_n^2}{a_1^2 + \cdots + a_n^2}}
                = \sqrt{1} 
                = 1
        \end{aligned}
    \]
    Es necesario que \(a\) sea diferente del vector nulo ya que su norma seria 0, por lo que su inverso multiplicativo no existiría.
    Bueno, retomando, primero calculemos la norma de cada vector \(u_i\), y después con esté obtendremos cada vector unitario.
    \[
        \begin{aligned}
            \fnorm{u_1} &= \sqrt{1 + 0 + 0 + 1} = \sqrt{2}
        \end{aligned}
        \hspace{0.25cm}
        \begin{aligned}
            \fnorm{u_2} &= \sqrt{1 + 1 + 0 + 1} = \sqrt{3}
        \end{aligned}
    \]
    \[
        \begin{aligned}
            \fnorm{u_3} &= 
                \sqrt{
                    \left(-\frac{1}{6}\right){}^2
                    +
                    \left(\frac{1}{3}\right){}^2
                    +
                    1
                    +
                    \left(\frac{1}{6}\right){}^2
                } 
            =
            \sqrt{
                \frac{1}{36}
                +
                \frac{1}{9}
                +
                1
                +
                \frac{1}{36}
            }
            = 
            \sqrt{
                \frac{42}{46}
            }
            =
            \frac{\sqrt{42}}{6}
        \end{aligned}
    \]
    \[
        \frac{1}{\fnorm{u_1}}u_1 
        = 
        \frac{1}{\sqrt{2}}
        \fvec{1,0,0,1}
        =
        \fvec{
            \frac{1}{\sqrt{2}},
            0,
            0,
            \frac{1}{\sqrt{2}}
        }
        =
        w_1
        \hspace{1cm}
        \frac{1}{\fnorm{u_2}}u_2
        =
        \frac{1}{\sqrt{3}}
        \fvec{1,1,0,-1}
        =
        \fvec{
            \frac{1}{\sqrt{3}},
            \frac{1}{\sqrt{3}},
            0,
            -\frac{1}{\sqrt{3}}
        }
        =
        w_2
    \]
    \[
        \frac{1}{\fnorm{u_3}}u_3
        =
        \frac{6}{\sqrt{42}}
        \fvec{
            -\frac{1}{6},
            \frac{1}{3},
            1,
            \frac{1}{6}
        }
        =
        \fvec{
            -\frac{1}{\sqrt{42}},
            \frac{2}{\sqrt{42}},
            \frac{6}{\sqrt{42}},
            \frac{1}{\sqrt{42}}
        }
        =
        w_3
    \]
    Así, finalmente, podemos concluir que \(\fb'\) va a ser una base ortonormal de \(W\) definida como
    \[
        \fb' = \fset{w_1, w_2, w_3} = 
        \fset{
            \fvec{
                \frac{1}{\sqrt{2}},
                0,
                0,
                \frac{1}{\sqrt{2}}
            },
            \fvec{
                \frac{1}{\sqrt{3}},
                \frac{1}{\sqrt{3}},
                0,
                -\frac{1}{\sqrt{3}}
            },
            \fvec{
                -\frac{1}{\sqrt{42}},
                \frac{2}{\sqrt{42}},
                \frac{6}{\sqrt{42}},
                \frac{1}{\sqrt{42}}
            }
        }
    \]
    Ahora solo nos falta expresar a \(v\) como una combinación lineal de \(\fb'\).
    Sin embargo, como \(\fb'\) es una base ortonormal por el teorema de ortonormalidad y combinaciones lineales 
    vamos a poder obtener cada escalar de la combinación lineal de \(v\) en \(\fb'\) de la siguiente forma \(\lambda_i = v \cdot w_i\) donde \(i = 1, 2, 3\),
    entonces vamos a tener que 
    \[
        \lambda_1 = 
            \fvec{1,1,2,1} 
            \cdot 
            \fvec{
                \frac{1}{\sqrt{2}},
                0,
                0,
                \frac{1}{\sqrt{2}}
            }
        =
        \frac{1}{\sqrt{2}} + \frac{1}{\sqrt{2}}
        =
        \frac{2}{\sqrt{2}}
        =
        \sqrt{2}
    \hspace{0.5cm}
    \lambda_2 = 
        \fvec{1,1,2,1}
        \cdot
        \fvec{
            \frac{1}{\sqrt{3}},
            \frac{1}{\sqrt{3}},
            0,
            -\frac{1}{\sqrt{3}}
        }
        =
            \frac{1}{\sqrt{3}}
            +
            \frac{1}{\sqrt{3}}
            -
            \frac{1}{\sqrt{3}}
        =
        \frac{1}{\sqrt{3}}
        =
        \frac{\sqrt{3}}{3}
    \]
    \[
        \lambda_3 
        =
        \fvec{1,1,2,1}
        \cdot
        \fvec{
            -\frac{1}{\sqrt{42}},
            \frac{2}{\sqrt{42}},
            \frac{6}{\sqrt{42}},
            \frac{1}{\sqrt{42}}
        }
        =
        -\frac{1}{\sqrt{42}}
        +
        \frac{2}{\sqrt{42}}
        +
        \frac{12}{\sqrt{42}}
        +
        \frac{1}{\sqrt{42}}
        =
        \frac{14}{\sqrt{42}}
        =
        \frac{\sqrt{42}}{3}
    \]
    Para finalmente concluir, vamos a verificar si estos escalares nos dan a \(v\) cuando hacemos la combinación lineal
    \[
        \sqrt{2}
        \fvec{
            \frac{1}{\sqrt{2}},
            0,
            0,
            \frac{1}{\sqrt{2}}
        }
        +
        \frac{\sqrt{3}}{3}
        \fvec{
            \frac{1}{\sqrt{3}},
            \frac{1}{\sqrt{3}},
            0,
            -\frac{1}{\sqrt{3}}
        }
        +
        \frac{\sqrt{42}}{3}
        \fvec{
            -\frac{1}{\sqrt{42}},
            \frac{2}{\sqrt{42}},
            \frac{6}{\sqrt{42}},
            \frac{1}{\sqrt{42}}
        }
        =
        \fvec{1,0,0,1}
        +
        \fvec{
            \frac{1}{3},
            \frac{1}{3},
            0
            -\frac{1}{3}
        }
        +
        \fvec{
            -\frac{1}{3},
            \frac{2}{3},
            2,
            \frac{1}{3}
        }
        =
        \fvec{1,1,2,1}
        =
        v
    \]
    Es decir \(v \in W\).
