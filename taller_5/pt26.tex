\item Demuestre que, siendo \(n \in \realN\)
    \[
        \begin{pmatrix}
            \lambda & 1 \\
            0 & \lambda
        \end{pmatrix} ^ {n}
        =
        \begin{pmatrix}
            \lambda^{n} & n\lambda^{n - 1} \\
            0 & \lambda^{n}
        \end{pmatrix}
    \]
    \begin{proof}
        Sea 
        \(
            X = 
            \left\{
                m \in \realN |
                \left(
                \begin{smallmatrix}
                    \lambda & 1 \\
                    0 & \lambda
                \end{smallmatrix} 
                \right){} ^ {m}
                =
                \left(
                \begin{smallmatrix}
                    \lambda^{m} & m\lambda^{m - 1} \\
                    0 & \lambda^{m}
                \end{smallmatrix}
                \right)
            \right\}
        \), como caso base, veamos que 1 pertenece a \(X\).
        \[
            \begin{pmatrix}
                \lambda & 1 \\
                0 & \lambda
            \end{pmatrix} ^ {1}
            =
            \begin{pmatrix}
                \lambda^1 & 1 \cdot \lambda^{0} \\
                0 & \lambda^1
            \end{pmatrix}
            =
            \begin{pmatrix}
                \lambda^1 & 1 \cdot \lambda^{1 - 1} \\
                0 & \lambda^1
            \end{pmatrix}
        \]
        Ahora, como hipótesis de inducción vamos a tener que \(n\), sea un número natural fijo pero arbitrario, tal que
        \(n\) pertenece a \(X\), es decir
        \[
            \begin{pmatrix}
                \lambda & 1 \\
                0 & \lambda
            \end{pmatrix} ^ {n}
            =
            \begin{pmatrix}
                \lambda^{n} & n\lambda^{n - 1} \\
                0 & \lambda^{n}
            \end{pmatrix}
        \]
        Ahora, veamos que \(n + 1\) también pertence a \(X\), o equivalentemente
        \[
            \begin{aligned}
                \begin{pmatrix}
                    \lambda & 1 \\
                    0 & \lambda
                \end{pmatrix} ^ {n + 1}
                &=
                \begin{pmatrix}
                    \lambda & 1 \\
                    0 & \lambda
                \end{pmatrix}
                \cdot
                \begin{pmatrix}
                    \lambda & 1 \\
                    0 & \lambda
                \end{pmatrix} ^ {n}
                =
                \begin{pmatrix}
                    \lambda & 1 \\
                    0 & \lambda
                \end{pmatrix}
                \cdot
                \begin{pmatrix}
                    \lambda^{n} & n\lambda^{n - 1} \\
                    0 & \lambda^{n}
                \end{pmatrix}
                \\
                &=
                \begin{pmatrix}
                    \lambda \cdot \lambda^{n} + 1 \cdot 0 & \lambda \cdot n\lambda^{n - 1} + 1 \cdot \lambda^{n} \\
                    0 \cdot \lambda^{n} + \lambda \cdot 0 & 0 \cdot n\lambda^{n - 1} + \lambda \cdot \lambda^{n}
                \end{pmatrix}
                =
                \begin{pmatrix}
                    \lambda^{n + 1} & n\lambda^{n} + \lambda^{n} \\
                    0 & \lambda^{n + 1}
                \end{pmatrix}
                \\
                &=
                \begin{pmatrix}
                    \lambda^{n + 1} & (n + 1)\lambda^{n} \\
                    0 & \lambda^{n + 1}
                \end{pmatrix}
            \end{aligned}
        \]
        Entonces tenemos que, por \emph{PIM-1}, \(X = \realN\). En conclusión, para cualquier matriz de orden 2 se va a tener para 
        cualquier número natural que 
        \[
            \begin{pmatrix}
                \lambda & 1 \\
                0 & \lambda
            \end{pmatrix} ^ {n}
            =
            \begin{pmatrix}
                \lambda^{n} & n\lambda^{n - 1} \\
                0 & \lambda^{n}
            \end{pmatrix}
        \]
    \end{proof}
