\begin{definition}
    Decimos que una matriz \(A\) cuadrada es ortogonal, si y solo si, las columnas de \(A\) son vectores unitarios y ortogonales entre si.
\end{definition}
\item Demuestre porque son \textbf{VERDADERAS} las siguientes proposiciones.
    \begin{enumerate}[label=\listAlph]
        \item Si \(A\) es una matriz ortogonal, entonces \(A^{T} \cdot A = I\).
        \item Toda matriz ortogonal es invertible. 
        \item Si \(A\) es una matriz ortogonal, entonces \(A \cdot A^{T} = I\).
        \item Si \(A\) es una matriz ortogonal, entonces \(A^{T}\) también es ortogonal.
        \item La matriz idéntica es ortogonal.
        \item La inversa de una matriz ortogonal es su transpuesta.
        \item Si \(A\) es una matriz ortogonal, la solución del sistema de ecuaciones lineales \(A \cdot x = b\) es \(x = A^{T} \cdot b\).
    \end{enumerate}
