\item Si \(p(t) = t^3 − 6t^2 − 5t − 12\) y 
    \(A = \left[\begin{smallmatrix}2 & 3 & -2 \\ 0 & 5 & 4 \\ 1 & 0 & -1\end{smallmatrix}\right]\), 
    se verifica que \(p(A) = A^3 − 6A^2 − 5A − 12I\) es igual a la matriz nula? \\
    Simplemente operemos las matrices y veamos si es verdadera esta proposición.
    \[
        p(A)=
        A^3 − 6(A^2) − 5A − 12I
        =
        \begin{bmatrix}
            2 & 3 & -2 \\ 
            0 & 5 & 4 \\ 
            1 & 0 & -1
        \end{bmatrix}^3
        -
        6
        \left(
        \begin{bmatrix}
            2 & 3 & -2 \\ 
            0 & 5 & 4 \\ 
            1 & 0 & -1
        \end{bmatrix}^2
        \right)
        -
        5
        \left(
        \begin{bmatrix}
            2 & 3 & -2 \\ 
            0 & 5 & 4 \\ 
            1 & 0 & -1
        \end{bmatrix}
        \right)
        -
        12
        \left(
        \begin{bmatrix}
            1 & 0 & 0 \\
            0 & 1 & 0 \\
            0 & 0 & 1
        \end{bmatrix}
        \right)
    \]
    Para un desarrollo más simple, se va a realizar por partes.
    \[
        \begin{aligned}
            A^2 &= 
            \begin{bmatrix}
                2 & 3 & -2 \\ 
                0 & 5 & 4 \\ 
                1 & 0 & -1
            \end{bmatrix}
            \cdot
            \begin{bmatrix}
                2 & 3 & -2 \\ 
                0 & 5 & 4 \\ 
                1 & 0 & -1
            \end{bmatrix}
            =
            \begin{bmatrix}
                2 & 21 & 10 \\ 
                4 & 25 & 16 \\
                1 & 3 & -1
            \end{bmatrix};
            \hspace{0.5cm}
            -6 \left(A^2\right) =
            \begin{bmatrix}
                -12 & -126 & -60 \\ 
                -24 & -150 & -96 \\
                -6 & -18 & 6
            \end{bmatrix};
            \\
            A^3 &= A \cdot A^2 = 
            \begin{bmatrix}
                2 & 3 & -2 \\ 
                0 & 5 & 4 \\ 
                1 & 0 & -1
            \end{bmatrix}
            \cdot
            \begin{bmatrix}
                2 & 21 & 10 \\ 
                4 & 25 & 16 \\
                1 & 3 & -1
            \end{bmatrix}
            = 
            \begin{bmatrix}
                14 & 111 & 70 \\
                24 & 137 & 76 \\
                1 & 18 & 11
            \end{bmatrix};
            \\
            -5A &= 
            -5
            \begin{bmatrix}
                2 & 3 & -2 \\ 
                0 & 5 & 4 \\ 
                1 & 0 & -1
            \end{bmatrix}
            =
            \begin{bmatrix}
                -10 & -15 & 10 \\ 
                0 & -25 & -20 \\ 
                -5 & 0 & 5
            \end{bmatrix};
            \hspace{0.5cm}
            -12I =
            \begin{bmatrix}
                -12 & 0 & 0 \\
                0 & -12 & 0 \\
                0 & 0 & -12
            \end{bmatrix};
        \end{aligned}
    \]
    Ahora que tenemos cada matriz podemos simplemente sumarlas.
    \[
        \begin{aligned}
            p(A) &=
            \begin{bmatrix}
                14 & 111 & 70 \\
                24 & 137 & 76 \\
                1 & 18 & 11
            \end{bmatrix}
            +
            \begin{bmatrix}
                -12 & -126 & -60 \\ 
                -24 & -150 & -96 \\
                -6 & -18 & 6
            \end{bmatrix}
            +
            \begin{bmatrix}
                -10 & -15 & 10 \\ 
                0 & -25 & -20 \\ 
                -5 & 0 & 5
            \end{bmatrix}
            +
            \begin{bmatrix}
                -12 & 0 & 0 \\
                0 & -12 & 0 \\
                0 & 0 & -12
            \end{bmatrix} 
            \\
            &=
            \begin{bmatrix}
                2 & -15 & 10 \\
                0 & -13 & -20 \\
                -5 & 0 & 17
            \end{bmatrix}
            +
            \begin{bmatrix}
                -22 & -15 & 10 \\ 
                0 & -37 & -20 \\ 
                -5 & 0 & -7
            \end{bmatrix}
            \\
            &=
            \begin{bmatrix}
                -20 & -30 & 20 \\
                0 & -50 & -40 \\
                -10 & 0 & 10
            \end{bmatrix}
        \end{aligned}
    \]
    Como se puede ver, al operar cada matriz, el resultado de \(p(A)\) es diferente a la matriz nula.
    
