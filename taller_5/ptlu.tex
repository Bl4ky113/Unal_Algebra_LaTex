\item Explicar en qué consiste la factorización \(LU\) de una matríz y hacer un ejemplo. \\
    Antes de iniciar, para poder entender correctamente la factorización \(LU\) de una matriz es necesario entender los conceptos de 
    pre-multiplicación, obtención y inversa de matrices elementales, y el método de eliminación de \emph{Gauss} para escalonar matrices.
    Estos temas, conceptos y secciones dedicadas estan disponibles en~\cite{martinez}. \\[0.25cm]
    Inicialmente la factorización en \(LU\) se considera una forma de obtener una matriz base apartir de la multipliación de una matriz \emph{Lower},
    \(L\) o triangular hacía abajo, a una matriz \emph{Upper}, \(U\) o triangular hacía arriba. O de forma más clara, pre-multipliación por \(L\) a \(U\).
    El punto importante en la factorización \(LU\) es como se pueden obtener las matrices \(L\) y \(U\), lo cual vamos a necesitar los conceptos previamente mencionados. 
    \\[0.25cm]
    Previamente, vamos a definir a la matriz base como \(A\) una matriz de orden \(n\), por ahora solo por convenencia, y que sea inversible.
    Primero, vamos a obtener a \(U\), realizando el método de eliminación de \emph{Gauss}, aunque también aplica para el método de eliminación \emph{Gauss-Jordan},
    sin embargo debemos tener en cuenta que solo debemos realizar operaciones de eliminación, y enumerarlas en orden según cada operación realizada, 
    para así facilitar el proceso de obtener a \(L\). Una vez se haya terminado de aplicar este método y tengamos una matriz escalonada equivalente a \(A\), 
    vamos a tomarla como nuestra matriz \(U\). Después, vamos a tomar en orden cada operación elemental realizada para obtener a \(U\) 
    y vamos a obtener cada matriz elemental inversa a estas operaciones, definiendolas como \(E_i\) con \(i\) siendo el número de la operación elemental,
    para luego multiplicar a cada una de estas matrices siguiendo el orden, obteniendo la matriz \(L\). 
    Este procedimiento se puede ver de una manera bastante sencilla con el uso de las matrices elementales y sus inversas,
    partiendo desde que la premultiplicación de \(A\) por todas las matrices elementales obtenidas va a ser igual a \(U\), 
    siendo \(k\) el número de operaciones y de matrices elementales
    \[
        \begin{aligned}
            E_k \cdot \cdots \cdot E_2 \cdot E_1 \cdot A &= U \\
            E^{-1}_1 \cdot E^{-1}_2 \cdot \cdots \cdot E^{-1}_k \cdot E_k \cdot \cdots \cdot E_2 \cdot E_1 \cdot A &= E^{-1}_1 \cdot E^{-1}_2 \cdot \cdots \cdot E^{-1}_k \cdot U \\
            I \cdot A = A &= E^{-1}_1 \cdot E^{-1}_2 \cdot \cdots \cdot E^{-1}_k \cdot U \\
            A &= L \cdot U \\
        \end{aligned}
    \]
    Apartir de este desarrollo podemos crear una definición de la factorización \(LU\).
    \begin{definition}
        Sea una matriz \(A\) de orden \(m \times n\), tal que sea \(U\) la matriz escalonada equivalente a \(A\) teniendo la 
        restricción que se haya obtenido únicamente realizando operaciones de eliminación en \(A\). Y \(L\), una matriz 
        triangular inferior de orden \(m \times m\), cual se obtiene de la multiplicación en orden de las matrices elementales inversas 
        que se obtienen de las operaciones elementales aplicadas en \(A\) para obtener a \(U\), cabe aclarar que el orden de la multipliación entre 
        las matrices es el mismo orden de las operaciones realizadas. \\
        Con lo cual se logra \emph{factorizar} o \emph{descomponer} a \(A\) de la siguiente manera
        \[
            A = LU
        \]
        Definida esta factorización como \emph{factorización \(LU\)}.
    \end{definition}
    Un detalle bastante importante en la definición, que aún no se ha mencionado, es que se puede realizar una factorización \(LU\) una matriz no cuadrada. 
    Y además, que podemos usar esta factorización \(LU\) para encontrar soluciones para un sistema de ecuaciones lineales cuya matriz de coeficientes sea equivalente a \(A\).
    Ya que, tomando a \(Ax = b\), es equivalente a \(LUx = b\) y desde ahí podemos definir las igualdades
    \[
        Ly = b
        \hspace{2cm}
        Ux = y
    \]
    Donde primero vamos a obtener a \(y\), y apartir de este, obtener a \(x\), la solución del sistema de ecuaciones lineales equivalente.
    \subsubsection*{Ejemplo factorización \(LU\)}
    Ejercicios tomados de la sección de problemas 1.1, punto 5 y 36 en~\cite{grossman}.
    Apatir de una matriz \(A\) vamos a obtener su factorización \(LU\), inicialmente obteniendo a \(U\) y después obteniendo a \(L\). \\
    Primero realizamos el método de eliminación de \emph{Gauss} en \(A\) para obtener a \(U\)
    \[
        A = 
        \begin{pmatrix}
            2 & 1 & 7 \\
            4 & 3 & 5 \\ 
            2 & 1 & 6 
        \end{pmatrix}
        \sim
        \begin{aligned}
            F_2 - 2F_1 &\mapsto F_2 \\
            F_3 - F_1 &\mapsto F_3
        \end{aligned}
        \begin{pmatrix}
            2 & 1 & 7 \\
            0 & 1 & -9 \\ 
            0 & 0 & -1 
        \end{pmatrix}
    \]
    Con las operaciones realizadas para obtener a \(U\), ahora obtenemos a \(E_1, E_2\) y sus respectivas inversas
    \[
        \begin{aligned}
            E_1 &:
            \begin{pmatrix}
                1 & 0 & 0 \\
                0 & 1 & 0 \\
                0 & 0 & 1
            \end{pmatrix}
            \sim
            \begin{aligned}
                F_2 - 2F_1 &\mapsto F_2 \\
            \end{aligned}
            \begin{pmatrix}
                1 & 0 & 0 \\
                -2 & 1 & 0 \\
                0 & 0 & 1
            \end{pmatrix} \\
            E_1^{-1} &= 
            \begin{pmatrix}
                1 & 0 & 0 \\
                2 & 1 & 0 \\
                0 & 0 & 1
            \end{pmatrix}
        \end{aligned}
        \hspace{0.5cm}
        \begin{aligned}
            E_2 &: 
            \begin{pmatrix}
                1 & 0 & 0 \\
                0 & 1 & 0 \\
                0 & 0 & 1
            \end{pmatrix}
            \sim
            \begin{aligned}
                F_3 - F_1 &\mapsto F_3 \\
            \end{aligned}
            \begin{pmatrix}
                1 & 0 & 0 \\
                0 & 1 & 0 \\
                -1 & 0 & 1
            \end{pmatrix} \\
            E_2^{-1} &=
            \begin{pmatrix}
                1 & 0 & 0 \\
                0 & 1 & 0 \\
                1 & 0 & 1
            \end{pmatrix}
        \end{aligned}
    \]
    Y multiplicamos a \(E_1^{-1}, E_2^{-2}\) para obtener a \(L\).
    \[
        E_1^{-1} \cdot E_2^{-1} =
        \begin{pmatrix}
            1 & 0 & 0 \\
            2 & 1 & 0 \\
            0 & 0 & 1
        \end{pmatrix}
        \cdot
        \begin{pmatrix}
            1 & 0 & 0 \\
            0 & 1 & 0 \\
            1 & 0 & 1
        \end{pmatrix}
        =
        \begin{pmatrix}
            1 & 0 & 0 \\
            2 & 1 & 0 \\
            1 & 0 & 1 
        \end{pmatrix}
    \]
    Obteniendo así, y verificando
    \[
        A = LU = 
        \begin{pmatrix}
            1 & 0 & 0 \\
            2 & 1 & 0 \\
            1 & 0 & 1 
        \end{pmatrix}
        \cdot 
        \begin{pmatrix}
            2 & 1 & 7 \\
            0 & 1 & -9 \\ 
            0 & 0 & -1 
        \end{pmatrix}
        =
        \begin{pmatrix}
            2 & 1 & 7 \\
            4 & 3 & 5 \\
            2 & 1 & 6
        \end{pmatrix}
    \]
    \\[0.5cm]
    Sin embargo, se debe aclarar que existe la factorización \(PLU\), la cual es bastante similar a la factorización \(LU\),
    pero con el añadido de agregar las operaciones elementales de permutación a su proceso. 
    Donde, apartir de una matriz \(A\) de orden \(m \times n\), inicialmente al realizar el método de eliminación de \emph{Gauss}
    se tienen que hacer una o más operaciones elementales de permutación se puede obtener la matriz triangular hacía abajo \(P\). 
    Ahora, Se obtienen las matrices elementales de estas permutaciones, denotadas como \(D_i\) con \(i\) siendo el número de la operación elemental, 
    y se multiplican según el orden inverso de las operaciones elementales.
    Es decir, para \(k\) operaciones elementaes de permutación
    \[
        D_k \cdot D_{k - 1} \cdot \cdots \cdot D_2 \cdot D_1 = P
    \]
    Apartir del producto entre \(P\) y \(A\), se puede realizar el mismo procedimiento de la factorización \(LU\) para obtener a \(U\) y \(L\) respectivamente.
